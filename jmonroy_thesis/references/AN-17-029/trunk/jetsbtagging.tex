A complete reconstruction of the individual particles from
each collision event is obtained via the particle-flow (PF) algorithm.
The technique uses the information from all CMS sub-detectors to identify and
reconstruct individual particles in the collision
event~\cite{CMS-PAS-PFT-09-001, CMS-PAS-PFT-10-002}. The particles are
classified into mutually exclusive categories: charged hadrons,
neutral hadrons, photons, muons, and electrons.

\subsection{Jets and B-tagging}
Jets are reconstructed by clustering PF candidates using the
anti-$\mathrm{k_T}$ algorithm with distance parameter $\Delta
\mathrm{R}=0.4$ as implemented in the \textsc{fastjet} package~\cite{Cacciari:fastjet1,Cacciari:fastjet2}. 
The charged hadrons not coming from the primary vertices are subtracted from the
PF candidates considered in the clustering. The primary vertex is
chosen as the vertex with the highest sum of $\PT^2$ of its
constituent tracks. The prescribed jet energy corrections are applied as a function
of the jet $E_T$ and $\eta$~\cite{cmsJEC}. In addition, a multivariate
discriminator is applied to distinguish between jets coming from the
primary vertex and jets coming from pile-up vertices.  The
discrimination is based on the differences in the jet shapes, in the
relative multiplicity of charged and neutral components, and in the
different fraction of transverse momentum which is carried by the
hardest components.  Within the tracker acceptance the jet tracks are
also required to be compatible with the primary vertex. Jets are only
considered if they have a transverse energy above $25 \GeV$ and
$|\eta|<$2.4. In addition, they have to be separated from any lepton
candidates passing the Fakeable Object selection, described below, by requiring $\Delta \mathrm{R} = \sqrt{(\eta^{\ell} -
\eta^{jet})^{2} + (\phi^{\ell} - \phi^{jet})^{2}}>0.4$.

The CSVv2 b-tagging algorithm~\cite{Chatrchyan:2012jua} is used to
identify jets that are likely to originate from the hadronization of bottom
quarks.  This algorithm combines both secondary vertex information and
track impact parameter information together in a likelihood
discriminant. The discriminant output value ranges from
zero to one. It distinguishes between $b$-jets and jets
originating from light quarks, gluons and charm quarks.  The
efficiency to tag $b$-jets and the rate of misidentification of non-b
jets depend on the operating point chosen. Both the efficiency
and the fake rate are parameterised as a function of the transverse momentum and
pseudorapidity of the jets. These performance measurements are
obtained directly from data in samples that can be enriched in b jets,
such as $\ttbar$ and multijet events where a muon can be found inside the one of jets.
Two working points for the CSVv2 output discriminant  are used in the analysis. The
\emph{loose} one (CSVv2 $>$ 0.5426) has approximately ${85}$\% efficiency to tag jets
with $b$ quarks and a ${10}$\% chance to tag jets with only light quarks
or gluons.  The \emph{medium} working point (CSVv2 $>$ 0.8484) has
approximately ${70}$\% efficiency for tagging jets with $b$ quarks
and ${1.5}$\% efficiency to tag jets with only light quarks or gluons
~\cite{Chatrchyan:2012jua}.\\

%We correct for data/sim differences in the b-tagging performance by
%applying to the simulation per-jet weights dependent on the jet pt,
%eta, b-tagging discriminator and flavour (from simulation truth). 
%The weights are derived on ttbar and Z+jets events. 
%The per-event weight is taken as the product of the per-jet weight, 
%including those of the jets associated to the leptons.\\

