\section{Modeling of signal and background processes}\label{sec:experiment}
The \tHq\ and \tHW\ signal events are generated using \textsc{MG5\_}a\textsc{MC@NLO} (version 5.222)~\cite{amcatnlo} at leading-order precision, using the \textsc{MLM} merging scheme~\cite{Alwall:2007fs} and the \textsc{NNPDF3.0} PDF set~\cite{Ball:2014uwa}, and are normalized to next-to-leading order cross sections.
The \tHq\ events are generated with the four-flavor scheme while the \tHW\ process uses the five-flavor scheme to eliminate leading-order interference with the \ttH\ process~\cite{Demartin:2016axk}.
Event weights are produced in the generation of both samples to allow a reshaping of observables for 51 different coupling configurations: $17$ values of \Ct\ between $-3.0$ and $+3.0$ for three values of \CV: $+0.5$, $+1.0$, and $+1.5$, corresponding to 33 unique values of $\Ct/\CV$, ranging from $-6.0$ to $6.0$, and therefore 33 distinct kinematic configurations.

\textsc{MG5\_}a\textsc{MC@NLO} in NLO mode is used for the \ttH\ process and the main backgrounds: \ttW, \ttZ, \ttbar+jets, and $\ttbar\gamma$+jets, using parton-shower merging at NLO~\cite{Frederix:2012ps}.
Other minor backgrounds are simulated with different generators, such as \textsc{POWHEG}~\cite{Nason:2004rx,Frixione:2007vw,Alioli:2010xd,Re:2010bp,Alioli:2009je,Melia:2011tj} and \MADGRAPH\ at leading order (LO) QCD accuracy.
All generated events are interfaced to \textsc{Pythia8} (v8.205)~\cite{PYTHIA8} for the parton shower and hadronization steps.
Pileup interactions are simulated to reflect the observed multiplicity in data.
The simulated events are weighted according to the actual pileup in data, estimated from the measured bunch-to-bunch instantaneous luminosity and the total inelastic cross section, 69.2 mb.
All events are finally passed through a full simulation of the CMS detector based on \GEANTfour~\cite{GEANT}, and reconstructed using the same algorithms as used for the data.

Furthermore, the trigger selection is simulated and applied for generated signal events.
Residual differences in the trigger efficiency between data and MC are studied and corrected for, using the measured trigger efficiencies of the data.
