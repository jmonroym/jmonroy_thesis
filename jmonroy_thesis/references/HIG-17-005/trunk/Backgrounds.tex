\subsection{Non-prompt and charge mis-identified leptons}\label{sec:fakes}
The main contribution to the overall event yield in the signal selection, and one that can be reduced up to a certain point by tighter lepton selections, comes from processes with comparatively large cross sections in which one of the leptons is produced inside a jet (\ie\ it is non-prompt).
These are mostly real leptons from \cPqb\ hadron decays but also contain hadronic jets misidentified as leptons.
The yield of such events is estimated from a loose-to-tight extrapolation, in which a looser lepton selection is defined and the rate at which such leptons enter the tighter selection is measured in a control region and then used to extrapolate from a sideband with loose leptons to the signal selection with tight leptons.

The probability of a non-prompt lepton candidate passing a given loose selection to also pass the tight signal requirement is measured in a sample dominated by non-prompt leptons, as a function of \pt\ and $|\eta|$ and separately for muons and electrons.
The definitions of loose and tight leptons are given in Sec.~\ref{sec:objects}.
Two event samples are defined for the measurement of tight-to-loose ratios: one dominated by QCD multijet events, collected using single lepton triggers at relatively high \pt\ thresholds; and one dominated by $\Z+\mathrm{jets}$ events, where the two high \pt\ leptons from the \Z\ decay can be used to trigger the events without biasing the \pt\ spectrum of a third lepton at low transverse momentum.
The QCD-dominated sample is then used to extract ratios for lepton candidates with \pt\ above 30\GeV, whereas the ratios for low \pt\ leptons are determined in the $\Z+\mathrm{jets}$ sample.
For both regions, contributions from prompt leptons, mainly from \PW\ and $\Z+\mathrm{jets}$ or from \WZ\ and \ZZ\ events, respectively, are first suppressed by vetoing additional leptons in the selection, and the residual contamination is then subtracted using the transverse mass as a discriminating variable.  %Lepton-MET transverse mass?  Presumably, but would be good to say.

A sideband control region is then defined by relaxing the lepton selection criteria to ``loose'' (see Sec.~\ref{sec:objects}), while keeping all other selections equivalent to the full signal selection.
By weighting events in this expanded selection with a factor dependent on the measured tight-to-loose ratios, a fully data-driven estimation for the contribution of non-prompt leptons to the signal selection can be obtained.
In events where just one of the two leptons fails the tight criteria, the applied event weight is $f/(1-f)$ (where $f$ is the tight-to-loose ratio measured as described above), while events where both leptons fail the tight criteria are weighted by $-f_1f_2/[(1-f_1)(1-f_2)]$.
The resulting prediction of the event yield in the signal selection carries an uncertainty of 30--50\%, arising from the statistical uncertainty in the measurement of the tight-to-loose ratios, and from a systematic uncertainty derived by comparing alternative methods of subtracting prompt lepton backgrounds and from testing the closure of the method in simulated background events.

Similarly, background from events where the charge of one of the leptons is wrongly assigned---relevant only in the same-sign dilepton channels---are determined by measuring the charge mis-assignment probability in a sample of same-sign dilepton event compatible with a \Z\ boson decay and weighting events with opposite-sign leptons in the signal selection.
The charge mis-assignment probability is found to be negligible for this analysis for muons, whereas for electrons it ranges from about 0.02\% in the barrel section ($|\eta| < 1.48$) up to about 0.4\% in the detector endcaps ($ 1.48 < |\eta| < 2.5$).
It is measured separately in these two regions, and additionally as a function of the electron \pt.
A systematic uncertainty of 30\% is assigned to the prediction from the statistical uncertainty of the probability measurement and from testing the performance of the method on simulated events.
