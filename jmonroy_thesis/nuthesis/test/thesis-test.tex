%%
%% This is file `thesis-test.tex',
%% generated with the docstrip utility.
%%
%% The original source files were:
%%
%% nuthesis.dtx  (with options: `thesis-test')
%% 

\documentclass[ms,testing]{nuthesis}
%% Needed to typset the math in this sample
\usepackage{amsmath}
\usepackage{amsfonts}
%% Let's use a different font
\usepackage[sc,osf]{mathpazo}

%% Makes things look better
\usepackage{microtype}

%% Makes things look better
\usepackage{booktabs}

%% Gives us extra list environments
\usepackage{paralist}

%% Be able to include graphicsx
\usepackage{graphicx}

%% I like darker colors
\usepackage{color}
\definecolor{dark-red}{rgb}{0.6,0,0}
\definecolor{dark-green}{rgb}{0,0.6,0}
\definecolor{dark-blue}{rgb}{0,0,0.6}

%% If you use hyperref, you need to load memhfixc *after* it.
%% See the memoir docs for details.
\usepackage[%
pdfauthor={Ned W. Hummel},
pdftitle={Test Thesis},
pdfsubject={Thesis},
pdfkeywords={LaTeX, Thesis, University of Nebrska, Test},
linkcolor=dark-blue,
pagecolor=dark-green,
citecolor=dark-blue,
urlcolor=dark-red,
colorlinks=true,
backref,
plainpages=false,% This helps to fix the issue with hyperref with page numbering
pdfpagelabels% This helps to fix the issue with hyperref with page numbering
]{hyperref}

%% Needed by memoir to fix things with hyperref
\usepackage{memhfixc}
\begin{document}
%% Start formating the first few special pages
%% frontmatter is needed to set the page numbering correctly
\frontmatter

\title{Test}
\author{Ned W. Hummel}
\adviser{Professor Someone}
\adviserAbstract{Someone}
\major{\LaTeX}
\degreemonth{December}
\degreeyear{2007}
%%
%% For most people the defaults will be correct, so they are commented
%% out. To manually set these, just uncomment and make the needed
%% changes.
%% \college{Your college}
%% \city{Your City}
%%
%% For most people the following can be changed with a class
%% option. To manually set these, just uncomment the following and
%% make the needed changes.
%% \doctype{Thesis or Dissertation}
%% \degree{Your degree}
%% \degreeabbreviation{Your degree abbr.}
%%
%% Now that we know everything we need, we can generate the title page
%% itself.
%%
\maketitle
%% You have a maximum of 350, which includes your title, name, etc.
\begin{abstract}
  A simple test of using \textsf{nuthesis}, which demonstrates most
  of the options the class has.
\end{abstract}

%% Optional
\begin{copyrightpage}
This file may be distributed and/or modified under the conditions of
the \LaTeX{} Project Public License, either version 1.3c of this license
or (at your option) any later version.  The latest version of this
license is in:
\begin{center}
   \url{http://www.latex-project.org/lppl.txt}
\end{center}
and version 1.3c or later is part of all distributions of \LaTeX version
2006/05/20 or later.
\end{copyrightpage}

%% Optional
\begin{dedication}
  Arma virumque cano, Troiae qui primus ab oris Italiam, fato
  profugus, Laviniaque venit litora, multum ille et terris iactatus et
  alto vi superum saevae memorem Iunonis ob iram; multa quoque et
  bello passus, dum conderet urbem, inferretque deos Latio, genus unde
  Latinum, Albanique patres, atque altae moenia Romae.
\end{dedication}

%% Optional
\begin{acknowledgments}
  Arma virumque cano, Troiae qui primus ab oris Italiam, fato
  profugus, Laviniaque venit litora, multum ille et terris iactatus et
  alto vi superum saevae memorem Iunonis ob iram; multa quoque et
  bello passus, dum conderet urbem, inferretque deos Latio, genus unde
  Latinum, Albanique patres, atque altae moenia Romae.
\end{acknowledgments}

%% Optional
\begin{grantinfo}
  I'm not funded by any grants.
\end{grantinfo}
%% The ToC is required
%% Uncomment these if need be

%% The ToC is required
\tableofcontents
%% Uncomment these if need be
\listoffigures
\listoftables

%%   mainmatter is needed after the ToC, (LoF, and LoT) to set the
%%   page numbering correctly for the main body
\mainmatter

%% Thesis goes here
\chapter{The \AE{}NIED, Book I}\label{chap:aenied}

\section{Start}

Arma virumque cano, Troiae qui primus ab oris Italiam, fato profugus,
Laviniaque venit litora, multum ille et terris iactatus et alto vi
superum saevae memorem Iunonis ob iram; multa quoque et bello passus,
dum conderet urbem, inferretque deos Latio, genus unde Latinum,
Albanique patres, atque altae moenia Romae.\cite{virgil}

\subsection{Middle}

Dixit, et avertens rosea cervice refulsit, ambrosiaeque comae divinum
vertice odorem spiravere, pedes vestis defluxit ad imos, et vera
incessu patuit dea.  Ille ubi matrem adgnovit, tali fugientem est voce
secutus: `Quid natum totiens, crudelis tu quoque, falsis iudis
imaginibus?  Cur dextrae iungere dextram non datur, ac veras audire et
reddere voces?'\cite{virgil}

\subsubsection{End}

Postquam prima quies epulis, mensaeque remotae, crateras magnos
statuunt et vina coronant.  Fit strepitus tectis, vocemque per ampla
volutant atria; dependent lychni laquearibus aureis incensi, et noctem
flammis funalia vincunt.  Hic regina gravem gemmis auroque poposcit
implevitque mero pateram, quam Belus et omnes a Belo soliti; tum facta
silentia tectis: `Iuppiter, hospitibus nam te dare iura loquuntur,
hunc laetum Tyriisque diem Troiaque profectis esse velis, nostrosque
huius meminisse minores.  Adsit laetitiae Bacchus dator, et bona Iuno;
et vos, O, coetum, Tyrii, celebrate faventes.'  Dixit, et in mensam
laticum libavit honorem, primaque, libato, summo tenus attigit ore,
tum Bitiae dedit increpitans; ille impiger hausit spumantem pateram,
et pleno se proluit auro post alii proceres.  Cithara crinitus Iopas
personat aurata, docuit quem maximus Atlas.  Hic canit errantem lunam
solisque labores; unde hominum genus et pecudes; unde imber et ignes;
Arcturum pluviasque Hyadas geminosque Triones; quid tantum Oceano
properent se tinguere soles hiberni, vel quae tardis mora noctibus
obstet.  Ingeminant plausu Tyrii, Troesque sequuntur.  Nec non et
vario noctem sermone trahebat infelix Dido, longumque bibebat amorem,
multa super Priamo rogitans, super Hectore multa; nunc quibus Aurorae
venisset filius armis, nunc quales Diomedis equi, nunc quantus
Achilles.  `Immo age, et a prima dic, hospes, origine nobis insidias,'
inquit, `Danaum, casusque tuorum, erroresque tuos; nam te iam septima
portat omnibus errantem terris et fluctibus aestas.'\cite{virgil}

\chapter{Some Tables and Figures}

\begin{table}[h]
  \centering
  \begin{tabular}{ll}\toprule
    First & Last \\ \midrule
    Ned & Hummel \\
    Ned & Hummel \\
    Ned & Hummel \\ \bottomrule
  \end{tabular}
  \caption{Arma virumque cano, Troiae qui primus ab oris Italiam, fato profugus,
Laviniaque venit litora, multum ille et terris iactatus et alto vi
superum saevae memorem Iunonis ob iram}
  \label{tab:tabular}
\end{table}

\begin{table}[h]
  \centering

  \begin{compactitem}[\checkmark]
    \item Foo
    \item Foo
    \item Foo
    \end{compactitem}

  \caption{Arma virumque cano, Troiae qui primus ab oris Italiam, fato profugus,
Laviniaque venit litora, multum ille et terris iactatus et alto vi
superum saevae memorem Iunonis ob iram}
  \label{tab:list}
\end{table}

\begin{figure}[h]
  \centering
  \includegraphics[width=3in]{unl}
  \caption{Arma virumque cano, Troiae qui primus ab oris Italiam, fato profugus,
Laviniaque venit litora, multum ille et terris iactatus et alto vi
superum saevae memorem Iunonis ob iram}
  \label{fig:test}
\end{figure}

\chapter{Some Math}\label{chap:math}

This is a triviality, but we include it for completeness.
\begin{equation}
\int_0^\infty f(x) \, dx =
\begin{cases} 1 & \mbox{if $f=\delta$,} \\
0 & \mbox{if $f=0$.} \end{cases}
\end{equation}

Here is an aligned set of equations.
\begin{align}
f(x) &= f(x) \cdot 1 \\
     &= f(x) \cdot (2-1)\label{eq:fun}\\
     &= f(x)
\end{align}

The clever step is~\eqref{eq:fun}.

%% backmatter is needed at the end of the main body of your thesis to
%% set up page numbering correctly for the remainder of the thesis
\backmatter

%% Start the correct formatting for the appendices
\appendix

\chapter{Testing, 1, 2, 3, \ldots}

This has been a test of the thesis typesetting system.
Had this been an actual thesis, this would have been
preceded by an actual thesis.

%% Bibliography goes here (You better have one)
%% BibTeX is your friend
\bibliographystyle{plain}
\bibliography{nuthesis}
%% Pull in all the entries in the bibtex file. Is is a useful trick to
%% check all your references.
\nocite{*}

%% Index go here (if you have one)

\end{document}
\endinput
%%
%% End of file `thesis-test.tex'.
