\footnotesize
The exciting work in high energy physics includes not only the analysis of the data taken by the experiment but also the development of detection systems. In this thesis, the results of a search for the production of a Higgs boson in association with a single top quark (\tH) are presented. This process is of particular interest due to its sensitivity to the relative sign of the top-Higgs coupling and the vector bosons-Higgs coupling. The focus is on leptonic signatures provided by the $H \to WW$, $H \to \tau\tau$, and $H \to ZZ$ decay modes. 

%% %This sensitivity has its origin in the fact that the process can proceed through two ways where the Higgs boson is emitted either by the exchanged W boson or by the top quark; the strong interference of the two leading order Feynman diagrams makes the production cross section of this process highly sensitive to the relative sign of the top-Higgs coupling modifier, \Ct, and the coupling modifier of vector bosons to the Higgs, \CV.

%% %In an scenario where \Ct/\CV=-1.0, known as the inverted top coupling scenario (ITC), the cross section is increased by a factor of ten with respect to the standard model (SM) scenario where \Ct/\CV=-1.   

%% The analysis exploits signatures with two same-sign leptons or three leptons in the final state, and uses the 2016 data sample collected with the CMS detector at the LHC at a center of mass energy of 13 TeV
%% %, which corresponds to an integrated luminosity of 35.9 \fbinv
%% . Multivariate techniques are used to discriminate the signal from the dominant backgrounds. The analysis yields a 95\% confidence level (C.L.) upper limit on the combined \tH + \ttH production cross section times branching ratio of 0.64 pb, with an expected limit of 0.32 pb, for a scenario with \Ct = −1.0 and \CV = 1.0.
%% %Developments in Monte Carlo simulation methods allow the study of different scenarios in the \Ct-\CV phase space, thus, it was possible to exclude
%% Values of \Ct outside the range of -1.25 to +1.60 are exclude at 95\% C.L., assuming \CV = 1.0.

%% Sensitivity to CP mixing in the Higgs sector was investigated by considering scenarios for different values of the mixing angle $\alpha_{CP}$. Upper limits on the combined \tH + \ttH production cross section times branching ratio of 0.6 pb is set for a scenario with $\alpha_{CP}=180^o$ which corresponds to the scenario with \Ct = −1.0 and \CV = 1.0.

%% On the hardware side, contributions to the construction of the CMS forward pixel detector (FPix), responsible for tracking with extreme accuracy the paths of particles emerging from the proton-proton collisions at CMS. FPix is a modular detector composed of 672 modules built using a semiautomatic pick-and-place robotic system which integrates optical tools, pattern recognition algorithms, and glue dispensing subsystems, to locate the constituent module parts on the work field and glue them together with a precision of ~10 um. Fully assembled modules were tested and characterized.
The analysis exploits final states with two same-sign leptons or three leptons and uses the 2016 data sample collected with the Compact Muon Solenoid (CMS) detector at the Large Hadron Collider from proton-proton ($pp$) collisions at a center of mass-energy of 13 TeV. Multivariate techniques are used to discriminate the signal from the dominant backgrounds. The analysis yields a 95\% confidence level (C.L.) upper limit on the combined ${\tH + \ttH}$ production cross section times branching ratio of 0.64 pb, with an expected limit of 0.32 pb, for a scenario with \Ct = -1.0 and \CV = 1.0. Values of \Ct outside the range of -1.25 to +1.60 are excluded at 95\% C.L., assuming \CV = 1.0. Sensitivity to CP mixing in the Higgs sector was investigated by considering scenarios for different values of the mixing angle $\alpha_{CP}$. An upper limit on the combined \tH + \ttH production cross section times branching ratio of 0.6 pb is set for a scenario with $\alpha_{CP}=180^\circ$ which corresponds to the scenario with \Ct = -1.0 and \CV = 1.0.

On the detection systems side, contributions to the construction of the CMS forward pixel detector (FPix) are presented; FPix is responsible for tracking with extreme accuracy the paths of particles emerging from the $pp$ collisions at CMS. FPix is a modular detector composed of 672 modules built using a semiautomatic pick-and-place robotic system which integrates optical tools, pattern recognition algorithms, and glue dispensing subsystems to locate the constituent module parts on the work field and glue them together with a precision of ~10 $\mu$m. Fully assembled modules were tested and characterized.
