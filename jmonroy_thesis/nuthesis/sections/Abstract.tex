Multivariate techniques are used to discriminate the signal from the dominant backgrounds. The analysis yields a 95\% confidence level (C.L.) upper limit on the combined tH + ttH production cross section times b\
ranching ratio of 0.64 pb, with an expected limit of 0.32 pb, for a scenario with kt = −1.0 and kV = 1.0. Values of kt outside the range of −1.25 to +1.60 are excluded at 95\% C.L., assuming kV = 1.0.








%Novel semiconductor boron carbide films and boron carbide films doped with aromatic compounds have been investigated and characterized. Most of these semiconductors were formed by plasma enhanced chemical vapor deposition. The aromatic compound additives used, in this thesis, were pyridine (Py), aniline, and diaminobenzene (DAB). \\ %This growth process has been partly modeled through experimental investigations of the cation chemistry of the pertinent source molecules. 

%As one of the key parameters for semiconducting device functionality is the metal contact and, therefore, the chemical interactions or band bending that may occur at the metal/semiconductor interface, X-ray photoemission spectroscopy has been used to investigate the interaction of gold (Au) with these novel boron carbide-based semiconductors. Both n- and p-type films have been tested and pure boron carbide devices are compared to those containing aromatic compounds. The results show that boron carbide seems to behave differently from other semiconductors, opening a way for new analysis and approaches in device's functionality.\\

%By studying the electrical and optical properties of these films, it has been found that samples containing the aromatic compound exhibit an improvement in the electron-hole separation and charge extraction, as well as a decrease in the band gap. The hole carrier lifetimes for each sample were extracted from the capacitance-voltage, C(V), and current-voltage, I(V), curves. Additionally, devices, with boron carbide with the addition of pyridine, exhibited better collection of neutron capture generated pulses at ZERO applied bias, compared to the pure boron carbide samples. This is consistent with the longer carrier lifetimes estimated for these films.\\

%The I-V curves, as a function of external magnetic field, of the pure boron carbide films and films containing DAB demonstrate that significant room temperature negative magneto-resistance (> 100\% for pure samples, and >50\% for samples containing DAB) is possible in the resulting dielectric thin films. Inclusion of DAB is not essential for significant negative magneto-resistance, however, these results suggest practical device applications, especially as such effects are manifested in nanoscale films with facile fabrication. Overall, the greater negative magneto-resistance, when undoped with an aromatic, suggests a material with more defects and is consistent with a shorter carrier lifetime.
