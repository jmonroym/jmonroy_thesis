\hyphenation{ma-te-rials}
%%%%%%%%%%%%%%%%%%%%% Introduction %%%%%%%%%%%%%%%%%
\chapter{INTRODUCTION}
\label{ch:Intro}


\noindent Semiconducting boron carbides have been well known for their applications as semiconductors suitable for solid state neutron detectors \cite{Intro_1, Intro_2, Intro_3, Intro_4}, as well as their potential for spintronics and semiconductor devices in harsh environments \cite{Intro_5, Intro_6, Intro_7}. Their applications are limited mainly by the high defect concentrations and low carrier mobilities \cite{5, 6} of the typical semiconducting boron carbides, but they are found to be very robust with respect the radiation damage, i.e. they are self healing with modest amounts of radiation damage.\\


\noindent What is key to the general motivation behind this work is that modifications to boron carbide may result in a better boron carbide that might overcome the natural limitations of the more conventional boron carbides. If these devices can capture neutrons at zero applied bias, then neutron voltaics are possible, as well as low power neutron detectors. This is important, because, these semiconductors are commonly used as neutron detectors for identifying nuclear materials used in weapons, or even to detect and treat cancer \cite{cancer, cancer2}. \\

\noindent It is also known that not only defects in boron carbide based semiconductors materials can play an important role in the funcionality of the device \cite{Caretti, Cennignani, Pasquale_DBA, Pasquale_2}, but also the carbon to boron ratio  \cite{Sunwoo, Shirai, Park, Werheit}. Previous studies have shown that the boron carbide band gap decreases with increasing the carbon concentration, its values varies from 0.77 eV (highest carbon concentration) to 1.8 eV \cite{Sunwoo, Shirai}, affecting the conductivity of the sample and therefore the electric transport measurements in the boron carbide based samples \cite{Tallant}.\\


\noindent Novel boron carbide semiconductors, with aromatic moiety inclusions are seen \cite{Pasquale_Intro} to be good candidate materials to study neutron capture, as their properties, like the band gap, can be modified by varying the carbon concentration. On other hand, the presence of nitrogen in the aromatic compounds may marginally increase neutron capture cross-sections at very high neutron energies (584 keV), as seen in Fig. \ref{N}, where the strong boron capture cross-section falls sharply, without leading to an increase in cross-section to hard X-ray or gamma radiation, since nitrogen and boron are both low Z elements. Therefore, it seems natural to use aromatic compounds as dopings for these heterojunctions. \\

\begin{figure}[h!]
\centering
%\includegraphics[scale=0.3]{Nitrogen.png}
\caption{$^{14}N$ neutron capture in a PECVD polymerized pyridine film. The Q-value of this reaction is 626 keV. Taken from \cite{Tan}}\label{N}	
\end{figure}

\noindent Additionally, semiconducting boron carbides have been shown to be either p-type, in the absence of transition metal doping \cite{intro_2_Hwang,intro_2_Hwang2, intro_2_Hwang3, intro_2_Carlson}, or n-type \cite{intro_2_Peterson, intro_2_Robertson, intro_2_caruso}. In the case of a p-type boron carbide, the lifetimes are of about 30 $\mu s$, while these for n-type are 50 $ns$. Since these values are several orders of magnitude different, it seems interesting to study both types of conductivities. \\

\noindent Contacts are also an issue. In particular, Au/semiconductor interfaces will be studied to aid in the characterization of these novel semiconducting boron carbides.  \\
%with mobilities from $1.4 x 10^{-4}$ to $1.4 x 10^{-2}$ $cm^2/Vs$,  

%Different ratios of the aromatic compound to the boron carbide

%%\noindent  \\

\noindent In the study of any new semiconductor, a combination of surface science studies and transport studies can prove to be quite effective in developing a better understanding of the potential devices. Thus, chapter \ref{ch:Exp} will be devoted to describe these techniques, and how samples were growth. Next, in chapter \ref{ch:Contacts} X-ray photoemission spectroscopy will be used to study and characterize the meta/semiconductor interface. Then, chapter \ref{ch:Pyridine} will describe the effects of doping films with pyridine by studying the optical properties, the electrical responses, and the neutron capture of heterojunctions containing pyridine, and comparing these results to the pure boron carbide diodes, and the results from doping with benzene and aniline. Finally, magneto-resistance effects, by electrical measurements, will be discussed in chapter \ref{ch:MR}. 
%All of this provides more than
%significant motivation for studies that lead to a better understanding of the semiconductor and the
%semiconductor surface and its interface with contact metals. Better understanding of the materials
%can suggest routes to better boron carbides.


%the capacitance of a p-n homostructure diode or Schottky diode in the condition of zero or applied reverse bias is easily related to the total width of the depletion region within the diode
%______________________ References ______________________