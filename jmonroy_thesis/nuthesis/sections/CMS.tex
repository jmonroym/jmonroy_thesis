\chapter{The CMS experiment at the LHC}\label{ch:cms}

\section{Introduction}\label{sec:cms_intro}
\noindent Located in the Swiss-French border, the European Council for Nuclear Research (CERN) is the largest scientific organization leading the particle physics research. About 13000 people in a broad range of fields including users, students, scientists, engineers among others, contribute to the data taking and analysis, with the goal of unveiling the secrets of the nature and revealing the fundamental structure of the universe. CERN is also the home of the Large Hadron Collider (LHC), the largest circular particle accelerator around the world, where protons (or heavy ions) traveling close to the speed of light, are made to collide. These collisions open a window to investigate how particles (and their constituents if they are composite) interact with each other, providing clues about the laws of the nature. This chapter present an overview of the LHC structure and operation; a brief mention of the four main experiments that collect the information coming from the collisions is also included. A more detaled description of the Compact Muon Solenoid (CMS) detector is offered, given that the data used in this thesis have been taken with this detector.     

\section{The LHC}

\noindent With 27 km of circunference, the LHC is currently the largest and most powerful accelerator in the world. It is installed in the same tunnel where the large Electron-Positron (LEP) collider was located, taking advantage of the existing infraestructure. The LHC is also the larger accelerator in the CERN's accelerator complex and is assisted by several successive accelerating stages before the particles are injected into the LHC ring where they reach their maximum eneregy (see figure \ref{fig:cern}).

\begin{figure}[!h]
  \centering
  \includegraphics[width=0.7\textwidth]{cern_ac}
  \caption[CERN accelerator complex]{CERN accelerator complex. Blue arrows show the path followed by protons along the acceleration process \cite{cern}.}\label{fig:cern}
\end{figure}

\noindent LHC run in three modes depending on the particles being accelerated

\begin{itemize}
\item Proton-Proton collisions (pp) for multiple physics experiments.
\item Lead-Lead collisions (Pb-Pb) for heavy ion experiments. 
\item Proton-Lead collisions (p-Pb) for quark-gluon plasma experiments.
\end{itemize}

\begin{figure}[!h]
\centering
\includegraphics[width=4.5cm,height=3.3cm]{hbottle}
\includegraphics[width=6.0cm,height=3.3cm]{proton_source}\\
\includegraphics[width=4.5cm,height=3.3cm]{rfq2}
\includegraphics[width=6.0cm,height=3.3cm]{rfq3}
\caption[LHC protons source and first acceleration stage.]{LHC protons source and first acceleration stage. Top: the bottle contains hydrogen gas (white dots) which is injected into the metal cylinder to be broken down into electrons(blue dots) and protons(red dots); Bottom: the obtained protons are directed towards the radio frequency quadrupole which perform the first acceleration, focus the beam and create the bunches of protons.\cite{rfq2,video}}\label{fig:hbottle}
\end{figure}

\noindent In this thesis pp collisions will be considered.\\

\noindent Collection of protons starts with hydrogen atoms taken from a bottle, containing hydrogen gas, and injecting them in a metal cillinder; hydrogen atoms are broken down into electrons and protons by an intense electric field (see figure\ref{fig:hbottle} top). The resulting protons leave the metal cylinder towards a radio frecuency quadrupole (RFQ) that focus the beam, accelerate the protons and create the packets of protons called bunches. In the RFQ, an electric field is generated by a RF wave at a frecuency that matches the resonance frecuency of the cavity where the electrodes are contained. The beam of protons traveling on the RFQ axis experience an alternating electric field gradient that generates the focusing forces.\\

\noindent In order to accelerate the protons, a longitudinal time-variying electric field component is added to the system; it is done by giving the electrodes a sinus-like profile as shown in figure \ref{fig:hbottle} bottom. By matching the speed and phase of the protons with the longitudinal electric field the bunching is performed; protons synchronized with the RFQ (synchronous proton) does not feel an accelerating force, but those protons in the beam that have more (or less) energy than the synchronous proton (asynchronous protons) will feel a decelerating (accelerating) force; therefore, asynchronous protons will oscillate around the synchronous ones forming bunches of protons \cite{rfq}. From the RFQ emerges protons with energy 750 keV in bunches of about $1.15 \times 10^{11}$ protons\cite{lyndon}.        

\begin{figure}[!h]
  \centering
  \includegraphics[scale=0.5]{linac}
  \includegraphics[width=3.0cm,height=3.0cm]{linac2}
  \caption [The LINAC2 accelerating system at CERN.]{The LINAC2 accelerating system at CERN. Radio frecuency (RF) generated electric fields create acceleration and deceleration zones inside the cavity; deceleration zones are blocked by drift tubes where quadrupole magnets focus the proton beam.\cite{linac}}\label{fig:linac}
\end{figure}

\noindent Proton bunches coming from the RFQ goes to the linear accelerator 2 (LINAC2) where they are accelerated to reach 50 MeV energy. In the LINAC2 stage, acceleration is performed using radio frecuency generated electric fields which create zones of acceleration and deceleration as shown in figure \ref{fig:linac}. In the decelerations zones the electric field is blocked using drift tubes where protons are free to drift while quadrupole magnets focus the beam.\\   

\noindent The beam coming from LINAC2 is injected into the proton synchrotron booster (PSB) to reach 1.4 GeV in energy. The next acceleration is provided at the proton synchrotron (PS) up to 26 GeV, followed by the injection into the super proton synchrotron (SPS) where protons are accelerated to 450 GeV. Finally, protons are injected into the LHC where they are accelerated to the target energy of 6.5 TeV.
\noindent PSB, PS, SPS and LHC accelerate protons using the same RF acceleration technic described before. 

\begin{figure}[!h]
\centering
\includegraphics[scale=0.6]{lep}
\includegraphics[width=7cm,height=4.2cm]{lhc_rfc}
\includegraphics[scale=0.15]{rfc_lhc}
\caption[LHC layout and RF cavities module.]{Top: LHC layout. The red zones indicate the infrastructure additions to the LEP installations, built to accomodate the ATLAS and CMS experiments which exceed the size of the former experiments located there\cite{lep}. Bottom: LHC RF cavities. A module accomodates 4 cavities that accelerate protons and preserve the bunch structure of the beam.\cite{video,lhc_rfc}}\label{fig:lep_rfc}
\end{figure}

\noindent LHC have a system of 16 RF cavities located in the so-called point 4, as shown in figure \ref{fig:lep_rfc} top, tunned at a frecuency of 400 MHz and the protons are carefully timed so additionally to the acceleration effect the bunch structure of the beam is preserved. Bottom side of figure \ref{fig:lep_rfc} shows a picture of a Rf module composed of 4 RF cavities working in a superconducting state at 4.5 K; also is showed a representation of the accelerating electric field that accelerates the protons in the bunch.\\ 

\noindent While protons are accelerated in one section of the LHC ring, where the RF cavities are located, in the rest of their path they have to be kept in the curved trajectory defined by the LHC ring. Technically, LHC is not a perfect circle; RF, injection, beam dumping, beam cleaning and sections before and after the experimental points where protons collide are all straight sections. In total, there are 8 arcs 2.45 Km long each and 8 straight sections 545 m long each. In order to curve the proton's trajectory in the the arc sections, superconducting dipole magnets are used.\\               

\noindent Inside the LHC ring, there are two proton beams traveling in opposite directions in two separated beam pipes; the beam pipes are kept at ultra high vacuum ($\sim 10^{-9}$ Pa) to ensure that there are no particles that interact with the proton beams. The superconducting dipole magnets used in LHC are made of a NbTi alloy, capable of transporting currents of about $12000$ A when cooled at a temperature below 2K using liquid helium (see figure \ref{fig:lhcdipole}).

\begin{figure}[!h]
\centering
\includegraphics[width=0.7\textwidth]{lhcdipole}
\includegraphics[width=6.0cm,height=4cm]{lhc_dipole2}
\includegraphics[width=6.0cm,height=4cm]{beam_dev}
\caption [LHC dipole magnet.]{Top: LHC dipole magnet transversal view; cooling, shielding and mechanical support are indicated. Bottom left: Magnetic field generated by the dipole magnets; note that the direction of the field inside one beam pipe is opposite with respect to the other beam pipe which guarantee that both proton beams are curved in the same direction towards the ceneter of the ring. The effect of the dipole magnetic field on the proton beam is represented in the bottom right side \cite{lhc_dipole, dipole_field,video}.}\label{fig:lhcdipole}
\end{figure}

\noindent Protons in the arc sections of LHC feel a centripetal force exerted by the dipole magnets which is perpendicular to the beam trajectory; The magnitude of magnetic field needed can be found assuming that protons travel at $v \approx c$, using the standard values for proton mass and charge and the LHC radius, as
\beqn
F_m=\frac{mv^2}{r}=qBv \quad \to B=8.33 T
\eeqn
\noindent wich is about 100000 times the Earth's magnetic field. A representation of the magnetic field generated by the dipole magnets is shown in the bottom left side of figure \ref{fig:lhcdipole}. The bending effect of the magnetic field on the proton beam is shown in the bottom right side of figure \ref{fig:lhcdipole}. Note that the dipole magnets are not curved; the arc section of the LHC ring is composed of straight dipole magnets of about 15 m. In total there are 1232 dipole magnets along the LHC ring.

\noindent In addition to bending the beam trajectory, the beam has to be focused so it stays in side the beam pipe. The focusing is performed by quadrupole magnets installed in another straight section. Other effects like electromagnetic interaction among bunches, electron cluods from the beam pipe, gravitational force on the protons, differences in energy among protons in the same bunch among others, are corrected using sextupole magnets and other magnetic multipoles.     

\noindent The two proton beams inside the LHC ring are made of bunches with a cylindrical shape of $\approx 7.5$ cm long and $\approx 1$ mm in diameter when they are not close to the collision points (IP); close to the IP, the beam is focused up to a diameter of $\approx 16 \mu$m in order to maximize the expected number of collisions per unit area and per second which is know as luminosity(L). Luminosity can be calculated using

\beqn
L=fn\frac{N_1 N_2}{4\pi \sigma_x\sigma_y}
\eeqn

\noindent where f is the revolution frecuency, n is the number of bunches per beam,  $N_1$ and $N_2$ are the number of protons per bunch,  $\sigma_x$ and $\sigma_y$ are the gaussian transverse sizes of the bunches. Using

\begin{align}
  f=&\frac{v}{2\pi r_{LHC}}\approx\frac{3\times10^8m/s}{27km}\approx 11.1 kHz,\nonumber \\
  n=&2808\nonumber \\ 
  N_1=&N_2=1.5\times 10^{11}\nonumber\\
  \sigma_x=&\sigma_y=16\mu m\nonumber
\end{align}
\beqn
L= 1.28\times 10^{34} cm^{-2}s^{-1}
\eeqn

\noindent Luminosity is fundamental aspect for LHC given that the bigger luminosity, the bigger number of collisions which means that processes with a very small cross section the number of expected occurrencies is increased and so the chances of being detected. The integrated luminosty collected by the CMS experiment during 2016 is shown in figure \ref{fig:lumi}; the data analized in this thesis corresponds to an integrated luminosity of 35.9 fb$^{-1}$ at $\sqrt{s}=13$ TeV.   

\begin{figure}[!h]
\centering
\includegraphics[width=0.7\textwidth]{int_lumi_2016_cms}
\caption [2016 CMS Integrated luminosity]{Integrated luminosity delivered by LHC and recorded by CMS during 2016. The difference between the delivered and the recorded luminosities is due to fails and issues occured during the data taking in the CMS experiment\cite{lumi}.}\label{fig:lumi}
\end{figure}

Once the beams reach the desired energy, they are brought to cross each other producing proton-proton collisions. The bunch crossing happens in precise places where the LHC experiments are located. As seen in Figure \ref{lep}

The beams intersect at four points where collisions take place. In 2008, the first set of collisions involved protons with center-of-mass energy ($\sqrt{s}$) of 7 TeV; the energy was increased to 8 TeV in 2012 and to 13 TeV in 2015.



 the CMS (point 5) and ATLAS (point 1) experiments, which are multi-purpose experiments, enabled to explore physics in any of the collision modes. LHCb (point 8) experiment is optimized to explore B-physics, while ALICE (point 2) is optimized for heavy ion collisions researches; TOTEM (point 5) and LHCf (point 1) are dedicated to forward physics studies and MoEDAL (point 8) is intended for monopoles or massive pseudo stable particles searches.


\section{The CMS experiment}

The Compact Muon Solenoid (CMS) is a general purpose detector designed to conduct research in a wide range of physics from standard model to new physics like extra dimensions and dark matter. Located at the point 5 in the LHC layout as shown in Figure \ref{lep}, CMS is composed by several detection systems distributed in a cylindrical structure where the main feature is a solenoid magnet made of superconducting cable capable to generate a 3.8 T magnetic field. In total, CMS weight about 14000 tons in a very compact 21.6 m long and 14.6 m diameter cylinder (include areference for CMS TDR). It was built in 15 separated sections at the ground level and lowered to the cavern individually to be assembled. Figure \ref{cms} show the layout of the CMS detector (CMS TDR).          

\begin{figure}[!h]
  \centering
  \includegraphics[width=\textwidth]{cms}
  \caption {ref: CMS Collaboration, ``Detector Drawings'', CMS-PHO-GEN-2012-002, http://cds.cern.ch/record/1433717, 2012. }\label{cms}
\end{figure}


\subsection{coordinate system }
\subsection{tracker- pixels and strips }
\subsection{calorimeters}
\subsection{magnet }
\subsection{muon system }
\subsection{trigger system - HLT- L1 }
\subsection{ computing model}
\section{Event generation  simulation and reconstruction}
\subsection{ event generation}
\subsection{Hard scattering  }
\subsection{parton shower }
\subsection{hadronization and decays }
\subsection{underlying events and pileup }
\subsection{ MC - MadEvent, MadGraph and madgraph\@NLO, powheg, pythia, tauola}
\subsection{ detector simulation}
\subsection{event reconstruction- particle flow algorithm, vertexing , muon reco, electron reco, photon and hadron reco, jets reco, anti-kt algoritm, jet energy corrections, btagging, MET  }
\subsection{ MVA methods, NN, BDT, boosting, overtraining, variable ranking  }
\subsection{statistical inference, likelihood parametrization}
\subsection{ nuisance paraeters}
\subsection{exclusion limits }
\subsection{asymptotic limits }

