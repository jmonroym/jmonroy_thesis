\hyphenation{diffe-rent}
%%%%%%%%%%%%%%%%%%%%% Conclusions %%%%%%%%%%%%%%%%%
\chapter{CONCLUSIONS}
\label{ch:Conclusions}

\hspace{1cm} Metal/semiconductor interface formation was studied by X-ray photoemission spectroscopy. It was found that boron carbide-based semiconductors behave different from other semiconductor materials, since it presents Schottky barrier formation when working with p-type heterojunctions. Being the work function of the gold, the metal used for this study, bigger than this for the semiconductor, it is expected to see signatures of omhic contacts in the interface between the metal and the semiconductor, i.e., binding energies shifting to small values, or not change in the binding energy. However, shifting of the binding energies for the B(1s) and C(1s) core levels to high energies was found, can be explained by having a Schottky barrier formation in the interface.  The opposite situation is found when working with n-type boron carbide semiconductors and gold. No band bending signatures were found, therefore an ohmic contact is assigned to this surface interaction, confirming the unusual behavior of these heterojunctions. Inclusion of the aromatic compunds aniline seems to increase the Schottky barrier formed on the interface in the case of p-type heterojunctions. \\ 

\noindent \hspace{1cm} From electrical and optical measurements on these boron carbide-based heterojunctions, it was found that semiconducting boron carbides exhibit significantly enhanced electron-hole separation with inclusion of the aromatic compounds. For the case of \textit{ortho}-carborane boron carbide films, carrier lifetimes increase from 35 $\mu s$, for the pure boron carbide, to 350 $\mu s$ with pyridine inclusion, and even better to 2.5 $ms$ with benzene inclusion. The findings of substantially enhanced electron-hole separation and carrier lifetime in the doped films versus pure boron carbide films are certainly encouraging for the application of these materials as solid state neutron detectors. \\

\noindent \hspace{1cm} In the case of the addition of pyridine linking groups to PECVD semiconducting hydrogenated boron carbide films, synthesized on n-type silicon, it was found that charge collection increases, after neutron capture, in a heterojunction diode with silicon at zero bias, and the charge collection is much improved compared to heterojunction diodes fabricated by PECVD but without pyridine. The spatial overlap of the HOMO and LUMO states in cluster calculations, if applicable to the solid, suggests that exciton decay is facile, but hindered by symmetry constraints. \\ %(J. Phys. D: Appl. Phys. 49 355302)


\noindent \hspace{1cm} Finally, it was shown that semiconducting boron carbide polymers, formed by site-specific cross-linking of orthocarborane icosahedra with and without 1,4 diaminobenzene, exhibit significant negative magnetoresistive effect at room temperature. Values over 450\% negative magnetoresistance, depending on the bias voltage, are found for the pure boron carbide, while for samples with diaminobenzene doping it was about 100\%. Although inclusion of diaminobenzene does not improve the negative magnetoresistance values, other aromatic compounds need to be tested to determine whatever or not doping will affect the magnetoresistance of the films.  

