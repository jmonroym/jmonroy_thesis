% Customizable fields and text areas start with % >> below.
% Lines starting with the comment character (%) are normally removed before release outside the collaboration, but not those comments ending lines

% svn info. These are modified by svn at checkout time.
% The last version of these macros found before the maketitle will be the one on the front page,
% so only the main file is tracked.
% Do not edit by hand!
\RCS$Revision: 401995 $
\RCS$HeadURL: svn+ssh://svn.cern.ch/reps/tdr2/notes/AN-16-378/trunk/AN-16-378.tex $
\RCS$Id: AN-16-378.tex 401995 2017-05-02 11:44:29Z stiegerb $
%%%%%%%%%%%%% local definitions %%%%%%%%%%%%%%%%%%%%%
% This allows for switching between one column and two column (cms@external) layouts
% The widths should  be modified for your particular figures. You'll need additional copies if you have more than one standard figure size.
\newlength\cmsFigWidth
\ifthenelse{\boolean{cms@external}}{\setlength\cmsFigWidth{0.85\columnwidth}}{\setlength\cmsFigWidth{0.4\textwidth}}
\ifthenelse{\boolean{cms@external}}{\providecommand{\cmsLeft}{top\xspace}}{\providecommand{\cmsLeft}{left\xspace}}
\ifthenelse{\boolean{cms@external}}{\providecommand{\cmsRight}{bottom\xspace}}{\providecommand{\cmsRight}{right\xspace}}
\newcommand{\tHq}{\ensuremath{\cPqt\PH\cPq}}
\newcommand{\tHW}{\ensuremath{\cPqt\PH\PW}}
\newcommand{\tH}{\ensuremath{\cPqt\PH}}
\newcommand{\ttH}{\ensuremath{\ttbar\PH}}
\newcommand{\ttZ}{\ensuremath{\ttbar\Z}}
\newcommand{\ttW}{\ensuremath{\ttbar\PW}}
\newcommand{\ttV}{\ensuremath{\ttbar\mathrm{V}}}
\newcommand{\WW}{\ensuremath{\PW\PW}\xspace}
\newcommand{\WZ}{\ensuremath{\PW\Z}\xspace}
\newcommand{\ZZ}{\ensuremath{\Z\Z}\xspace}
\newcommand{\tautau}{\ensuremath{\Pgt\Pgt}\xspace}
\newcommand{\Zll}{\ensuremath{\mathrm{Z}\to\ell^+\ell^-}\xspace}
\newcommand{\Ztt}{\ensuremath{\mathrm{Z}\to\tau^+\tau^-}\xspace}
\newcommand{\reliso}{\ensuremath{I_\mathrm{rel}}\xspace}
\newcommand{\sip}{\ensuremath{S_\mathrm{IP3D}}\xspace}
\newcommand{\Pgth}{\ensuremath{\Pgt_{\rm h}}\xspace}
\newcommand{\ptRatio}{\ensuremath{\pt^\text{ratio}}\xspace}
\newcommand{\ptRel}{\ensuremath{\pt^\text{rel}}\xspace}
\newcommand{\relIso}{\ensuremath{I_\text{rel}}\xspace}
\newcommand{\miniIso}{\ensuremath{I_\text{mini}}\xspace}
\newcommand{\CV}{\ensuremath{\kappa_\text{V}}\xspace}
\newcommand{\Ct}{\ensuremath{\kappa_\cPqt}\xspace}
\newcommand{\ft}{\ensuremath{f_\cPqt}\xspace}
\newcommand{\mumu}{\ensuremath{\Pgm^\pm\Pgm^\pm}\xspace}
\newcommand{\emu}{\ensuremath{\Pe^\pm\Pgm^\pm}\xspace}
\newcommand{\ee}{\ensuremath{\Pe^\pm\Pe^\pm}\xspace}
\newcommand{\threel}{\ensuremath{\ell\ell\ell}\xspace}
%%%%%%%%%%%%%%%  Title page %%%%%%%%%%%%%%%%%%%%%%%%
\cmsNoteHeader{AN-16-378} % This is over-written in the CMS environment: useful as preprint no. for export versions
% >> Title: please make sure that the non-TeX equivalent is in PDFTitle below
\title{Search for tHq production in \\ multilepton final states at 13~TeV}

% >> Authors
%Author is always "The CMS Collaboration" for PAS and papers, so author, etc, below will be ignored in those cases
%For multiple affiliations, create an address entry for the combination
%To mark authors as primary, use the \author* form
\address[unl]{University of Nebraska-Lincoln}
\address[tifr]{Tata Institute for Fundamental Research, Mumbai}
\author[unl]{Jose Monroy}
\author[tifr]{Pallabi Das}
\author[unl]{Benjamin Stieger}
\author[tifr]{Sandhya Jain}
\author[unl]{Ken Bloom}
\author[tifr]{Kajari Mazumdar}

% >> Date
% The date is in yyyy/mm/dd format. Today has been
% redefined to match, but if the date needs to be fixed, please write it in this fashion.
% For papers and PAS, \today is taken as the date the head file (this one) was last modified according to svn: see the RCS Id string above.
% For the final version it is best to "touch" the head file to make sure it has the latest date.
\date{\today}

% >> Abstract
% Abstract processing:
% 1. **DO NOT use \include or \input** to include the abstract: our abstract extractor will not search through other files than this one.
% 2. **DO NOT use %**                  to comment out sections of the abstract: the extractor will still grab those lines (and they won't be comments any longer!).
% 3. For PASs: **DO NOT use tex macros**         in the abstract: CDS MathJax processor used on the abstract doesn't understand them _and_ will only look within $$. The abstracts for papers are hand formatted so macros are okay.
\abstract{
   This note presents a search for the associated production of a Higgs boson with a single top quark in multilepton (same-sign dilepton or three leptons) final states, targeting Higgs decay modes to WW, ZZ, and $\tau\tau$. The full 2016 LHC dataset at 13 TeV is used. Limits are evaluated as a function of the Higgs couplings to vector bosons and top quarks.
}

% >> PDF Metadata
% Do not comment out the following hypersetup lines (metadata). They will disappear in NODRAFT mode and are needed by CDS.
% Also: make sure that the values of the metadata items are sensible and are in plain text:
% (1) no TeX! -- for \sqrt{s} use sqrt(s) -- this will show with extra quote marks in the draft version but is okay).
% (2) no %.
% (3) No curly braces {}.
\hypersetup{%
pdfauthor={Jose Monroy, Benjamin Stieger, Pallabi Das},%
pdftitle={Search for tHq production in multilepton final states at 13 TeV},%
pdfsubject={CMS},%
pdfkeywords={CMS, physics, higgs, top}}

\maketitle %maketitle comes after all the front information has been supplied
% >> Text
%%%%%%%%%%%%%%%%%%%%%%%%%%%%%%%%  Begin text %%%%%%%%%%%%%%%%%%%%%%%%%%%%%
%% **DO NOT REMOVE THE BIBLIOGRAPHY** which is located before the appendix.
%% You can take the text between here and the bibiliography as an example which you should replace with the actual text of your document.
%% If you include other TeX files, be sure to use "\input{filename}" rather than "\input filename".
%% The latter works for you, but our parser looks for the braces and will break when uploading the document.
%%%%%%%%%%%%%%%




\tableofcontents
\clearpage

\section{Introduction}
\label{sec:intro}
The LHC Run I data have been exploited to measure all the accessible
properties of the newly-discovered Higgs
boson\,\cite{cms_higgs,atlas_higgs}. ATLAS and CMS have combined their
effort in order to reach an already very precise measurement of the
boson mass, $125.09\pm 0.21\,(\mathrm{stat.})\,\pm 0.11\,(\mathrm{syst.})$
GeV\,\cite{atlas_cms_mass}. This precise mass result has created an
opportunity to test the predictions of the standard model by measuring
the other properties of the Higgs boson. Measurements of the Higgs boson production and decay rates and
constraints on its couplings have been performed by both
experiments~\cite{atlas_couplings,cms_couplings},
and, in general, agreement with the SM predictions given the current uncertainties
(10-30~$\%$) have been found. It is of great interest to use
the 13 TeV LHC data to further constrain these measurements, as any deviation from
expectation could be a sign of new physics.\\

Among these measurements, it is of particular interest to measure the coupling of the Higgs
boson to the top quark ($\ttbar \PH$) because the top quark
could play a special role in the context of electroweak symmetry
breaking due to its large mass.  The Higgs boson does not decay to top quarks. The
$\mathrm{t \bar t H}$ interaction vertex, however, is present in a
rare production mechanism where the Higgs boson is produced in
association with a top quark-antiquark pair as shown in
Fig.~\ref{fig:feyn}.  At LHC energies the largest contribution to the standard
model Higgs boson production is a gluon-gluon induced loop
dominated by virtual top exchange. The comparison of a direct
measurement of the $\ttbar \PH$ coupling with the one inferred from
the cross section measurement can put limits on the contribution of
new physics to the gluon-gluon loop.\\

The $\ttbar \PH$ process has been used by both ATLAS and CMS experiments to directly measure the
top-Higgs coupling at tree level with the 20~fb$^{-1}$ of 8~TeV
collisions of the LHC Run I. Via this process, both experiments reached a
30$\%$ accuracy on the top Yukawa coupling, a great achivement
given that the production cross section (130~fb at 8~TeV at
next-to-leading order (NLO)~\cite{YellowReport}) was two orders of magnitude
lower with respect to the dominant Higgs production mode (gluon-gluon
fusion). In order to achive this result several decay channels of the
Higgs boson have been considered by both experiments, and three main searches have
been designed. The first channel searches for
$\mathrm{t \bar t H}$ in events where the Higgs boson decays to
$\bbbar$; the best fit value for the combined signal strength obtained
by the CMS experiment is $0.7^{+1.9}_{-1.9}$ (95\%
CL))~\cite{cms_ttH}. The second channel searches 
for $\mathrm{t \bar t H}$ in events where the Higgs boson decays to
$\gamma \gamma$; the best fit value for the combined signal strength obtained
by the CMS experiment is $2.7^{+2.6}_{-1.8}$ (95\%
CL))~\cite{cms_ttH}. \\ 

We designed the third search to probe $\mathrm{t \bar t
H}$ events where the Higgs boson decays into $\mathrm{ZZ}^{*}$,
$\mathrm{WW}^{*}$, or $\tau\tau$, with at least one Z, W or $\tau$
decaying leptonically. Despite
the small branching ratio, the presence of one
or two additional leptons from the top quark pair decays leads to the
following clean experimental signatures:
%
\begin{itemize}
\item two same-sign leptons
(electrons or muons) plus b-tagged jets;
\item three leptons plus b-tagged jets;
\item four leptons plus b-tagged jets.
\end{itemize}
Examples of Feynman diagrams for $\mathrm{t \bar t H}$,
followed by the decays of the top quark and the Higgs boson that lead to the signatures
described above are shown  in Fig.\,\ref{fig:feyn}.
With this search we obtained the most precise measurement of the $\mathrm{t \bar t
H}$ signal strength: $3.7^{+1.9}_{-1.9}$
(95\%CL))~\cite{cms_multilepton}.\\

The combined best-fit signal strength
obtained assuming a Higgs boson mass of $125 \GeV$ was $\mu =
2.9^{+1.1}_{-0.9}$.  This result corresponds to a 3.5 standard
deviation excess over the background-only ($\mu = 0$) hypothesis, and
represents a 2.1 standard deviation upward fluctuation on the SM
$\ttbar \PH$ ($\mu = 1$) expectation. Although the combined observed signal strength is consistent with SM
expectations, with a roughly 2 standard deviation upward
fluctuation, it is interesting to point out that the excess was mainly
driven by the multilepton analysis, and in particular by the same-sign
di-muon subsample~\cite{cms_multilepton}.\\


With respect to 8~TeV, the 13~TeV $\mathrm{t \bar t H}$ cross section increased by a factor of 4 
with the higher center of mass energy, while the cross sections of the main backgrounds
$\ttbar\PW$, $\ttbar\Z$, $\ttbar$+jets increased by roughly a factor of 3.
We thus expect to increase our sensitivity during Run II, compared to Run I.

The first multilepton search at 13~TeV analyzed 2.3~fb$^{-1}$ of the 2015 dataset.
It measured the expected 95\% confidence level upper limit on the Higgs boson production cross section for a Higgs boson mass of 125 GeV/c$^2$
to be $2.6$ times the standard model expectation, compared to the observed limit of $3.3$.
The signal strength $\mu$, relative to the expectation for the standard model
Higgs boson, was measured to be $0.6_{-1.1}^{+1.4}$~\cite{cms_multilepton_2015}.\\

The 2016 data has been preliminarly analysed for the ICHEP conference considering 12.9~fb$^{-1}$ \cite{cms_multilepton_2016ICHEP}.
The results have been combined with the 2015 dataset and yield
a $\mathrm{t \bar t H}$ signal strength of $2.0_{-0.7}^{+0.8}$ times the standard model prediction.
They are used to set a 95\% confidence level upper limit on the signal production cross section of 3.4
times the standard model expectation, compared to an expected upper limit of $1.3_{-0.4}^{+0.6}$ in the absence of a signal.

In this note we perform the $\mathrm{t \bar t H}$ multilepton search with the full 2016 data, corresponding to 36.9 ~fb$^{-1}$,
collected by the CMS experiment at $\sqrt{s}$ = 13 TeV.
The general strategy remains similar to the previous searches.
Multivariate analysis techniques are used to identify objects with high purity and
to distinguish background from signal events.
The amount of signal is fit to the multivariate discriminant output distribution in all the final states simultaneously. \\

\begin{figure}[htb]
\centering
\includegraphics[width=0.30\linewidth]{diagrams/gg-ttH-tt-2lss.pdf} 
\hspace{0.5cm}
\includegraphics[width=0.30\linewidth]{diagrams/gg-ttH-ZZ-3l.pdf}
\hspace{0.5cm}
\includegraphics[width=0.30\linewidth]{diagrams/gg-ttH-WW-4l.pdf}
\caption{Examples of leading order Feynman diagrams for $t\bar{t}H$ production at pp colliders, with the Higgs boson decaying to
$\tau\tau$, $\mathrm{ZZ}^{*}$, and
$\mathrm{WW}^{*}$ (from left to right). The first, second, and third diagrams are examples of the two same-sign lepton signature,
the three lepton signature, and the four lepton signature, respectively.} 
\label{fig:feyn}
\end{figure}

\clearpage

\clearpage

\section{Data and MC Samples}
\label{sec:samples}
\input{samples}
\clearpage

\section{Object identification and event selection}
\label{sec:objects}
\input{objects}
The event selection aims at rejecting events that do not match the decay signatures targeted by this analysis. We require that at least two leptons passing the tight selection are present in the event. Moreover, events where a pair of loose leptons with an invariant mass smaller than 12\GeV is found are rejected, as they are not modeled by the simulation.

For all events passing the selection, we require at least two jets with transverse momentum greater than 25\GeV be reconstructed in the $|\eta|<2.4$ region.
We also require that both jets satisfy the loose working point of the CSV $\cPqb$-tag algorithm, or that at least one of them satisfies the medium working point,
as a top quark pair decaying into $\cPqb$-jets is present in all signal events.

\subsection{Two lepton same-sign category}

In events where no additional tight lepton with a transverse momentum greater than 10\GeV is present, we require that the two tight leptons have the same charge and transverse momenta greater than 25\GeV and 15\GeV respectively. These events constitute the two lepton same-sign (\textit{2lss}) category of the analysis. % If the sub-leading lepton is an electron, and only for this category, its transverse momentum requirement is tightened to 15\GeV.

In addition to the requirements described above, we discard 2lss events that contain less than four jets with transverse momentum greater than 25\GeV and $|\eta|<2.4$ in the final state.

The event is also rejected if the two selected leptons do not pass the requirements aimed at rejecting leptons from conversions and those on the quality of the charge measurement described in Section~\ref{sec:objects}.
The background from electrons from $\Z$ decays, where the charge of one electron is mismeasured, is further reduced by vetoing events where the di-electron invariant mass differs by less than 10\GeV from the $\Z$ mass. For the same reason, we also require that the $E_\mathrm{T}^\mathrm{miss}LD$ variable is larger than 0.2 in di-electron events. 

\subsection{Three and four lepton categories}
The three lepton (\textit{3l}) category consists of events that contain exactly three tight leptons with a transverse momenta greater than 25, 15 and 15 \GeV respectively. 
%No requirement is applied on the possible presence of an even larger number of leptons; i.e., this category implicitly includes events with four or more leptons.
In order to reject backgrounds from processes with $\Z$ bosons in the final state, we require that no pair of same-flavor opposite-sign loose leptons has an invariant mass closer than 10\GeV to the mass of the $\Z$ boson. We then add an $E_\mathrm{T}^\mathrm{miss}LD > 0.2$ requirement.
The $E_\mathrm{T}^\mathrm{miss}LD$ threshold is tighter (0.3)
if the event has a pair of leptons with the same flavor and opposite sign. 
For events with large jet multiplicity ($\geq$ 4 jets), where the contamination
from the $\Z$ background is smaller, no requirement
on $E_\mathrm{T}^\mathrm{miss}LD$ is applied.
The event is also rejected if the three selected leptons do not pass the conversion veto requirements, or if the sum of their charges is not equal to +1 or -1.
We further veto events with two OSSF pairs and their 4 lepton mass lower than 140 GeV, where the leptons pass the loose identification.

The four lepton (\textit{4l}) category is defined exactly as the (\textit{3l}) category, except that it requires the presence of 
at least four tight leptons with a transverse momenta greater than 25, 15, 15, and 10 \GeV respectively.

%The observed event yields in data for each final state and the
%expectations from the different physical processes are summarised in
%table~\ref{tab:yields-sel}.

%\begin{table}[thb]
%\centering
%\begin{tabular}{l@{\qquad}rrr@{\qquad}r}\hline
%\multicolumn{1}{c@{\qquad}}{} &
%\multicolumn{1}{c}{$\Pgm\Pgm$} &
%\multicolumn{1}{c}{$\Pe\Pe$} &
%\multicolumn{1}{c@{\qquad}}{$\Pe\Pgm$} &
%\multicolumn{1}{c@{\qquad}}{$3\ell$} \\ \hline
%
%$\ttbar\,\PW$                 & $ 18.2 \pm 0.9   $&$  6.9 \pm 0.6       $&$ 24.6  \pm 1.1    $&$ 12.3   \pm 0.7   $ \\
%$\ttbar\,\Z\!/\!\gamma^*$     & $  5.8 \pm 0.6   $&$  7.5 \pm 0.6       $&$ 15.3  \pm 1.3    $&$ 22.7   \pm 1.0   $ \\ \hline
%di-boson                      & $  1.4 \pm 0.2   $&$  1.1 \pm 0.2       $&$  2.6 \pm  0.3    $&$ 5.7   \pm  0.4   $ \\
%$\mathrm{tttt}$               & $  0.8 \pm 0.2   $&$  0.4 \pm 0.1       $&$  1.5 \pm  0.2    $&$ 1.2   \pm  0.1   $ \\
%$\mathrm{tqZ}$                & $  0.2 \pm 0.3   $&$  0.4 \pm 0.4       $&$  0.6 \pm  0.6    $&$ 2.7   \pm  0.8   $ \\
%rare backgrounds              & $  1.6 \pm 0.3   $&$  0.5 \pm 0.1       $&$  1.8 \pm  0.1    $&$ 0.3   \pm  0.1   $ \\ \hline \hline      
%non-prompt (MC)               & $ 24.0 \pm 1.3   $&$  11.1 \pm 1.0      $&$  41.5 \pm 2.0    $&$ 28.3  \pm  1.5   $ \\ 
%charge mis-ID (MC)            &                   &$  5.8 \pm 2.3       $&$  6.6  \pm 1.7    $&$                  $ \\  \hline
%non-prompt (data-driven)      & $ 33.4 \pm 1.2   $&$ 23.1 \pm 1.1       $&$  61.9 \pm 1.7    $&$ 51.0   \pm 1.8   $ \\
%charge mis-ID (data-driven)   &                   &$  6.7 \pm 0.1       $&$  10.0 \pm 0.1    $&                     \\ \hline \hline
%all backgrounds (MC)          & $ 52.0 \pm 1.8   $&$  33.8 \pm 2.7      $&$  94.6 \pm 3.2    $&$ 73.2  \pm 2.1    $ \\
%all backgrounds (data-driven) & $ 61.3 \pm 1.7   $&$  46.6 \pm 1.5      $&$  118.2 \pm 2.5   $&$ 95.9   \pm 2.3   $ \\ \hline \hline
%$\ttH$                        & $  8.0 \pm 0.2   $&$  3.4 \pm 0.1       $&$  11.3 \pm 0.2    $&$ 10.8   \pm 0.2   $ \\ \hline \hline
%data                          &    74                    &   45                      &    154         &    105      \\ \hline
%\end{tabular}
%\caption{Expected and observed yields after the selection in all final states. The rare SM backgrounds include triboson, $\PW^\pm\PW^\pm\mathrm{qq}$, and $\PW\PW$ produced in double-parton interactions. Uncertainties are statistical only.}
%\label{tab:yields-sel}
%\end{table}

Figures~\ref{fig:2l_lepPt}-\ref{fig:2l_metLD} and \ref{fig:3l_lepPt}-\ref{fig:3l_ht} show the main event observables (lepton and jet multiplicities and spectra, energy sums) for events passing the $2\ell$ and $3\ell$. Figure \ref{fig:4l_numevt} shows the number of events in the $4\ell$ category.

%%%%%
%%   2l plots
%%%%%

%\begin{figure}[htb]
%	\centering 
%\includegraphics[width=0.32\textwidth]{plots_leptons/lep_evtsel/2lss_SR/mm/nLepFO.pdf}
%\includegraphics[width=0.32\textwidth]{plots_leptons/lep_evtsel/2lss_SR/ee/nLepFO.pdf}
%\includegraphics[width=0.32\textwidth]{plots_leptons/lep_evtsel/2lss_SR/em/nLepFO.pdf}
%	\caption{Number of leptons passing the fakeable object requirements in the 2$\ell$ ($\Pgm\Pgm$, $\Pe\Pe$, $\Pe\Pgm$) selections.}
%	\label{fig:2l_nLepFO}
%\end{figure}

%\begin{figure}[htb]
%	\centering 
%\includegraphics[width=0.32\textwidth]{plots_leptons/lep_evtsel/2lss_SR/mm/nTauTight}
%\includegraphics[width=0.32\textwidth]{plots_leptons/lep_evtsel/2lss_SR/ee/nTauTight}
%\includegraphics[width=0.32\textwidth]{plots_leptons/lep_evtsel/2lss_SR/em/nTauTight}
%	\caption{Number of reconstructed $\Pgth$ leptons passing the requirements described in Section~\ref{subsec:taus} in the 2$\ell$ ($\Pgm\Pgm$, $\Pe\Pe$, $\Pe\Pgm$) selections.}
%	\label{fig:2l_nTau}
%\end{figure}

\begin{figure}[htb]
	\centering 
\includegraphics[width=0.32\textwidth]{plots_leptons/lep_evtsel/2lss_SR/mm/kinMVA_input_LepGood0_conePt.pdf}
\includegraphics[width=0.32\textwidth]{plots_leptons/lep_evtsel/2lss_SR/ee/kinMVA_input_LepGood0_conePt.pdf}
\includegraphics[width=0.32\textwidth]{plots_leptons/lep_evtsel/2lss_SR/em/kinMVA_input_LepGood0_conePt.pdf}
\includegraphics[width=0.32\textwidth]{plots_leptons/lep_evtsel/2lss_SR/mm/kinMVA_input_LepGood1_conePt.pdf}
\includegraphics[width=0.32\textwidth]{plots_leptons/lep_evtsel/2lss_SR/ee/kinMVA_input_LepGood1_conePt.pdf}
\includegraphics[width=0.32\textwidth]{plots_leptons/lep_evtsel/2lss_SR/em/kinMVA_input_LepGood1_conePt.pdf}
	\caption{Lepton transverse momentum spectra in the 2$\ell$ ($\Pgm\Pgm$, $\Pe\Pe$, $\Pe\Pgm$) selections.}
	\label{fig:2l_lepPt}
\end{figure}
\begin{figure}[htb]
	\centering 
\includegraphics[width=0.32\textwidth]{plots_leptons/lep_evtsel/2lss_SR/mm/2lep_mll.pdf}
\includegraphics[width=0.32\textwidth]{plots_leptons/lep_evtsel/2lss_SR/ee/2lep_mll.pdf}
\includegraphics[width=0.32\textwidth]{plots_leptons/lep_evtsel/2lss_SR/em/2lep_mll.pdf}
	\caption{Di-lepton invariant mass spectra in the 2$\ell$ ($\Pgm\Pgm$, $\Pe\Pe$, $\Pe\Pgm$) selections.}
	\label{fig:2l_mll}
\end{figure}

\begin{figure}[htb]
	\centering 
\includegraphics[width=0.32\textwidth]{plots_leptons/lep_evtsel/2lss_SR/mm/2lep_charge.pdf}
\includegraphics[width=0.32\textwidth]{plots_leptons/lep_evtsel/2lss_SR/ee/2lep_charge.pdf}
\includegraphics[width=0.32\textwidth]{plots_leptons/lep_evtsel/2lss_SR/em/2lep_charge.pdf}
	\caption{Sum of lepton charges in the 2$\ell$ ($\Pgm\Pgm$, $\Pe\Pe$, $\Pe\Pgm$) selections.}
	\label{fig:2l_charge}
\end{figure}


\begin{figure}[htb]
	\centering 
\includegraphics[width=0.32\textwidth]{plots_leptons/lep_evtsel/2lss_SR/mm/2lep_nJet25_from4.pdf}
\includegraphics[width=0.32\textwidth]{plots_leptons/lep_evtsel/2lss_SR/ee/2lep_nJet25_from4.pdf}
\includegraphics[width=0.32\textwidth]{plots_leptons/lep_evtsel/2lss_SR/em/2lep_nJet25_from4.pdf}
	\caption{Jet multiplicity in the 2$\ell$ ($\Pgm\Pgm$, $\Pe\Pe$, $\Pe\Pgm$) selections.}
	\label{fig:2l_nJet}
\end{figure}

\begin{figure}[htb]
	\centering 
\includegraphics[width=0.32\textwidth]{plots_leptons/lep_evtsel/2lss_SR/mm/nBJetLoose25.pdf}
\includegraphics[width=0.32\textwidth]{plots_leptons/lep_evtsel/2lss_SR/ee/nBJetLoose25.pdf}
\includegraphics[width=0.32\textwidth]{plots_leptons/lep_evtsel/2lss_SR/em/nBJetLoose25.pdf}
	\caption{Multiplicity of jets passing the loose working point of the CSV tagger in the 2$\ell$ ($\Pgm\Pgm$, $\Pe\Pe$, $\Pe\Pgm$) selections.}
	\label{fig:2l_nBJetLoose}
\end{figure}

\begin{figure}[htb]
	\centering 
\includegraphics[width=0.32\textwidth]{plots_leptons/lep_evtsel/2lss_SR/mm/nBJetMedium25.pdf}
\includegraphics[width=0.32\textwidth]{plots_leptons/lep_evtsel/2lss_SR/ee/nBJetMedium25.pdf}
\includegraphics[width=0.32\textwidth]{plots_leptons/lep_evtsel/2lss_SR/em/nBJetMedium25.pdf}
	\caption{Multiplicity of jets passing the medium working point of the CSV tagger in the 2$\ell$ ($\Pgm\Pgm$, $\Pe\Pe$, $\Pe\Pgm$) selections.}
	\label{fig:2l_nBJetMedium}
\end{figure}


\begin{figure}[htb]
	\centering 
\includegraphics[width=0.32\textwidth]{plots_leptons/lep_evtsel/2lss_SR/mm/htJet25j.pdf}
\includegraphics[width=0.32\textwidth]{plots_leptons/lep_evtsel/2lss_SR/ee/htJet25j.pdf}
\includegraphics[width=0.32\textwidth]{plots_leptons/lep_evtsel/2lss_SR/em/htJet25j.pdf}
	\caption{$H_T$ spectra in the 2$\ell$ ($\Pgm\Pgm$, $\Pe\Pe$, $\Pe\Pgm$) selections.}
	\label{fig:2l_ht}
\end{figure}
\begin{figure}[htb]
	\centering 
\includegraphics[width=0.32\textwidth]{plots_leptons/lep_evtsel/2lss_SR/mm/kinMVA_input_met.pdf}
\includegraphics[width=0.32\textwidth]{plots_leptons/lep_evtsel/2lss_SR/ee/kinMVA_input_met.pdf}
\includegraphics[width=0.32\textwidth]{plots_leptons/lep_evtsel/2lss_SR/em/kinMVA_input_met.pdf}
	\caption{$E_\mathrm{T}^\mathrm{miss}$ spectra in the 2$\ell$ ($\Pgm\Pgm$, $\Pe\Pe$, $\Pe\Pgm$) selections.}
	\label{fig:2l_met}
\end{figure}
\begin{figure}[htb]
	\centering 
\includegraphics[width=0.32\textwidth]{plots_leptons/lep_evtsel/2lss_SR/mm/metLD.pdf}
\includegraphics[width=0.32\textwidth]{plots_leptons/lep_evtsel/2lss_SR/ee/metLD.pdf}
\includegraphics[width=0.32\textwidth]{plots_leptons/lep_evtsel/2lss_SR/em/metLD.pdf}
	\caption{$E_\mathrm{T}^\mathrm{miss}LD$ spectra in the 2$\ell$ ($\Pgm\Pgm$, $\Pe\Pe$, $\Pe\Pgm$) selections.}
	\label{fig:2l_metLD}
\end{figure}


%%%%%
%%   3l plots
%%%%%

%\begin{figure}[htb]
%	\centering 
%\includegraphics[width=0.32\textwidth]{plots_leptons/lep_evtsel/3l_SR/nLepFO.pdf}
%%\includegraphics[width=0.32\textwidth]{plots_leptons/lep_evtsel/3l_SR/nTauTight.pdf}
%\includegraphics[width=0.32\textwidth]{plots_leptons/lep_evtsel/3l_SR/3lep_charge.pdf}
%	\caption{Number and charge of leptons passing the fakeable object requirements and $\Pgth$ leptons passing the requirements described in Section~\ref{subsec:taus} in the 3$\ell$ selection.}
%	\label{fig:3l_nLepFO}
%\end{figure}

\begin{figure}[htb]
	\centering 
\includegraphics[width=0.32\textwidth]{plots_leptons/lep_evtsel/3l_SR/kinMVA_input_LepGood0_conePt.pdf}
\includegraphics[width=0.32\textwidth]{plots_leptons/lep_evtsel/3l_SR/kinMVA_input_LepGood1_conePt.pdf}
\includegraphics[width=0.32\textwidth]{plots_leptons/lep_evtsel/3l_SR/kinMVA_input_LepGood2_conePt.pdf}
	\caption{Lepton transverse momentum spectra in the 3$\ell$ selection.}
	\label{fig:3l_lepPt}
\end{figure}

\begin{figure}[htb]
	\centering 
\includegraphics[width=0.32\textwidth]{plots_leptons/lep_evtsel/3l_SR/3lep_nJet25.pdf}
\includegraphics[width=0.32\textwidth]{plots_leptons/lep_evtsel/3l_SR/nBJetLoose25.pdf}
\includegraphics[width=0.32\textwidth]{plots_leptons/lep_evtsel/3l_SR/nBJetMedium25.pdf}
	\caption{Jet multiplicities (all jets, jets passing the loose working point of the CSV tagger, jets passing the medium working point of the CSV tagger) in the the 3$\ell$ selection.}
	\label{fig:3l_nJet}
\end{figure}

\begin{figure}[htb]
	\centering 
\includegraphics[width=0.32\textwidth]{plots_leptons/lep_evtsel/3l_SR/htJet25j.pdf}
\includegraphics[width=0.32\textwidth]{plots_leptons/lep_evtsel/3l_SR/met.pdf}
\includegraphics[width=0.32\textwidth]{plots_leptons/lep_evtsel/3l_SR/metLD.pdf}
	\caption{$H_T$, $E_\mathrm{T}^\mathrm{miss}$ and $E_\mathrm{T}^\mathrm{miss}LD$ distributions in the 3$\ell$ selection.}
	\label{fig:3l_ht}
\end{figure}

%%%%%
%%   4l plots
%%%%%
\begin{figure}[htb]
	\centering 
\includegraphics[width=0.32\textwidth]{plots_leptons/lep_evtsel/4l_SR/tot_weight.pdf}
	\caption{Number of events selected in the $4\ell$ category.}
	\label{fig:4l_numevt}
\end{figure}

%% flush plots
\clearpage

\clearpage

\section{Background predictions}
\label{sec:backgrounds}
\input{backgrounds}
\clearpage

\section{Signal discrimination}
\label{sec:sigdisc}
\input{signaldiscr}
\clearpage

\section{Signal extraction}
\label{sec:extraction}
\subsection{Signal extraction procedure}
The two BDT classifiers introduced in the previous section, trained against the dominant \ttbar\ and \ttV\ backgrounds in each channel, are used to evaluate the limits in a fit to the classifier shape.
Figure~\ref{fig:mva12} shows the expected output distributions in a 2D plane of one training vs.\ the other.
Each event is now classified into one of ten 2D-bins according to its position in the plane, see Fig.~\ref{fig:binning}.
The number of bins is chosen such that no bins are entirely empty for any process.
The bin boundary positions have been studied and optimized with respect to the expected limit, see Sec.~\ref{sec:binopt}.

\begin{figure} [!h]
 \centering
 \includegraphics[width=0.45\textwidth]{figures/hthq.pdf}
 \includegraphics[width=0.45\textwidth]{figures/hthw.pdf}\\
 \includegraphics[width=0.45\textwidth]{figures/hbg.pdf}
 \includegraphics[width=0.45\textwidth]{figures/hratio.pdf}
\caption{BDT classifier output planes (training vs \ttbar\ on x-axis and vs \ttV\ on y-axis) for the \tHq\ and \tHW\ signals (top row), and for the combined backgrounds (bottom left). Backgrounds are evaluated as in the final background prediction, \ie\ these are not the samples used in the MVA training and this includes data-driven backgrounds. Bottom right shows the S/B ratio (combining \tHq\ and \tHW) in the same plane. Three lepton channel only.}
\label{fig:mva12}
\end{figure}

\begin{figure} [!h]
 \centering
 \includegraphics[width=0.8\textwidth]{figures/hratio_binning.pdf}
\caption{Binning overlaid on the S/B ratio map on the plane of classifier outputs.}
\label{fig:binning}
\end{figure}

From this event categorization, a 1D histogram of expected distribution is produced for each signal and background process, and fit to the observed data (or the Asimov dataset for expected limits).

\subsection{Signal model}
The goal of this analysis is to test the compatibility of points in the parameter space of Higgs-to-vector boson and Higgs-to-top quark couplings.
The simulated \tHq, \tHW, and \ttH\ signal events are used with event-by-event weights to reflect the impact of the couplings on kinematic distributions, and together with different predictions of the respective production cross sections and branching ratios, we can produce limits for different values of \CV\ and \Ct.
(See Tab.~\ref{tab:reweight} for the set of \Ct\ and \CV\ values generated.)
The slight shape-dependence of the BDT outputs as a function of the couplings is documented in App.~\ref{sec:bdtvscvct}.

Apart from the \Ct/\CV\ interference of the \tHq\ and \tHW\ production cross sections, the cross section of \ttH\ scales as $\Ct^2$.
Furthermore, the Higgs branching fractions to vector bosons depend on \CV, and the overall Higgs decay width depend both on \Ct\ and \CV\ when considering resolved top-quark loops in the $\PH\to\gamma\gamma$, $\PH\to\Z\gamma$, and $\PH\to\Pg\Pg$ decays.
The relative contributions from $\PH\to\WW$, $\PH\to\ZZ$, and $\PH\to\tautau$ changes with changing \CV.

We hence set an upper limit on the combined cross section times branching ratio of \tHq, \tHW, and \ttH.

If we assume a modifier for the Higgs-to-tau coupling ($\kappa_\tau$) to be equal to $\Ct$, the relative fractions of $\WW$, $\ZZ$, and $\tautau$ in our selection will only depend on the ratio of $\Ct/\CV$.
Any limit set at any given value of $\Ct/\CV$ is thus valid for all values of $\Ct$ and $\CV$ with that ratio, and could then be compared with theoretical predictions of cross sections at different values of either modifier.
Rather than as a function of the $\Ct/\CV$ ratio, limits could (equivalently) be reported as a function of the relative strength of Higgs-top and Higgs-vector-boson couplings, multiplied by the relative sign.
Such a parameter, further referred to as \ft, as defined in Eq.~\ref{eq:ft}, spans the entire possible parameter space between $-1.0$ and $1.0$, with the SM expectation at $0.5$.
Absolute values of $1.0$ or $0.0$ would then correspond to purely Higgs-top and purely Higgs-V couplings, respectively.
\begin{equation} \label{eq:ft}
	\ft = \mathrm{sign}(\Ct/\CV) \times \frac{\Ct^2}{\Ct^2+\CV^2}.
\end{equation}

Table~\ref{tab:ctcvvalues} shows the points in the $\Ct/\CV$ and \ft\ parameter space that are mapped by the 51 individual \Ct\ and \CV\ points.

\begin{table}[h!]
\centering
\begin{tabular}{rrrrr}
 $\ft$ & \Ct/\CV & $\CV=0.5$ & $\CV=1.0$ & $\CV=1.5$ \\ \hline
  -0.973 & -6.000 & -3.00 &       &       \\
  -0.941 & -4.000 & -2.00 &       &       \\
  -0.900 & -3.000 & -1.50 & -3.00 &       \\
  -0.862 & -2.500 & -1.25 &       &       \\
  -0.800 & -2.000 & -1.00 & -2.00 & -3.00 \\
  -0.692 & -1.500 & -0.75 & -1.50 &       \\
  -0.640 & -1.333 &       &       & -2.00 \\
  -0.610 & -1.250 &       & -1.25 &       \\
  -0.500 & -1.000 & -0.50 & -1.00 & -1.50 \\
  -0.410 & -0.833 &       &       & -1.25 \\
  -0.360 & -0.750 &       & -0.75 &       \\
  -0.308 & -0.667 &       &       & -1.00 \\
  -0.200 & -0.500 & -0.25 & -0.50 & -0.75 \\
  -0.100 & -0.333 &       &       & -0.50 \\
  -0.059 & -0.250 &       & -0.25 &       \\
  -0.027 & -0.167 &       &       & -0.25 \\
   0.000 &  0.000 &  0.00 &  0.00 &  0.00 \\
   0.027 &  0.167 &       &       &  0.25 \\
   0.059 &  0.250 &       &  0.25 &       \\
   0.100 &  0.333 &       &       &  0.50 \\
   0.200 &  0.500 &  0.25 &  0.50 &  0.75 \\
   0.308 &  0.667 &       &       &  1.00 \\
   0.360 &  0.750 &       &  0.75 &       \\
   0.410 &  0.833 &       &       &  1.25 \\
   0.500 &  1.000 &  0.50 &  1.00 &  1.50 \\
   0.610 &  1.250 &       &  1.25 &       \\
   0.640 &  1.333 &       &       &  2.00 \\
   0.692 &  1.500 &  0.75 &  1.50 &       \\
   0.800 &  2.000 &  1.00 &  2.00 &  3.00 \\
   0.862 &  2.500 &  1.25 &       &       \\
   0.900 &  3.000 &  1.50 &  3.00 &       \\
   0.941 &  4.000 &  2.00 &       &       \\
   0.973 &  6.000 &  3.00 &       &       \\ \hline
\end{tabular}
\caption{The 33 distinct values of $\Ct/\CV$ and \ft\ as mapped by the 51 \Ct\ and \CV\ points.}
\label{tab:ctcvvalues}
\end{table}

The overall higgs decay width (modified by both \Ct\ and \CV) becomes irrelevant if limits are quoted as absolute cross sections rather than multiples of the expected cross section (which depends on the overall Higgs decay width).

% Two possibilities are explored: one where the $\gamma\gamma$, $\Z\gamma$, and $\Pg\Pg$ decays are modified with \Ct\ (referred to as the ``resolved'' model henceforth), and one where they are kept fixed at their SM values.
% In both cases, the $\PH\to\cPqc\cPqc$ branching is left unchanged with \Ct.

The 1D histograms of events as categorized in regions of the 2D BDT plane is then used in a maximum likelihood fit of signal and background shapes, where the \tHq, \tHW, and \ttH\ signals are floating with a common signal strength modifier $r$, producing a 95\% C.L. upper limit the observed cross section of $\tHq+\tHW+\ttH$.

This is done separately for each point of \Ct\ and \CV, where the cross sections and branching fractions are scaled accordingly in each point.
Limits at fixed values of $\Ct/\CV$ are by construction identical.
Tables~\ref{tab:brscalingK6_0p5}--\ref{tab:brscalingK6_1p5} and~\ref{tab:xsbrscalingK6_0p5}--\ref{tab:xsbrscalingK6_1p5} in Appendix~\ref{sec:xsbrscalings} show the scalings of cross section times branching fraction, as well as branching fractions alone for each of the Higgs decay modes and each of the signal components.

%% Leaving out the limit plots in each channel for now.
% \begin{figure} [!h]
%  \centering
%  \includegraphics[width=0.32\textwidth]{figures/limits/limits_3l_cv_1p0.pdf}
%  \includegraphics[width=0.32\textwidth]{figures/limits/limits_2lss_mm_cv_1p0.pdf}
%  \includegraphics[width=0.32\textwidth]{figures/limits/limits_2lss_em_cv_1p0.pdf} \\
%  \includegraphics[width=0.32\textwidth]{figures/limits/limits_3l_cv_1p5.pdf}
%  \includegraphics[width=0.32\textwidth]{figures/limits/limits_2lss_mm_cv_1p5.pdf}
%  \includegraphics[width=0.32\textwidth]{figures/limits/limits_2lss_em_cv_1p5.pdf} \\
%  \includegraphics[width=0.32\textwidth]{figures/limits/limits_3l_cv_0p5.pdf}
%  \includegraphics[width=0.32\textwidth]{figures/limits/limits_2lss_mm_cv_0p5.pdf}
%  \includegraphics[width=0.32\textwidth]{figures/limits/limits_2lss_em_cv_0p5.pdf}
% \
% \caption{Expected asymptotic limit on $\frac{\sigma}{\sigma_{theor}}$ as a function of \Ct\ for $\CV=1.0$, $\CV=1.5$, $\CV=0.5$ (top to bottom) for the three lepton channel (left), the \mumu\ channel (middle), and the \emu\ channel (right).}
% \label{fig:limits_cv_3l}
% \end{figure}




\input{bincutopt}
\clearpage


\section{Systematics}
\label{sec:systematics}
\input{systematics}
\clearpage


\section{Results}
\label{sec:results}
The results are interpreted by comparing the observed yields with the expectation from background and a 125\GeV SM Higgs boson. We introduce a signal strength parameter $\mu = \sigma/\sigma_\mathrm{SM}$, and we scale by that value the expected yields from $\ttH$ without altering the branching fractions or the kinematics of the events.

Results in terms of the asymptotic $95\%$ CL upper limit on $\mu$ are presented in Table~\ref{tab:res_limit}.

The observed (median expected in absence of signal) upper limit from the combination of all decay modes is 2.5 (0.8). The observed (expected) best fit signal strength for the SM Higgs hypothesis is $1.78_{-0.54}^{+0.60}$ ($1.00_{-0.42}^{+0.46}$) times the SM expectation, as shown in Table~\ref{tab:res_mu}. The observed (expected) significance is $3.4\,\sigma$ ($2.4\,\sigma$).

The impact of statistical, theoretical and experimental sources of uncertainty is detailed in Table~\ref{tab:res_mu_uncsplit}.

\begin{table}[thb]
\centering
\begin{tabular}{l@{\qquad}r@{\qquad}r}\hline
\multicolumn{1}{c@{\qquad}}{Category} &
\multicolumn{1}{c}{Observed limit} & \multicolumn{1}{c}{Expected limit $\pm 1\sigma$} \\ \hline
same-sign di-lepton       & 2.8       & $ 0.86\,(-0.25)\,(+0.39) $ \\
three lepton              & 2.7       & $ 1.34\,(-0.41)\,(+0.64) $ \\
four lepton               & 6.1       & $ 4.70\,(-1.66)\,(+2.96) $ \\ \hline
combined                  & 2.5       & $ 0.76\,(-0.23)\,(+0.34) $ \\
\end{tabular}\\
\caption{Asymptotic $95\%$ CL upper limits on $\mu$ under the background-only hypothesis.}
\label{tab:res_limit}
\end{table}


\begin{table}[thb]
\centering
\begin{tabular}{l@{\qquad}r@{\qquad}r}\hline
\multicolumn{1}{c@{\qquad}}{Category} &
\multicolumn{1}{c}{Observed $\mu$ fit $\pm 1\sigma$} & \multicolumn{1}{c}{Expected $\mu$ fit $\pm 1\sigma$} \\ \hline
same-sign di-lepton             & $ 1.78\,(-0.54)\,(+0.60) $  & $ 1.00\,(-0.47)\,(+0.51)   $ \\
three lepton                    & $ 1.16\,(-0.76)\,(+0.84) $  & $ 1.00\,(-0.67)\,(+0.76)   $ \\
four lepton                     & $ 1.05\,(-1.58)\,(+2.35) $  & $ 1.00\,(-1.56)\,(+2.29)   $ \\ \hline
combined                        & $ 1.56\,(-0.48)\,(+0.54) $  & $ 1.00\,(-0.42)\,(+0.46)   $ \\
\end{tabular}\\
\caption{Best fit of the signal strength parameter.}
\label{tab:res_mu}
\end{table}

\begin{table}[thb]
\centering
\begin{tabular}{l@{\qquad}r}\hline
\multicolumn{1}{c@{\qquad}}{Category} &
\multicolumn{1}{c}{Expected uncertainty on $\mu$} \\ \hline
Statistical sources       & $ (-0.26)\,(+0.27) $ \\
Theoretical sources       & $ (-0.21)\,(+0.24) $ \\
Experimental sources      & $ (-0.25)\,(+0.28) $ \\ \hline
Total                     & $ (-0.42)\,(+0.46) $ \\
\end{tabular}\\
\caption{Split of expected uncertainty in statistical, theoretical and experimental contributions.}
\label{tab:res_mu_uncsplit}
\end{table}

%Including the ttZ control region described in Appendix~\ref{sec:ttZto3l} in the fit (continuing to float only $\mu_{\ttH}$), the expected best fit signal strength is $1.00_{-0.42}^{+0.46}$ times the SM expectation, with an expected significance of $2.42\,\sigma$.

Figure~\ref{fig:postfit_distr} show the post-fit distribution of the binned discriminating variables and the population of the categories used in the fit.

\begin{figure}[!htb]
\centering
\includegraphics[width=0.32\linewidth]{plots_postfit/kinMVA_2lss_bins8_withBDTv8_withHj_ourBinning.pdf}
\includegraphics[width=0.32\linewidth]{plots_postfit/2lep_catIndex.pdf}\\
\includegraphics[width=0.32\linewidth]{plots_postfit/kinMVA_3l_bins5_withMEM_ourBinning.pdf}
\includegraphics[width=0.32\linewidth]{plots_postfit/3lep_catIndex.pdf}
\caption{Post-fit distributions of discriminating variables and category population for 2lss (top row) and 3l (bottom row)}
\label{fig:postfit_distr}
\end{figure}


Figure~\ref{fig:impacts} shows the post-fit values of the nuisances and their correlation with the fitted signal strength.

\begin{figure}[!htb]
\centering
\includegraphics[width=0.80\linewidth]{plots_postfit/impacts1.pdf}\\
\includegraphics[width=0.80\linewidth]{plots_postfit/impacts2.pdf}
\caption{Impact plot showing the correlation between the main nuisance parameters and the best fit signal strength.}
\label{fig:impacts}
\end{figure}

\clearpage

\appendix

\section{Control region plots}
\label{sec:controlregions}
\input{controlregions}
\clearpage

\section{BDT output change with varying \CV/\Ct}
\label{sec:bdtvscvct}
\input{bdtcvct.tex}
\clearpage

\section{Further channel categorization}
\label{sec:categorization}
\input{further_categorization.tex}
\clearpage

\section{Cross section times BR scalings}
\label{sec:xsbrscalings}
\chapter{Cross sections and Branching ratios scalings}\label{sec:xsbrscalings}

\begin{table}[h!]
  \centering
  \footnotesize
  \begin{tabular}{ll rrrrrrrrr}\hline
   \CV\ & \Ct\   & HWW    & HZZ    & H$\tau\tau$& H$\mu\mu$ & Hbb & Hcc & H$\gamma\gamma$ & H$Z\gamma$ & Hgg \\ \hline
   0.5  & -6.0   & 0.0827 & 0.0827 & 11.9098 & 11.9098 & 0.3308 & 0.3308 & 0.3308 & 0.3308 & 0.3308 \\
   0.5  & -4.0   & 0.1417 & 0.1417 & 9.0699  & 9.0699  & 0.5669 & 0.5669 & 0.5669 & 0.5669 & 0.5669 \\
   0.5  & -3.0   & 0.1889 & 0.1889 & 6.7999  & 6.7999  & 0.7555 & 0.7555 & 0.7555 & 0.7555 & 0.7555 \\
   0.5  & -2.5   & 0.2173 & 0.2173 & 5.4325  & 5.4325  & 0.8692 & 0.8692 & 0.8692 & 0.8692 & 0.8692 \\
   0.5  & -2.0   & 0.2478 & 0.2478 & 3.9647  & 3.9647  & 0.9912 & 0.9912 & 0.9912 & 0.9912 & 0.9912 \\
   0.5  & -1.5   & 0.2782 & 0.2782 & 2.5034  & 2.5034  & 1.1126 & 1.1126 & 1.1126 & 1.1126 & 1.1126 \\
   0.5  & -1.333 & 0.2877 & 0.2877 & 2.0448  & 2.0448  & 1.1508 & 1.1508 & 1.1508 & 1.1508 & 1.1508 \\
   0.5  & -1.25  & 0.2922 & 0.2922 & 1.8264  & 1.8264  & 1.1689 & 1.1689 & 1.1689 & 1.1689 & 1.1689 \\
   0.5  & -1.0   & 0.3048 & 0.3048 & 1.2194  & 1.2194  & 1.2194 & 1.2194 & 1.2194 & 1.2194 & 1.2194 \\
   0.5  & -0.833 & 0.3122 & 0.3122 & 0.8665  & 0.8665  & 1.2487 & 1.2487 & 1.2487 & 1.2487 & 1.2487 \\
   0.5  & -0.75  & 0.3154 & 0.3154 & 0.7097  & 0.7097  & 1.2617 & 1.2617 & 1.2617 & 1.2617 & 1.2617 \\
   0.5  & -0.667 & 0.3184 & 0.3184 & 0.5666  & 0.5666  & 1.2736 & 1.2736 & 1.2736 & 1.2736 & 1.2736 \\
   0.5  & -0.5   & 0.3235 & 0.3235 & 0.3235  & 0.3235  & 1.2938 & 1.2938 & 1.2938 & 1.2938 & 1.2938 \\
   0.5  & -0.333 & 0.3272 & 0.3272 & 0.1451  & 0.1451  & 1.3087 & 1.3087 & 1.3087 & 1.3087 & 1.3087 \\
   0.5  & -0.25  & 0.3285 & 0.3285 & 0.0821  & 0.0821  & 1.3139 & 1.3139 & 1.3139 & 1.3139 & 1.3139 \\
   0.5  & -0.167 & 0.3294 & 0.3294 & 0.0367  & 0.0367  & 1.3177 & 1.3177 & 1.3177 & 1.3177 & 1.3177 \\
   0.5  & 0.0    & 0.3302 & 0.3302 & 0.0000  & 0.0000  & 1.3207 & 1.3207 & 1.3207 & 1.3207 & 1.3207 \\
   0.5  & 0.167  & 0.3294 & 0.3294 & 0.0367  & 0.0367  & 1.3177 & 1.3177 & 1.3177 & 1.3177 & 1.3177 \\
   0.5  & 0.25   & 0.3285 & 0.3285 & 0.0821  & 0.0821  & 1.3139 & 1.3139 & 1.3139 & 1.3139 & 1.3139 \\
   0.5  & 0.333  & 0.3272 & 0.3272 & 0.1451  & 0.1451  & 1.3087 & 1.3087 & 1.3087 & 1.3087 & 1.3087 \\
   0.5  & 0.5    & 0.3235 & 0.3235 & 0.3235  & 0.3235  & 1.2938 & 1.2938 & 1.2938 & 1.2938 & 1.2938 \\
   0.5  & 0.667  & 0.3184 & 0.3184 & 0.5666  & 0.5666  & 1.2736 & 1.2736 & 1.2736 & 1.2736 & 1.2736 \\
   0.5  & 0.75   & 0.3154 & 0.3154 & 0.7097  & 0.7097  & 1.2617 & 1.2617 & 1.2617 & 1.2617 & 1.2617 \\
   0.5  & 0.833  & 0.3122 & 0.3122 & 0.8665  & 0.8665  & 1.2487 & 1.2487 & 1.2487 & 1.2487 & 1.2487 \\
   0.5  & 1.0    & 0.3048 & 0.3048 & 1.2194  & 1.2194  & 1.2194 & 1.2194 & 1.2194 & 1.2194 & 1.2194 \\
   0.5  & 1.25   & 0.2922 & 0.2922 & 1.8264  & 1.8264  & 1.1689 & 1.1689 & 1.1689 & 1.1689 & 1.1689 \\
   0.5  & 1.333  & 0.2877 & 0.2877 & 2.0448  & 2.0448  & 1.1508 & 1.1508 & 1.1508 & 1.1508 & 1.1508 \\
   0.5  & 1.5    & 0.2782 & 0.2782 & 2.5034  & 2.5034  & 1.1126 & 1.1126 & 1.1126 & 1.1126 & 1.1126 \\
   0.5  & 2.0    & 0.2478 & 0.2478 & 3.9647  & 3.9647  & 0.9912 & 0.9912 & 0.9912 & 0.9912 & 0.9912 \\
   0.5  & 2.5    & 0.2173 & 0.2173 & 5.4325  & 5.4325  & 0.8692 & 0.8692 & 0.8692 & 0.8692 & 0.8692 \\
   0.5  & 3.0    & 0.1889 & 0.1889 & 6.7999  & 6.7999  & 0.7555 & 0.7555 & 0.7555 & 0.7555 & 0.7555 \\
   0.5  & 4.0    & 0.1417 & 0.1417 & 9.0699  & 9.0699  & 0.5669 & 0.5669 & 0.5669 & 0.5669 & 0.5669 \\
   0.5  & 6.0    & 0.0827 & 0.0827 & 11.9098 & 11.9098 & 0.3308 & 0.3308 & 0.3308 & 0.3308 & 0.3308 \\\hline
    \end{tabular}
    \caption[Scalings of Higgs decay branching ratios vs.\ \Ct\ and \CV=0.5\ ]{Scalings of Higgs decay branching ratios vs.\ \Ct\ and \CV=0.5\ for the non-resolved model.}\label{tab:brscalingK6_0p5}
 \end{table}

\begin{table}[h!]
  \centering
  \footnotesize
  \begin{tabular}{ll rrrrrrrrr}\hline
   \CV\ & \Ct\   & HWW    & HZZ    & H$\tau\tau$& H$\mu\mu$ & Hbb    & Hcc    & H$\gamma\gamma$ & H$Z\gamma$ & Hgg \\ \hline
   1.0  & -6.0   & 0.3122 & 0.3122 & 11.2408    & 11.2408   & 0.3122 & 0.3122 & 0.3122          & 0.3122     & 0.3122 \\
   1.0  & -4.0   & 0.5144 & 0.5144 & 8.2305     & 8.2305    & 0.5144 & 0.5144 & 0.5144          & 0.5144     & 0.5144 \\
   1.0  & -3.0   & 0.6651 & 0.6651 & 5.9862     & 5.9862    & 0.6651 & 0.6651 & 0.6651          & 0.6651     & 0.6651 \\
   1.0  & -2.5   & 0.7517 & 0.7517 & 4.6979     & 4.6979    & 0.7517 & 0.7517 & 0.7517          & 0.7517     & 0.7517 \\
   1.0  & -2.0   & 0.8412 & 0.8412 & 3.3647     & 3.3647    & 0.8412 & 0.8412 & 0.8412          & 0.8412     & 0.8412 \\
   1.0  & -1.5   & 0.9271 & 0.9271 & 2.0859     & 2.0859    & 0.9271 & 0.9271 & 0.9271          & 0.9271     & 0.9271 \\
   1.0  & -1.333 & 0.9534 & 0.9534 & 1.6941     & 1.6941    & 0.9534 & 0.9534 & 0.9534          & 0.9534     & 0.9534 \\
   1.0  & -1.25  & 0.9658 & 0.9658 & 1.5091     & 1.5091    & 0.9658 & 0.9658 & 0.9658          & 0.9658     & 0.9658 \\
   1.0  & -1.0   & 1.0000 & 1.0000 & 1.0000     & 1.0000    & 1.0000 & 1.0000 & 1.0000          & 1.0000     & 1.0000 \\
   1.0  & -0.833 & 1.0196 & 1.0196 & 0.7075     & 0.7075    & 1.0196 & 1.0196 & 1.0196          & 1.0196     & 1.0196 \\
   1.0  & -0.75  & 1.0283 & 1.0283 & 0.5784     & 0.5784    & 1.0283 & 1.0283 & 1.0283          & 1.0283     & 1.0283 \\
   1.0  & -0.667 & 1.0362 & 1.0362 & 0.4610     & 0.4610    & 1.0362 & 1.0362 & 1.0362          & 1.0362     & 1.0362 \\
   1.0  & -0.5   & 1.0495 & 1.0495 & 0.2624     & 0.2624    & 1.0495 & 1.0495 & 1.0495          & 1.0495     & 1.0495 \\
   1.0  & -0.333 & 1.0593 & 1.0593 & 0.1175     & 0.1175    & 1.0593 & 1.0593 & 1.0593          & 1.0593     & 1.0593 \\
   1.0  & -0.25  & 1.0627 & 1.0627 & 0.0664     & 0.0664    & 1.0627 & 1.0627 & 1.0627          & 1.0627     & 1.0627 \\
   1.0  & -0.167 & 1.0652 & 1.0652 & 0.0297     & 0.0297    & 1.0652 & 1.0652 & 1.0652          & 1.0652     & 1.0652 \\
   1.0  & 0.0    & 1.0672 & 1.0672 & 0.0000     & 0.0000    & 1.0672 & 1.0672 & 1.0672          & 1.0672     & 1.0672 \\
   1.0  & 0.167  & 1.0652 & 1.0652 & 0.0297     & 0.0297    & 1.0652 & 1.0652 & 1.0652          & 1.0652     & 1.0652 \\
   1.0  & 0.25   & 1.0627 & 1.0627 & 0.0664     & 0.0664    & 1.0627 & 1.0627 & 1.0627          & 1.0627     & 1.0627 \\
   1.0  & 0.333  & 1.0593 & 1.0593 & 0.1175     & 0.1175    & 1.0593 & 1.0593 & 1.0593          & 1.0593     & 1.0593 \\
   1.0  & 0.5    & 1.0495 & 1.0495 & 0.2624     & 0.2624    & 1.0495 & 1.0495 & 1.0495          & 1.0495     & 1.0495 \\
   1.0  & 0.667  & 1.0362 & 1.0362 & 0.4610     & 0.4610    & 1.0362 & 1.0362 & 1.0362          & 1.0362     & 1.0362 \\
   1.0  & 0.75   & 1.0283 & 1.0283 & 0.5784     & 0.5784    & 1.0283 & 1.0283 & 1.0283          & 1.0283     & 1.0283 \\
   1.0  &  0.833 & 1.0196 & 1.0196 & 0.7075     & 0.7075    & 1.0196 & 1.0196 & 1.0196          & 1.0196     & 1.0196 \\
   1.0  & 1.0    & 1.0000 & 1.0000 & 1.0000     & 1.0000    & 1.0000 & 1.0000 & 1.0000          & 1.0000     & 1.0000 \\
   1.0  & 1.25   & 0.9658 & 0.9658 & 1.5091     & 1.5091    & 0.9658 & 0.9658 & 0.9658          & 0.9658     & 0.9658 \\
   1.0  & 1.333  & 0.9534 & 0.9534 & 1.6941     & 1.6941    & 0.9534 & 0.9534 & 0.9534          & 0.9534     & 0.9534 \\
   1.0  & 1.5    & 0.9271 & 0.9271 & 2.0859     & 2.0859    & 0.9271 & 0.9271 & 0.9271          & 0.9271     & 0.9271 \\
   1.0  & 2.0    & 0.8412 & 0.8412 & 3.3647     & 3.3647    & 0.8412 & 0.8412 & 0.8412          & 0.8412     & 0.8412 \\
   1.0  & 2.5    & 0.7517 & 0.7517 & 4.6979     & 4.6979    & 0.7517 & 0.7517 & 0.7517          & 0.7517     & 0.7517 \\
   1.0  & 3.0    & 0.6651 & 0.6651 & 5.9862     & 5.9862    & 0.6651 & 0.6651 & 0.6651          & 0.6651     & 0.6651 \\
   1.0  & 4.0    & 0.5144 & 0.5144 & 8.2305     & 8.2305    & 0.5144 & 0.5144 & 0.5144          & 0.5144     & 0.5144 \\
   1.0  & 6.0    & 0.3122 & 0.3122 & 11.2408    & 11.2408   & 0.3122 & 0.3122 & 0.3122          & 0.3122     & 0.3122 \\\hline
    \end{tabular}
    \caption[Scalings of Higgs decay branching ratios vs.\ \Ct\ and \CV=1.0 ]{Scalings of Higgs decay branching ratios vs.\ \Ct\ and \CV=1.0\ for the non-resolved model.}\label{tab:brscalingK6_1}
 \end{table}

\begin{table}[h!]
  \centering
  \footnotesize
  \begin{tabular}{ll rrrrrrrrr}\hline
   \CV\ & \Ct\   & HWW    & HZZ    & H$\tau\tau$& H$\mu\mu$ & Hbb & Hcc & H$\gamma\gamma$ & H$Z\gamma$ & Hgg \\ \hline
   1.5  & -6.0   & 0.6424 & 0.6424 & 10.2785 & 10.2785 & 0.2855 & 0.2855 & 0.2855 & 0.2855 & 0.2855 \\
   1.5  & -4.0   & 1.0028 & 1.0028 & 7.1307  & 7.1307  & 0.4457 & 0.4457 & 0.4457 & 0.4457 & 0.4457 \\
   1.5  & -3.0   & 1.2477 & 1.2477 & 4.9909  & 4.9909  & 0.5545 & 0.5545 & 0.5545 & 0.5545 & 0.5545 \\
   1.5  & -2.5   & 1.3802 & 1.3802 & 3.8338  & 3.8338  & 0.6134 & 0.6134 & 0.6134 & 0.6134 & 0.6134 \\
   1.5  & -2.0   & 1.5115 & 1.5115 & 2.6870  & 2.6870  & 0.6718 & 0.6718 & 0.6718 & 0.6718 & 0.6718 \\
   1.5  & -1.5   & 1.6322 & 1.6322 & 1.6322  & 1.6322  & 0.7254 & 0.7254 & 0.7254 & 0.7254 & 0.7254 \\
   1.5  & -1.333 & 1.6682 & 1.6682 & 1.3175  & 1.3175  & 0.7414 & 0.7414 & 0.7414 & 0.7414 & 0.7414 \\
   1.5  & -1.25  & 1.6851 & 1.6851 & 1.1702  & 1.1702  & 0.7489 & 0.7489 & 0.7489 & 0.7489 & 0.7489 \\
   1.5  & -1.0   & 1.7310 & 1.7310 & 0.7693  & 0.7693  & 0.7693 & 0.7693 & 0.7693 & 0.7693 & 0.7693 \\
   1.5  & -0.833 & 1.7570 & 1.7570 & 0.5419  & 0.5419  & 0.7809 & 0.7809 & 0.7809 & 0.7809 & 0.7809 \\
   1.5  & -0.75  & 1.7684 & 1.7684 & 0.4421  & 0.4421  & 0.7860 & 0.7860 & 0.7860 & 0.7860 & 0.7860 \\
   1.5  & -0.667 & 1.7788 & 1.7788 & 0.3517  & 0.3517  & 0.7906 & 0.7906 & 0.7906 & 0.7906 & 0.7906 \\
   1.5  & -0.5   & 1.7962 & 1.7962 & 0.1996  & 0.1996  & 0.7983 & 0.7983 & 0.7983 & 0.7983 & 0.7983 \\
   1.5  & -0.333 & 1.8089 & 1.8089 & 0.0891  & 0.0891  & 0.8039 & 0.8039 & 0.8039 & 0.8039 & 0.8039 \\
   1.5  & -0.25  & 1.8133 & 1.8133 & 0.0504  & 0.0504  & 0.8059 & 0.8059 & 0.8059 & 0.8059 & 0.8059 \\
   1.5  & -0.167 & 1.8165 & 1.8165 & 0.0225  & 0.0225  & 0.8073 & 0.8073 & 0.8073 & 0.8073 & 0.8073 \\
   1.5  & 0.0    & 1.8191 & 1.8191 & 0.0000  & 0.0000  & 0.8085 & 0.8085 & 0.8085 & 0.8085 & 0.8085 \\
   1.5  & 0.167  & 1.8165 & 1.8165 & 0.0225  & 0.0225  & 0.8073 & 0.8073 & 0.8073 & 0.8073 & 0.8073 \\
   1.5  & 0.25   & 1.8133 & 1.8133 & 0.0504  & 0.0504  & 0.8059 & 0.8059 & 0.8059 & 0.8059 & 0.8059 \\
   1.5  & 0.333  & 1.8089 & 1.8089 & 0.0891  & 0.0891  & 0.8039 & 0.8039 & 0.8039 & 0.8039 & 0.8039 \\
   1.5  & 0.5    & 1.7962 & 1.7962 & 0.1996  & 0.1996  & 0.7983 & 0.7983 & 0.7983 & 0.7983 & 0.7983 \\
   1.5  & 0.667  & 1.7788 & 1.7788 & 0.3517  & 0.3517  & 0.7906 & 0.7906 & 0.7906 & 0.7906 & 0.7906 \\
   1.5  & 0.75   & 1.7684 & 1.7684 & 0.4421  & 0.4421  & 0.7860 & 0.7860 & 0.7860 & 0.7860 & 0.7860 \\
   1.5  & 0.833  & 1.7570 & 1.7570 & 0.5419  & 0.5419  & 0.7809 & 0.7809 & 0.7809 & 0.7809 & 0.7809 \\
   1.5  & 1.0    & 1.7310 & 1.7310 & 0.7693  & 0.7693  & 0.7693 & 0.7693 & 0.7693 & 0.7693 & 0.7693 \\
   1.5  & 1.25   & 1.6851 & 1.6851 & 1.1702  & 1.1702  & 0.7489 & 0.7489 & 0.7489 & 0.7489 & 0.7489 \\
   1.5  & 1.333  & 1.6682 & 1.6682 & 1.3175  & 1.3175  & 0.7414 & 0.7414 & 0.7414 & 0.7414 & 0.7414 \\
   1.5  & 1.5    & 1.6322 & 1.6322 & 1.6322  & 1.6322  & 0.7254 & 0.7254 & 0.7254 & 0.7254 & 0.7254 \\
   1.5  & 2.0    & 1.5115 & 1.5115 & 2.6870  & 2.6870  & 0.6718 & 0.6718 & 0.6718 & 0.6718 & 0.6718 \\
   1.5  & 2.5    & 1.3802 & 1.3802 & 3.8338  & 3.8338  & 0.6134 & 0.6134 & 0.6134 & 0.6134 & 0.6134 \\
   1.5  & 3.0    & 1.2477 & 1.2477 & 4.9909  & 4.9909  & 0.5545 & 0.5545 & 0.5545 & 0.5545 & 0.5545 \\
   1.5  & 4.0    & 1.0028 & 1.0028 & 7.1307  & 7.1307  & 0.4457 & 0.4457 & 0.4457 & 0.4457 & 0.4457 \\
   1.5  & 6.0    & 0.6424 & 0.6424 & 10.2785 & 10.2785 & 0.2855 & 0.2855 & 0.2855 & 0.2855 & 0.2855 \\\hline
    \end{tabular}
    \caption[Scalings of Higgs decay branching ratios vs.\ \Ct\ and \CV=1.5]{Scalings of Higgs decay branching ratios vs.\ \Ct\ and \CV=1.5\ for the non-resolved model.}\label{tab:brscalingK6_1p5}
 \end{table}
\begin{landscape}
\begin{table}[h!]                                                                                                                                                                          
  \centering                                                                                                                                                                               
  \footnotesize                                                                                                                                                                            
  \begin{tabular}{ll rrr rrr rrr}\hline                                                                                                                                                          
   \CV\ & \Ct\  & ttHWW  & ttHZZ  & ttH$\tau\tau$& tHqWW & tHqZZ & tHq$\tau\tau$& tHWWW & tHWZZ & tHW$\tau\tau$ \\ \hline   
   0.5 & -6.0   & 2.9775 & 2.9775 & 428.7530 & 9.2066 & 9.2066 & 1325.7460 & 9.7660 & 9.7660 & 1406.3049 \\
   0.5 & -4.0   & 2.2675 & 2.2675 & 145.1182 & 7.5740 & 7.5740 & 484.7357  & 7.8819 & 7.8819 & 504.4411 \\
   0.5 & -3.0   & 1.7000 & 1.7000 & 61.1988  & 6.1214 & 6.1214 & 220.3702  & 6.2562 & 6.2562 & 225.2227 \\
   0.5 & -2.5   & 1.3581 & 1.3581 & 33.9529  & 5.1857 & 5.1857 & 129.6430  & 5.2277 & 5.2277 & 130.6931 \\
   0.5 & -2.0   & 0.9912 & 0.9912 & 15.8589  & 4.1227 & 4.1227 & 65.9633   & 4.0762 & 4.0762 & 65.2197 \\
   0.5 & -1.5   & 0.6259 & 0.6259 & 5.6327   & 2.9838 & 2.9838 & 26.8544   & 2.8645 & 2.8645 & 25.7805  \\
   0.5 & -1.333 & 0.5112 & 0.5112 & 3.6333   & 2.6025 & 2.6025 & 18.4974   & 2.4648 & 2.4648 & 17.5190 \\
   0.5 & -1.25  & 0.4566 & 0.4566 & 2.8538   & 2.4154 & 2.4154 & 15.0962   & 2.2700 & 2.2700 & 14.1878 \\
   0.5 & -1.0   & 0.3048 & 0.3048 & 1.2194   & 1.8696 & 1.8696 & 7.4784    & 1.7078 & 1.7078 & 6.8310 \\
   0.5 & -0.833 & 0.2166 & 0.2166 & 0.6012   & 1.5271 & 1.5271 & 4.2386    & 1.3605 & 1.3605 & 3.7760 \\
   0.5 & -0.75  & 0.1774 & 0.1774 & 0.3992   & 1.3657 & 1.3657 & 3.0729    & 1.1987 & 1.1987 & 2.6970 \\
   0.5 & -0.667 & 0.1417 & 0.1417 & 0.2521   & 1.2111 & 1.2111 & 2.1553    & 1.0451 & 1.0451 & 1.8598 \\
   0.5 & -0.5   & 0.0809 & 0.0809 & 0.0809   & 0.9236 & 0.9236 & 0.9236    & 0.7640 & 0.7640 & 0.7640 \\
   0.5 & -0.333 & 0.0363 & 0.0363 & 0.0161   & 0.6720 & 0.6720 & 0.2981    & 0.5249 & 0.5249 & 0.2328 \\
   0.5 & -0.25  & 0.0205 & 0.0205 & 0.0051   & 0.5618 & 0.5618 & 0.1405    & 0.4231 & 0.4231 & 0.1058 \\
   0.5 & -0.167 & 0.0092 & 0.0092 & 0.0010   & 0.4622 & 0.4622 & 0.0516    & 0.3334 & 0.3334 & 0.0372 \\
   0.5 & 0.0    & 0.0000 & 0.0000 & 0.0000   & 0.2953 & 0.2953 & 0.0000    & 0.1909 & 0.1909 & 0.0000 \\
   0.5 & 0.167  & 0.0092 & 0.0092 & 0.0010   & 0.1755 & 0.1755 & 0.0196    & 0.1010 & 0.1010 & 0.0113 \\
   0.5 & 0.25   & 0.0205 & 0.0205 & 0.0051   & 0.1339 & 0.1339 & 0.0335    & 0.0762 & 0.0762 & 0.0191 \\
   0.5 & 0.333  & 0.0363 & 0.0363 & 0.0161   & 0.1043 & 0.1043 & 0.0463    & 0.0647 & 0.0647 & 0.0287 \\
   0.5 & 0.5    & 0.0809 & 0.0809 & 0.0809   & 0.0809 & 0.0809 & 0.0809    & 0.0809 & 0.0809 & 0.0809 \\
   0.5 & 0.667  & 0.1417 & 0.1417 & 0.2521   & 0.1044 & 0.1044 & 0.1859    & 0.1480 & 0.1480 & 0.2634 \\
   0.5 & 0.75   & 0.1774 & 0.1774 & 0.3992   & 0.1329 & 0.1329 & 0.2991    & 0.1993 & 0.1993 & 0.4485 \\
   0.5 & 0.833  & 0.2166 & 0.2166 & 0.6012   & 0.1720 & 0.1720 & 0.4775    & 0.2620 & 0.2620 & 0.7272 \\
   0.5 & 1.0    & 0.3048 & 0.3048 & 1.2194   & 0.2811 & 0.2811 & 1.1243    & 0.4200 & 0.4200 & 1.6801 \\
   0.5 & 1.25   & 0.4566 & 0.4566 & 2.8538   & 0.5119 & 0.5119 & 3.1993    & 0.7270 & 0.7270 & 4.5438 \\
   0.5 & 1.333  & 0.5112 & 0.5112 & 3.6333   & 0.6041 & 0.6041 & 4.2939    & 0.8449 & 0.8449 & 6.0051 \\
   0.5 & 1.5    & 0.6259 & 0.6259 & 5.6327   & 0.8096 & 0.8096 & 7.2863    & 1.1020 & 1.1020 & 9.9179 \\
   0.5 & 2.0    & 0.9912 & 0.9912 & 15.8589  & 1.5402 & 1.5402 & 24.6428   & 1.9827 & 1.9827 & 31.7238 \\
   0.5 & 2.5    & 1.3581 & 1.3581 & 33.9529  & 2.3549 & 2.3549 & 58.8716   & 2.9329 & 2.9329 & 73.3233 \\
   0.5 & 3.0    & 1.7000 & 1.7000 & 61.1988  & 3.1686 & 3.1686 & 114.0678  & 3.8625 & 3.8625 & 139.0502 \\
   0.5 & 4.0    & 2.2675 & 2.2675 & 145.1182 & 4.6200 & 4.6200 & 295.6829  & 5.4873 & 5.4873 & 351.1881 \\
   0.5 & 6.0    & 2.9775 & 2.9775 & 428.7530 & 6.6207 & 6.6207 & 953.3740  & 7.6698 & 7.6698 & 1104.4467 \\\hline
  \end{tabular}
  \caption[Scalings of $\sigma\times$BR for the signal components and \CV=0.5\ ]{Scalings of cross section times BR for the non-resolved model, for the different \ttH, \tHq, \tHW\ signal components and \CV=0.5\ .}\label{tab:xsbrscalingK6_0p5}
\end{table}

\begin{table}[h!]
  \centering
  \footnotesize
  \begin{tabular}{ll rrr rrr rrr}\hline
   \CV\ & \Ct\  & ttHWW  & ttHZZ  & ttH$\tau\tau$& tHqWW & tHqZZ & tHq$\tau\tau$& tHWWW & tHWZZ & tHW$\tau\tau$ \\ \hline
   1.0 & -6.0   & 11.2408 & 11.2408 & 404.6686 & 40.4768 & 40.4768 & 1457.1666 & 41.3681 & 41.3681 & 1489.2533 \\
   1.0 & -4.0   & 8.2305  & 8.2305  & 131.6886 & 34.2339 & 34.2339 & 547.7422  & 33.8480 & 33.8480 & 541.5676 \\
   1.0 & -3.0   & 5.9862  & 5.9862  & 53.8759  & 28.5396 & 28.5396 & 256.8562  & 27.3983 & 27.3983 & 246.5850 \\
   1.0 & -2.5   & 4.6979  & 4.6979  & 29.3616  & 24.8511 & 24.8511 & 155.3195  & 23.3557 & 23.3557 & 145.9734 \\
   1.0 & -2.0   & 3.3647  & 3.3647  & 13.4590  & 20.6360 & 20.6360 & 82.5440   & 18.8497 & 18.8497 & 75.3987 \\
   1.0 & -1.5   & 2.0859  & 2.0859  & 4.6933   & 16.0557 & 16.0557 & 36.1254   & 14.0919 & 14.0919 & 31.7068 \\
   1.0 & -1.333 & 1.6941  & 1.6941  & 3.0102   & 14.4942 & 14.4942 & 25.7545   & 12.5059 & 12.5059 & 22.2216 \\
   1.0 & -1.25  & 1.5091  & 1.5091  & 2.3579   & 13.7201 & 13.7201 & 21.4377   & 11.7273 & 11.7273 & 18.3239 \\
   1.0 & -1.0   & 1.0000  & 1.0000  & 1.0000   & 11.4220 & 11.4220 & 11.4220   & 9.4484  & 9.4484  & 9.4484 \\
   1.0 & -0.833 & 0.7075  & 0.7075  & 0.4909   & 9.9372  & 9.9372  & 6.8953    & 8.0059  & 8.0059  & 5.5552 \\
   1.0 & -0.75  & 0.5784  & 0.5784  & 0.3254   & 9.2212  & 9.2212  & 5.1869    & 7.3200  & 7.3200  & 4.1175 \\
   1.0 & -0.667 & 0.4610  & 0.4610  & 0.2051   & 8.5229  & 8.5229  & 3.7917    & 6.6579  & 6.6579  & 2.9620 \\
   1.0 & -0.5   & 0.2624  & 0.2624  & 0.0656   & 7.1807  & 7.1807  & 1.7952    & 5.4076  & 5.4076  & 1.3519 \\
   1.0 & -0.333 & 0.1175  & 0.1175  & 0.0130   & 5.9375  & 5.9375  & 0.6584    & 4.2814  & 4.2814  & 0.4748 \\
   1.0 & -0.25  & 0.0664  & 0.0664  & 0.0042   & 5.3616  & 5.3616  & 0.3351    & 3.7730  & 3.7730  & 0.2358 \\
   1.0 & -0.167 & 0.0297  & 0.0297  & 0.0008   & 4.8163  & 4.8163  & 0.1343    & 3.3009  & 3.3009  & 0.0921 \\
   1.0 & 0.0    & 0.0000  & 0.0000  & 0.0000   & 3.8183  & 3.8183  & 0.0000    & 2.4676  & 2.4676  & 0.0000 \\
   1.0 & 0.167  & 0.0297  & 0.0297  & 0.0008   & 2.9624  & 2.9624  & 0.0826    & 1.7981  & 1.7981  & 0.0501 \\
   1.0 & 0.25   & 0.0664  & 0.0664  & 0.0042   & 2.5928  & 2.5928  & 0.1620    & 1.5284  & 1.5284  & 0.0955 \\
   1.0 & 0.333  & 0.1175  & 0.1175  & 0.0130   & 2.2612  & 2.2612  & 0.2507    & 1.3014  & 1.3014  & 0.1443 \\
   1.0 & 0.5    & 0.2624  & 0.2624  & 0.0656   & 1.7115  & 1.7115  & 0.4279    & 0.9742  & 0.9742  & 0.2435 \\
   1.0 & 0.667  & 0.4610  & 0.4610  & 0.2051   & 1.3198  & 1.3198  & 0.5871    & 0.8188  & 0.8188  & 0.3643 \\
   1.0 & 0.75   & 0.5784  & 0.5784  & 0.3254   & 1.1834  & 1.1834  & 0.6657    & 0.8042  & 0.8042  & 0.4524 \\
   1.0 & 0.833  & 0.7075  & 0.7075  & 0.4909   & 1.0852  & 1.0852  & 0.7530    & 0.8301  & 0.8301  & 0.5760 \\
   1.0 & 1.0    & 1.0000  & 1.0000  & 1.0000   & 1.0000  & 1.0000  & 1.0000    & 1.0000  & 1.0000  & 1.0000 \\
   1.0 & 1.25   & 1.5091  & 1.5091  & 2.3579   & 1.1380  & 1.1380  & 1.7782    & 1.5278  & 1.5278  & 2.3872 \\
   1.0 & 1.333  & 1.6941  & 1.6941  & 3.0102   & 1.2492  & 1.2492  & 2.2197    & 1.7691  & 1.7691  & 3.1434 \\
   1.0 & 1.5    & 2.0859  & 2.0859  & 4.6933   & 1.5628  & 1.5628  & 3.5163    & 2.3434  & 2.3434  & 5.2727 \\
   1.0 & 2.0    & 3.3647  & 3.3647  & 13.4590  & 3.1023  & 3.1023  & 12.4092   & 4.6362  & 4.6362  & 18.5449 \\
   1.0 & 2.5    & 4.6979  & 4.6979  & 29.3616  & 5.2667  & 5.2667  & 32.9167   & 7.4799  & 7.4799  & 46.7493 \\
   1.0 & 3.0    & 5.9862  & 5.9862  & 53.8759  & 7.7435  & 7.7435  & 69.6914   & 10.5403 & 10.5403 & 94.8625 \\
   1.0 & 4.0    & 8.2305  & 8.2305  & 131.6886 & 12.7892 & 12.7892 & 204.6276  & 16.4642 & 16.4642 & 263.4266 \\
   1.0 & 6.0    & 11.2408 & 11.2408 & 404.6686 & 20.9516 & 20.9516 & 754.2573  & 25.5403 & 25.5403 & 919.4497 \\\hline
  \end{tabular}
  \caption[Scalings of $\sigma\times$BR for the signal components and \CV=1.0\ ]{Scalings of cross section times BR for the non-resolved model, for the different \ttH, \tHq, \tHW\ signal components and \CV=1.0\ .}\label{tab:xsbrscalingK6_1}
\end{table}

\begin{table}[h!]
  \centering
  \footnotesize
  \begin{tabular}{ll rrr rrr rrr}\hline
   \CV\ & \Ct\  & ttHWW   & ttHZZ & ttH$\tau\tau$& tHqWW   & tHqZZ & tHq$\tau\tau$& tHWWW & tHWZZ & tHW$\tau\tau$ \\ \hline
   1.5 & -6.0   & 23.1266 & 23.1266 & 370.0260   & 96.1923 & 96.1923 & 1539.0768  & 95.1080 & 95.1080 & 1521.7272 \\
   1.5 & -4.0   & 16.0441 & 16.0441 & 114.0913   & 81.6690 & 81.6690 & 580.7570   & 77.3512 & 77.3512 & 550.0531 \\
   1.5 & -3.0   & 11.2295 & 11.2295 & 44.9178    & 68.8703 & 68.8703 & 275.4812   & 62.9086 & 62.9086 & 251.6344 \\
   1.5 & -2.5   & 8.6261  & 8.6261  & 23.9614    & 60.7939 & 60.7939 & 168.8720   & 54.1622 & 54.1622 & 150.4505 \\
   1.5 & -2.0   & 6.0458  & 6.0458  & 10.7481    & 51.7152 & 51.7152 & 91.9381    & 44.6227 & 44.6227 & 79.3293 \\
   1.5 & -1.5   & 3.6725  & 3.6725  & 3.6725     & 41.9469 & 41.9469 & 41.9469    & 34.6991 & 34.6991 & 34.6991 \\
   1.5 & -1.333 & 2.9643  & 2.9643  & 2.3410     & 38.6171 & 38.6171 & 30.4971    & 31.4016 & 31.4016 & 24.7987 \\
   1.5 & -1.25  & 2.6330  & 2.6330  & 1.8284     & 36.9629 & 36.9629 & 25.6687    & 29.7807 & 29.7807 & 20.6810 \\
   1.5 & -1.0   & 1.7310  & 1.7310  & 0.7693     & 32.0233 & 32.0233 & 14.2326    & 25.0144 & 25.0144 & 11.1175 \\
   1.5 & -0.833 & 1.2192  & 1.2192  & 0.3760     & 28.7953 & 28.7953 & 8.8803     & 21.9653 & 21.9653 & 6.7740 \\
   1.5 & -0.75  & 0.9948  & 0.9948  & 0.2487     & 27.2234 & 27.2234 & 6.8058     & 20.5014 & 20.5014 & 5.1254 \\
   1.5 & -0.667 & 0.7914  & 0.7914  & 0.1565     & 25.6778 & 25.6778 & 5.0772     & 19.0767 & 19.0767 & 3.7720 \\
   1.5 & -0.5   & 0.4491  & 0.4491  & 0.0499     & 22.6628 & 22.6628 & 2.5181     & 16.3435 & 16.3435 & 1.8159 \\
   1.5 & -0.333 & 0.2006  & 0.2006  & 0.0099     & 19.7986 & 19.7986 & 0.9758     & 13.8117 & 13.8117 & 0.6807 \\
   1.5 & -0.25  & 0.1133  & 0.1133  & 0.0031     & 18.4397 & 18.4397 & 0.5122     & 12.6364 & 12.6364 & 0.3510 \\
   1.5 & -0.167 & 0.0507  & 0.0507  & 0.0006     & 17.1281 & 17.1281 & 0.2123     & 11.5203 & 11.5203 & 0.1428 \\
   1.5 & 0.0    & 0.0000  & 0.0000  & 0.0000     & 14.6443 & 14.6443 & 0.0000     & 9.4640  & 9.4640  & 0.0000 \\
   1.5 & 0.167  & 0.0507  & 0.0507  & 0.0006     & 12.3858 & 12.3858 & 0.1535     & 7.6760  & 7.6760  & 0.0951 \\
   1.5 & 0.25   & 0.1133  & 0.1133  & 0.0031     & 11.3529 & 11.3529 & 0.3154     & 6.8916  & 6.8916  & 0.1914 \\
   1.5 & 0.333  & 0.2006  & 0.2006  & 0.0099     & 10.3820 & 10.3820 & 0.5117     & 6.1783  & 6.1783  & 0.3045 \\
   1.5 & 0.5    & 0.4491  & 0.4491  & 0.0499     & 8.6227  & 8.6227  & 0.9581     & 4.9621  & 4.9621  & 0.5513 \\
   1.5 & 0.667  & 0.7914  & 0.7914  & 0.1565     & 7.1299  & 7.1299  & 1.4098     & 4.0411  & 4.0411  & 0.7990 \\
   1.5 & 0.75   & 0.9948  & 0.9948  & 0.2487     & 6.4888  & 6.4888  & 1.6222     & 3.6932  & 3.6932  & 0.9233 \\
   1.5 & 0.833  & 1.2192  & 1.2192  & 0.3760     & 5.9148  & 5.9148  & 1.8241     & 3.4176  & 3.4176  & 1.0540 \\
   1.5 & 1.0    & 1.7310  & 1.7310  & 0.7693     & 4.9627  & 4.9627  & 2.2057     & 3.0782  & 3.0782  & 1.3681 \\
   1.5 & 1.25   & 2.6330  & 2.6330  & 1.8284     & 4.0340  & 4.0340  & 2.8014     & 3.0873  & 3.0873  & 2.1440 \\
   1.5 & 1.333  & 2.9643  & 2.9643  & 2.3410     & 3.8531  & 3.8531  & 3.0429     & 3.2206  & 3.2206  & 2.5434 \\
   1.5 & 1.5    & 3.6725  & 3.6725  & 3.6725     & 3.6725  & 3.6725  & 3.6725     & 3.6725  & 3.6725  & 3.6725 \\
   1.5 & 2.0    & 6.0458  & 6.0458  & 10.7481    & 4.4580  & 4.4580  & 7.9254     & 6.3144  & 6.3144  & 11.2255 \\
   1.5 & 2.5    & 8.6261  & 8.6261  & 23.9614    & 6.8533  & 6.8533  & 19.0368    & 10.4359 & 10.4359 & 28.9887 \\
   1.5 & 3.0    & 11.2295 & 11.2295 & 44.9178    & 10.3536 & 10.3536 & 41.4143    & 15.4728 & 15.4728 & 61.8913 \\
   1.5 & 4.0    & 16.0441 & 16.0441 & 114.0913   & 18.9646 & 18.9646 & 134.8595   & 26.5208 & 26.5208 & 188.5926 \\
   1.5 & 6.0    & 23.1266 & 23.1266 & 370.0260   & 35.9359 & 35.9359 & 574.9741   & 46.2619 & 46.2619 & 740.1909 \\\hline
    \end{tabular}                                                                                                                                                                          
    \caption[Scalings of $\sigma\times$BR for the signal components and \CV=1.5\ ]{Scalings of cross section times BR for the non-resolved model, for the different \ttH, \tHq, \tHW\ signal components and \CV=1.5\ .}\label{tab:xsbrscalingK6_1p5}                              
 \end{table}   

\end{landscape}







\begin{table}[h!]
  \centering
  \footnotesize
  \begin{tabular}{lcccccccc}\hline
                     & \multicolumn{3}{c}{Cross section (pb)}                & \multicolumn{2}{c}{95\% C.L. Limits}          & \multicolumn{2}{c}{Cross section limits (pb)} \\ 
  $\cos(\alpha_{CP})$& \tHq                & \tHW                & \ttH      &     Exp.               & Obs.  & Exp.                   & Obs.   \\ \hline
    -1.0             & $0.794^{+2.8}_{-4.0}$ & $0.146^{+0.2}_{-0.2}$ & 0.503436  & $0.691^{0.204}_{-0.299}$ & 1.489 & $0.302^{0.089}_{-0.131}$ & 0.651  \\
    -0.9             & $0.728^{+2.7}_{-4.1}$ & $0.135^{+0.2}_{-0.2}$ & 0.426117  & $0.768^{0.228}_{-0.333}$ & 1.632 & $0.300^{0.089}_{-0.130}$ & 0.637  \\
    -0.8             & $0.664^{+2.7}_{-4.2}$ & $0.123^{+0.2}_{-0.2}$ & 0.355670  & $0.860^{0.256}_{-0.374}$ & 1.797 & $0.298^{0.088}_{-0.130}$ & 0.622  \\
    -0.7             & $0.601^{+2.8}_{-4.0}$ & $0.112^{+0.2}_{-0.2}$ & 0.295533  & $0.965^{0.287}_{-0.423}$ & 1.984 & $0.295^{0.088}_{-0.129}$ & 0.606  \\
    -0.6             & $0.546^{+2.9}_{-4.3}$ & $0.102^{+0.2}_{-0.2}$ & 0.242268  & $1.082^{0.323}_{-0.476}$ & 2.172 & $0.292^{0.087}_{-0.128}$ & 0.586  \\
    -0.5             & $0.497^{+3.1}_{-4.2}$ & $0.092^{+0.2}_{-0.2}$ & 0.197595  & $1.209^{0.362}_{-0.534}$ & 2.362 & $0.288^{0.086}_{-0.127}$ & 0.563  \\
    -0.4             & $0.446^{+3.1}_{-4.5}$ & $0.083^{+0.2}_{-0.2}$ & 0.159794  & $1.362^{0.409}_{-0.605}$ & 2.595 & $0.284^{0.085}_{-0.126}$ & 0.542  \\
    -0.3             & $0.398^{+3.2}_{-4.6}$ & $0.074^{+0.2}_{-0.2}$ & 0.132302  & $1.538^{0.463}_{-0.684}$ & 2.870 & $0.282^{0.085}_{-0.125}$ & 0.526  \\
    -0.2             & $0.353^{+3.5}_{-4.8}$ & $0.066^{+0.2}_{-0.2}$ & 0.111684  & $1.739^{0.524}_{-0.776}$ & 3.205 & $0.280^{0.084}_{-0.125}$ & 0.515  \\
    -0.1             & $0.314^{+3.7}_{-4.9}$ & $0.059^{+0.2}_{-0.2}$ & 0.099656  & $1.952^{0.588}_{-0.870}$ & 3.597 & $0.280^{0.084}_{-0.125}$ & 0.515  \\
    0.0              & $0.275^{+3.6}_{-5.2}$ & $0.052^{+0.2}_{-0.2}$ & 0.094502  & $2.208^{0.664}_{-0.982}$ & 4.149 & $0.282^{0.085}_{-0.125}$ & 0.530  \\
    0.1              & $0.242^{+4.0}_{-5.5}$ & $0.045^{+0.2}_{-0.2}$ & 0.099656  & $2.458^{0.737}_{-0.089}$ & 4.822 & $0.288^{0.086}_{-0.128}$ & 0.565  \\
    0.2              & $0.211^{+4.1}_{-5.8}$ & $0.040^{+0.2}_{-0.2}$ & 0.111684  & $2.690^{0.801}_{-0.178}$ & 5.583 & $0.296^{0.088}_{-0.129}$ & 0.613  \\
    0.3              & $0.182^{+4.1}_{-6.1}$ & $0.035^{+0.2}_{-0.2}$ & 0.132302  & $2.856^{0.842}_{-0.234}$ & 6.242 & $0.302^{0.089}_{-0.131}$ & 0.661  \\
    0.4              & $0.156^{+4.4}_{-6.5}$ & $0.030^{+0.2}_{-0.2}$ & 0.159794  & $2.898^{0.845}_{-0.235}$ & 6.536 & $0.304^{0.089}_{-0.129}$ & 0.685  \\
    0.5              & $0.134^{+4.5}_{-6.6}$ & $0.026^{+0.2}_{-0.2}$ & 0.197595  & $2.739^{0.793}_{-0.157}$ & 6.281 & $0.297^{0.086}_{-0.125}$ & 0.681  \\
    0.6              & $0.116^{+4.7}_{-6.9}$ & $0.023^{+0.2}_{-0.2}$ & 0.242268  & $2.460^{0.710}_{-0.038}$ & 5.692 & $0.284^{0.082}_{-0.120}$ & 0.658  \\
    0.7              & $0.100^{+5.0}_{-7.1}$ & $0.020^{+0.2}_{-0.2}$ & 0.295533  & $2.138^{0.617}_{-0.904}$ & 4.971 & $0.269^{0.078}_{-0.114}$ & 0.626  \\
    0.8              & $0.087^{+4.8}_{-7.1}$ & $0.018^{+0.2}_{-0.2}$ & 0.357388  & $1.830^{0.528}_{-0.776}$ & 4.263 & $0.256^{0.074}_{-0.109}$ & 0.597  \\
    0.9              & $0.077^{+4.7}_{-7.0}$ & $0.017^{+0.2}_{-0.2}$ & 0.426117  & $1.549^{0.448}_{-0.658}$ & 3.610 & $0.244^{0.071}_{-0.104}$ & 0.569  \\
    1.0              & $0.071^{+4.2}_{-6.7}$ & $0.016^{+0.2}_{-0.2}$ & 0.503436  & $1.322^{0.383}_{-0.562}$ & 3.080 & $0.236^{0.068}_{-0.100}$ & 0.551  \\\hline
\end{tabular}
  \caption[Cross sections for \tHq, \tHW and \ttH as a function of $\cos(\alpha_{CP})$ ]{Production cross sections for \tHq, \tHW and \ttH at $\sqrt{s}=13$ TeV, as a function of $\cos(\alpha_{CP}$). The quoted uncertainties on the cross section correspond to scale variations in \%. The used \ttH NLO cross sections are obtained from \cite{maltoni2} and are interpolated to the angles for which the LHE weights in the signal MC samples are available. Also listed are the expected and observed asymptotic limits at 95\% C.L. for all studied CP-mixing angles. The uncertainties on the expected limit correspond to $\pm 1\sigma$.}\label{tab:cp_xsec}
\end{table}


























































































\clearpage

\section{\tHq-\ttH\ overlap}
\label{sec:overlap}
\input{overlap.tex}
\clearpage



\bibliography{auto_generated}   % will be created by the tdr script.

%% examples of appendices. **DO NOT PUT \end{document} at the end
%\clearpage
% \appendix
% \section{PTDR symbol definitions\label{app:symdef}}


%%% DO NOT ADD \end{document}!

