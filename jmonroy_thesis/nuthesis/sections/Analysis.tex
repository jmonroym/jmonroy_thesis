\hyphenation{Colli-sions}
\hyphenation{Single}
%______________________ Analysis ______________________
\chapter{Search for production of a Higgs boson and a single top quark in multilepton final states in pp collisions at $\sqrt{s}=13$ TeV}\label{ch:analysis}

%______________________ INTRODUCCION ______________________
\section{Introduction}\label{sec:Intro_analysis}

The Higgs boson discovery, supported on experimental observations and theoretical predictions made about the SM, gives the clue of the way in that elementary particles acquire mass through the Higgs mechanism; therefore, knowing the Higgs mass, the Higgs-vector boson and Higgs-fermion couplings can be determined. In order to test the Higgs-top coupling, several measurements have been performed, as stated in the chapter \ref{ch:theory}, but they are limited in sensitivity to measure the square of the coupling. The production of a Higgs boson in association with a single top quark (\tH) not only offers access to the sign of the coupling, but also, to the CP phase of the Higgs couplings.

This chapter presents the search for the associated production of a Higgs boson and a single top quark (\tHq) events, focusing on leptonic signatures provided by the Higgs decay modes to $\WW$, $\ZZ$, and $\tautau$; the 13 TeV dataset produced in 2016, which corresponds to an integrated luminosity of 35.9\fbinv, is used.

As shown in Section \ref{sec:thq}, the SM cross section of \tHq process is driven by a destructive interference between two contributions (see Figure \ref{fig:th_prod}), where the Higgs couples to either the W boson or the top quark; however, if the sign of the Higgs-top coupling is flipped with respect to the SM prediction, a large enhancement of the cross section occurs, making this analysis sensitive to such deviation. A second process, where the Higgs boson and top quark are accompanied by a W boson (\tHW) has similar behavior, albeit with a weaker interference pattern and lower contribution to the cross section, therefore, a combination of both processes would increase the sensitivity to the sign of the coupling; in this analysis both contributions are combined and referred as \tH channel. A third contribution comes from \ttH process. The purpose of this analysis is to investigate the exclusion of the presence of the \tH + \ttH processes under the assumption of the anomalous Higgs-top coupling modifier (\Ct=-1). The analysis exploits signatures with two leptons of the same sign (\ti{2lss}) channel and three leptons (\ti{3l}) channel in the final state.

Constraints on the sign of the Higgs-top coupling ($y_t$) have been derived from the decay rate of Higgs boson to photon pairs \cite{biswas} and from the cross section for associated production of Higgs and Z bosons via gluon fusion \cite{hespel}, with recent results disfavoring negative signs of the coupling \cite{cms_ht_couplings,comb_ht_couplings,diboson}, although the negative sign coupling have not been completely excluded.

The analysis presented here, expands previous analyses performed at 8 TeV ~\cite{Khachatryan_2015,CMS_AN_2014-140} and searches for associated production of \ttbar pair and a Higgs boson in the multilepton final state channel ~\cite{CMS_AN_2016-211}; it also complements searches in other decay channels targeting $H\to b\bar{b}$~\cite{CMS_PAS_HIG_16-019}.

The first sections present the characteristic \tHq signature as well as the expected backgrounds. The MC samples, data sets, and the physics object definitions are then defined. Following, the background predictions, the signal extraction, and the statistical treatment of the selected events as well as the systematic uncertainties are described. The final section present the results for the exclusion limits as a function of the ratio of \Ct and the dimensionless modifier of the Higgs-vector boson coupling \CV.  

\begin{figure}[!h]
\begin{center}
\includegraphics[width=\textwidth]{workflow}
\end{center}
\caption[Analysis strategy workflow]{A schematic overview of the analysis workflow. Based on sets of optimized physics object definitions and selection criteria, signal and background events in a data sample are discriminated. The discrimination is performed by a BDT, previously trained using MC samples of the dominant backgrounds, using discriminant variables based on the \bjet multiplicity, the activity in the forward region of the detector, and the kinematic properties of leptons. The $CL_s$ limits on the combined \ttH + \tH production cross section, as a function of the relative coupling strengths are calculated. }
\label{fig:workflow}
\end{figure}

The analysis is designed to efficiently identify and select prompt leptons from on-shell W and Z boson decays and to reject non-prompt leptons from $b$ quark decays and spurious lepton signatures from hadronic jets. Events are then selected in the $2lss$ and $3l$ channels, and are required to contain hadronic jets, some of which must be consistent with $b$ quark hadronization. Finally, the signal yield is extracted by simultaneously fitting the output of two dedicated multivariate discriminants, trained to separate the \tHq signal from the two dominant backgrounds, in all categories. The fit result is then used to set an upper limit on the combined \ttH + \tH production cross section, as a function of the relative coupling strengths of Higgs-top quark and Higgs-Vector boson. Figure \ref{fig:workflow} shows an schematic overview of the analysis strategy workflow. 

With respect to the 8 TeV analysis, the object selections have been adjusted for the updated LHC running conditions at 13 TeV, the lepton identification has been improved, and more powerful multivariate analysis techniques are used for the signal extraction.

The analysis has been made public by CMS as a Physics Analysis Summary \cite{CMS_PAS_HIG_17-005} combining the result for the three lepton and two lepton same-sign channels; the content present in this chapter is based on that document and on References ~\cite{CMS_AN_2016-211, CMS_AN_2017-029} unless other Reference is stated. Currently, an effort to turn the analysis into a paper combining the multilepton and $H \to b\bar{b}$ is ongoing. 

%% dont forget to describe explicitely the analysis strategy here or later but do it

%---------------------------------tHq signature
\section{\tHq signature}\label{sec:thq_sign}

\begin{figure}[!h]
\begin{center}
\includegraphics[width=\textwidth]{thq_sign}
\end{center}
\caption[\tHq event signature]{\tHq event signature. Left: Feynman diagram including the whole evolution up to the final state for the case of the Higgs boson emitted by the W boson (top); Feynman diagram for the case where the Higgs boson is emitted by the top quark. Right: Schematic view as it would be seen in the detector; the circle in the Feynman diagrams on the left corresponds to the circle in the center of the schematic view as indicated by the line connecting them. In the $2lss$ channel, one of the W bosons from the Higgs boson decays to two light-quark jets while in the $3l$ channel both W bosons decay to leptons.}
\label{fig:thq_sign}
\end{figure}

In order to select events of \tHq process, its features are translated into a set of selection rules; Figure \ref{fig:thq_sign} shows the Feynman diagram and an schematic view of the \tHq process from the \pp collision to the final state configuration. A single top quark is produced accompanied by a light quark, denoted as q; this light quark is produced predominantly in the forward region of the detector. The Higgs boson can be either emitted by the exchanged W boson or directly by the singly produced top quark.

Due to their high masses/short lifetimes, top quark and Higgs boson decay after their production within the detector. The Higgs boson is required to decay into a W boson pair\footnote{ZZ and $\tau\tau$ decays are also include in the analysis but they are not separately reconstructed}. The top quark almost always decays into a bottom quark and a W boson, as encoded in the CMK matrix. The W bosons are required to decay leptonically either all the three in the $3l$ channel or the pair with equal electrical charge in the $2lss$ channel case; $\tau$ leptons are not reconstructed separately and only their leptonic decays into either electrons or muons are considered in this analysis.

In summary, the signal process is characterized by a the final state with

\begin{itemize}
\item one light-flavored forward jet,
\item one central b-jet,
\item $2lss$ channel $\to$ two leptons of the same sign, two neutrinos and two light (often soft) jets,
\item $3l$ channel $\to$ three leptons, three neutrinos and no central light-flavored jets,
\end{itemize}

The presence of neutrinos is inferred from the presence of MET.

%---------------------------------bg processes

\section{Background processes}\label{sec:bg}

The background processes are those that can mimic the signal signature or at least can be reconstructed as that as a result of certain circumstances. The backgrounds can be classified as

\begin{itemize}

\item irreducible backgrounds: where genuine prompt leptons are produced in on-shell W and Z boson decays; they can be reliantly estimated directly from MC simulated events, using higher-order cross sections or data control regions for the overall normalization.

\item reducible backgrounds: where at least one of the leptons is \ti{non-prompt}, \ie, produced within a hadronic jet; genuine leptons from heavy flavor decays and misreconstructed jets, also known as \ti{mis-ID leptons} or \ti{fake leptons}, are considered non-prompt leptons. These non-prompt leptons leave tracks and hits in the detection systems as would a prompt lepton, but correlating those hits with nearby jets could be a way of removing them. The misassignment of electron charge in processes like \ttbar or Drell-Yan, represent an additional source of background, but it is relevant only for the $2lss$ channel. Reducible backgrounds are not well predicted by simulation, hence, they are estimated using data-driven methods. 
\end{itemize}

The main sources of background events for \tHq process are \ttbar process and \ttbar + $X (X=W,Z,\gamma)$ processes, the latter regarded together as \ttV process. Figure \ref{fig:ttbar_sign} shows the signature for \ttbar and \ttW processes.     

\begin{figure}[!htb]
\centering
\includegraphics[width=\textwidth]{ttbar_signature}
\caption[\ttbar and \ttW events signature]{\ttbar(left) and \ttW(right) events signature as they would be seen in the detector; the Feynman diagrams including the whole evolution up to the final state are also showed. The \ttbar process signature is very similar to that of the signal process with one fake lepton and non forward activity. The \ttW process present a higher b-jet multiplicity compared to the signal process, a prompt lepton and no forward activity.}
\label{fig:ttbar_sign}
\end{figure}

The largest contribution to irreducible backgrounds comes from \ttW and \ttZ processes for which the number of ($b-$)jets (($b-$)jet multiplicity) is higher than that of the signal events, while for other contributing background events,  \WZ, $ZZ$, and rare SM processes like $W^\pm W^\pm qq$, $\ttbar\ttbar$, $tZq$, $tZW$, $WWW$, $WWZ$, $WZZ$, $ZZZ$, the ($b-$)jet multiplicity is lower compared to that of the signal events. None of the irreducible backgrounds present activity in the forward region of the detector.

On the side of the reducible backgrounds, the largest contribution comes from the \ttbar events which have a very similar signature to the signal events but does no present activity in the forward region of the detector either; A particular feature of the \ttbar events is their charge-symmetry, which is also a difference with respect to the signal events.
%______________________ Samples  ______________________
\section{Data and MC Samples} \label{secc:samples}

\subsection{ Full 2016 data set}

The data set used in this analysis was collected by the CMS experiment during 2016 at while running at $\sqrt{s}=13$TeV and corresponds to a total integrated luminosity of 35.9\fbinv. Only periods when the CMS magnet was on were considered when selecting the data samples; that corresponds to the \verb|23Sep2016 (Run B to G)| and \verb|PromptReco (Run H)| versions of the datasets.

Multilepton final states with either two same-sign leptons or three leptons target the case where the Higgs boson decays to a pair of W bosons, $\tau$ leptons, or Z bosons, and where the top quark decays leptonically, hence, the \verb|SingleElectron|, \verb|SingleMuon|, \verb|DoubleEG|, \verb|MuonEG|, \verb|DoubleMuon| dataset (see Table \ref{tab:dataset}) compose the full dataset. The certified luminosity sections are selected using the golden JSON file defined by the CMS experiment \cite{json}.

\subsection{Triggers}

The events considered are those online-reconstructed events triggered by one, two, or three leptons. Single-lepton triggers are included in order to boost the acceptance of events where the \pt of the sub-leading lepton falls below the threshold of the double-lepton triggers. The trigger efficiency is increased by including double-lepton triggers in the $3l$ category, and single-lepton triggers in all categories; it is possible given the logical ``or'' of the trigger decisions of all the individual triggers in a given category. Table ~\ref{tab:triggers} shows the lowest-threshold non-prescaled triggers present in the High-Level Trigger (HLT) menus for both Monte-Carlo and data in 2016.

\subsubsection*{Trigger efficiency scale factors}

\begin{figure}[htp]
\centering
\includegraphics[width=0.49\textwidth]{plots_trigger/1D_eff_lep1_pt_uu_ARCv2_change_3l_pt_ranges.pdf}
\includegraphics[width=0.49\textwidth]{plots_trigger/1D_eff_lep2_pt_uu_ARCv2_change_3l_pt_ranges.pdf} \\
\includegraphics[width=0.49\textwidth]{plots_trigger/1D_eff_lep1_eta_uu_ARCv2_change_3l_pt_ranges.pdf}
\includegraphics[width=0.49\textwidth]{plots_trigger/1D_eff_lep2_eta_uu_ARCv2_change_3l_pt_ranges.pdf}
\caption[Trigger efficiency for the same-sign $\mu\mu$ category]{Comparison between data an MC trigger efficiencies in the same-sign $\mu\mu$ category, as as a function of the \pt and  \etac of the leading lepton (left) and the sub-leading lepton (right) \cite{CMS_AN_2017-029}.}
\label{fig:trigeffsmumu}
\end{figure}

\begin{figure}[htp]
\centering
\includegraphics[width=0.49\textwidth]{plots_trigger/1D_eff_lep1_pt_eu_ARCv2_change_3l_pt_ranges.pdf}
\includegraphics[width=0.49\textwidth]{plots_trigger/1D_eff_lep2_pt_eu_ARCv2_change_3l_pt_ranges.pdf} \\
\includegraphics[width=0.49\textwidth]{plots_trigger/1D_eff_lep1_eta_eu_ARCv2_change_3l_pt_ranges.pdf}
\includegraphics[width=0.49\textwidth]{plots_trigger/1D_eff_lep2_eta_eu_ARCv2_change_3l_pt_ranges.pdf}
\caption[Trigger efficiency for the $e\mu$ category]{Comparison between data an MC trigger efficiencies in the same-sign $e\mu$ category as as a function of the \pt and $\eta$ of the leading lepton (left) and the sub-leading lepton (right) \cite{CMS_AN_2017-029}.}
\label{fig:trigeffsemu}
\end{figure}

%\begin{figure}[htp]
%\centering
%\includegraphics[width=0.49\textwidth]{plots_trigger/1D_eff_lep1_pt_ee_ARCv2_change_3l_pt_ranges.pdf}
%\includegraphics[width=0.49\textwidth]{plots_trigger/1D_eff_lep2_pt_ee_ARCv2_change_3l_pt_ranges.pdf} \\
%\includegraphics[width=0.49\textwidth]{plots_trigger/1D_eff_lep1_eta_ee_ARCv2_change_3l_pt_ranges.pdf}
%\includegraphics[width=0.49\textwidth]{plots_trigger/1D_eff_lep2_eta_ee_ARCv2_change_3l_pt_ranges.pdf}
%\caption[Trigger efficiency for the $ee$ category]{Comparison between data an MC trigger efficiencies in the same-sign $e\mu$ category ($1^{rs}$ and $2^{nd}$ rows) and same-sign $ee$ category ($3^{rd}$ and $4^{th}$ rows), as as a function of the \pt and $\eta$ of the leading lepton (left) and the sub-leading lepton (right) \cite{CMS_AN_2017-029}.}
%\label{fig:trigeffsee}
%\end{figure}

\begin{figure}[htp]
\centering
\includegraphics[width=0.49\textwidth]{plots_trigger/1D_eff_lep1_pt_3l_ARCv2_change_3l_pt_ranges.pdf}
\includegraphics[width=0.49\textwidth]{plots_trigger/1D_eff_lep2_pt_3l_ARCv2_change_3l_pt_ranges.pdf} \\
\includegraphics[width=0.49\textwidth]{plots_trigger/1D_eff_lep1_eta_3l_ARCv2_change_3l_pt_ranges.pdf}
\includegraphics[width=0.49\textwidth]{plots_trigger/1D_eff_lep2_eta_3l_ARCv2_change_3l_pt_ranges.pdf}
\caption[Trigger efficiency for the $3l$ category]{Comparison between data an MC trigger efficiencies in the $3l$ category, as as a function of the \pt and $\eta$ of the leading lepton (left) and the sub-leading lepton (right) \cite{CMS_AN_2017-029}.}
\label{fig:trigeffs3l}
\end{figure}

Trigger efficiency describes the ability of events to pass the trigger requirements. It is measured in simulated events using generator information given that there is no trigger bias with the MC sample. Measuring the trigger efficiency in data requires a more elaborated procedure; first, select a set of events collected by a trigger that is uncorrelated with the lepton triggers such that the selected events form an unbiased sample. In this analysis, that uncorrelated trigger is a MET trigger. Second step is looking for candidate events with exactly two good leptons (exactly three good leptons for the $3l$ channel). Finally,  measure the efficiency for the candidate events to pass the logical ``or'' of triggers being considered in a given event category as defined in Table ~\ref{tab:triggers}.

Comparisons between the data and MC efficiencies for each category, showed in Figures~\ref{fig:trigeffsmumu}, \ref{fig:trigeffsemu}, and \ref{fig:trigeffs3l}, reveal that they are in good agreement; the difference is corrected by applying scale factors derived from the ratio between both efficiencies.

Applied flat scale factors in each category are shown in Table ~\ref{tab:trigSFs}; they have been inherited from Reference \cite{CMS_AN_2017-029}. 
\begin{table}
\centering
\begin{tabular}{ll}
Category & Scale Factor \\\hline
    ee   & $1.01 \pm 0.02$ \\
e$\mu$   & $1.01 \pm 0.01$ \\
$\mu\mu$ & $1.00 \pm 0.01$ \\
3l       & $1.00 \pm 0.03$ \\\hline
\end{tabular}
\caption[Trigger efficiency scale factors and associated uncertainties.]{Trigger efficiency scale factors and associated uncertainties, shown here rounded to the nearest percent.}
\label{tab:trigSFs}
\end{table}

\subsection{Signal modeling and MC samples}

Current event generators allow for adjusting the kinematics of the generated events, based on an event-wise reweighting; in this way, several generation parameters phase spaces can be explored according to the experimental interests. The signal samples used in this analysis were generated in such a way that not only the case \Ct=-1, but an extended range of \Ct and \CV values may be investigated.

\begin{figure}[htp]
\centering
\includegraphics[width=0.49\textwidth]{tHQ_xsec_cfcv}
\includegraphics[width=0.49\textwidth]{tWH_xsec_cfcv} 
\caption[\tHq and \tHW cross section in the \Ct-\CV phase space]{\tHq and \tHW cross section in the \Ct-\CV phase space \cite{THQProdTwiki}.}
\label{fig:ktkv_phase_space}
\end{figure}

\tHq and \tHW cross section in the \Ct-\CV phase space are shown in Figure \ref{fig:ktkv_phase_space}. As said in section \ref{sec:event_generation}, the \tHq sample was generated using the 4F scheme which provides a better description of the additional $b$ quark from the initial gluon splitting, while the \tHW sample was generated using the 5F scheme in order to remove its the interference with \ttH at LO.


\subsubsection*{MC signal samples}

The two signal samples, \tHq\ and \tHW, correspond to the \verb|RunIISummer16MiniAODv2| campaign produced with CMSSW\_80X; they were produced with \textsc{MG5\_}a\textsc{MC@NLO} (version 5.2.2.3), in LO order mode at $\sqrt{s}=13$ TeV, and are normalized to NLO cross sections (see Table~\ref{tab:sigsamples}). The Higgs boson is assumed to be SM-like except for the values of its couplings to the top quark and W boson. Each sample was generated with a set of event weights corresponding to 51 different values of (\Ct, \CV) couplings, accessible in terms of LHE event weights as shown in Table ~\ref{tab:reweight}; however, the main interest is the (\Ct=-1,\CV=1) case. 

\begin{table}[h]
\centering \small
\begin{tabular}{lll}
Sample & $\sigma$ [pb] & BF \\ \hline
\verb|/THQ_Hincl_13TeV-madgraph-pythia8_TuneCUETP8M1/|                  & 0.7927 & 0.324 \\
\verb|/THW_Hincl_13TeV-madgraph-pythia8_TuneCUETP8M1/|                  & 0.1472 & 1.0   \\\hline
\verb|/ttHJetToNonbb_M125_13TeV_amcatnloFXFX_madspin_pythia8_mWCutfix/|   & 0.2151 & 1.0 \\\hline
\end{tabular}
\caption[MC signal samples.]{MC signal samples used in this analysis; cross section and branching fraction are also listed ~\cite{THQProdTwiki}.}\label{tab:sigsamples}
\end{table}

The \ttH sample was produced using \textsc{AMC@NLO} interfaced to \textsc{PYTHIA} 8 for the parton shower, and is scaled to NLO cross sections. The \ttH cross section depends quadratically on \Ct; however, in contrast to the \tHq and tHW samples, the scaling is not performed during the sample generation process but in the analysis code since it was decided to include the \ttH process as part of the signal in the course of the analysis.     

% add something about why tth is included as signal and why? if it is not sensitive to kt sign , 

\subsubsection*{MC background samples}

Several MC generators were used to generate the samples of the background processes. The dominant background sources (\ttbar, \ttW, \ttZ) were produced using \textsc{aMC@NLO} interfaced to PYTHIA8, and are scaled to NLO cross sections. Other minor background processes are simulated using POWHEG interfaced to PYTHIA, or bare PYTHIA as stated in the sample names in Table ~\ref{tab:bgsamples}. Pileup interactions are included in the simulation in order to reflect the observed multiplicity in data; the simulated events are weighted according to the actual pileup in data, estimated from the measured bunch-to-bunch instantaneous luminosity and the total inelastic cross section, 69.2 mb. All events are finally passed through a full simulation of the CMS detector based on GEANT4, and reconstructed using the same algorithms as used for the data.
%______________________ object id ______________________
\section{Object Identification}\label{sec:ob_id}


In this section, the specific definitions of the physical objects in terms of the numerical values assigned to the reconstruction parameters are presented; thus, the provided details summarize and complement the descriptions presented in previous chapters. The object reconstruction and selection strategy used in this thesis is inherited from the analyses in References \cite{CMS_AN_2016-211,CMS_AN_2017-029}, thus, the information in this section is extracted from those documents unless other References are stated.

\subsection{Lepton reconstruction and identification}

Two types of leptons are defined in this analysis: \ti{signal leptons} are those coming from $W, Z$ and $\tau$ decays which usually are isolated from other particles; \ti{background leptons} are defined as leptons produced in \bjet hadron decays, light-jets misidentification, and photon conversions. 

The process of reconstruction and identification of electron and muon candidates was described in chapter\ref{ch:gensimreco}, hence, the identification variables used in order to retain the highest possible efficiency for signal leptons while maximizing the rejection of background leptons are listed and described in the following sections \footnote{the studies performed to optimize the identification are far from the scope of this thesis, therefore, only general descriptions are provided}.

The identification variables include not only observables related directly to the reconstructed leptons themselves, but also to the clustered energy deposits and charged particles in a cone around the lepton direction (jet-related variables); an initial loose preselection of leptons candidates is performed and then an MVA discriminator, referred to as \ti{lepton MVA} discriminator, is used to distinguish signal leptons from background leptons.

\subsubsection*{Muons}

The Physics Objects Groups (POG) at CMS, are in charge of studying and defining the set of selection criteria applied on the course of reconstruction and identification of particles. These selection criteria are implemented in the CMS framework in the form of several object identification working points according to the strength of the requirements.

The muon candidates are reconstructed by combining information from the tracker system and the muon detection system of CMS detector and the POG defined three working points for muon identification \ti{MuonID}\cite{muid};

\begin{itemize}
\item \ti{POG Loose Muon ID} is a particle identified as a muon by the PF event reconstruction and also reconstructed either as a global-muon or as an arbitrated tracker-muon. This identification criteria is designed to be highly efficient for prompt muons and for muons from heavy and light quark decays; it can be complemented by applying impact parameter cuts in analyses with prompt muon signals.
\item \ti{POG Medium Muon ID} is a Loose muon with additional track-quality and muon-quality (spatial matching between the individual measurements in the tracker and the muon system) requirements. This identification criteria is designed to be highly efficient in the separation of the muons coming from decay in flight of heavy quarks and muons coming from B meson decays as well as prompt muons. An additional category \ti{MVA Prompt ID} is defined in this identification criteria directed to discriminated muons from B mesons and prompt muons (from W,Z and $\tau$ decays). The Medium ID provides the same fake rate as the Tight Muon ID but a higher efficiency on prompt and B-decays muons.\cite{medium_muon}
\item \ti{POG Tight Muon ID} is a global muon with additional muon-quality requirements Tight Muon ID selects a subset of the PF muons.  
\end{itemize}

Only muons within the muon system acceptance $|\eta| < 2.4$ and minimum \pt  of 5 GeV are considered. %The requirements for each working point are listed in table \ref{tab:muonid}. 

\subsubsection*{Electrons}

Electrons are reconstructed using information from the tracker and from the electromagnetic calorimeter and identified by an MVA algorithm (\ti{MVA eID} discriminant) using the shape of the calorimetric shower variables like the shape in \etac and \phic, the cluster circularity, widths along \etac and \phic; track-cluster matching variables like $E_{tot}/p_{in}$, $E_{Ele}/p_{out}$, $\Delta \eta_{in}$,  $\Delta \eta_{out}$, $\Delta \phi_{in}$, $1/E - 1/p$; and track quality variables like $\chi^2$ of the GSF tracks, the number of hits used by the GSF filter\cite{mva_eid}.

A loose selection based on \etac-dependent cuts on this discriminant is used to preselect electron candidates, the full shape of the discriminant is used in the lepton MVA selection to separate signal leptons from background leptons (described in Section \ref{sssec:leptonmva}).

In order to reject electrons from photon conversions, electron candidates with missing hits in the pixel tracker layers or matched to a conversion secondary vertex are discarded. Electrons are selected for the analysis if they have \pt > 7 GeV and are located within the tracker system acceptance region ($|\eta|$ < 2.5). %All the electron selection criteria are listed in table\ref{tab:eleid}.

\subsubsection*{Lepton vertexing and pile-up rejection}

The impact parameter in the transverse plane $d_0$ , impact parameter along the $z$-axis $d_z$ , and the impact parameter significance in the detector space $SIP_{3D}$, are considered to perform the identification and rejection of pile-up, misreconstructed tracks, and background leptons from b-hadron decays; pile-up and misreconstructed track mitigation is achieved by imposing loose cuts on the impact parameter variables. The full shape of the those variables is used in a lepton MVA classifier to achieve the best separation between the signal and the background leptons.

\subsubsection*{Lepton isolation}

PF is able to recognize leptons from two different sources: on one side, leptons from the decays of heavy particles, such as W and Z bosons, which are normally isolated in space from the hadronic activity in the event; on the other side, leptons from the decays of hadrons and jets misidentified as leptons, which are not isolated as the former. For highly boosted systems, like the lepton and the $b$-jet generated in the semileptonic decay of a boosted top, the decay products tend to be more closer and sometimes they even overlap; thus, the PF standard definition of isolation in terms of the separation between the lepton candidates and other PF objects in the \etac-\phic plane,

\beqn
\Delta R =\sqrt{( \eta^l - \eta^i)^2 + (\phi^l - \phi^i)^2} < 0.3
\eeqn

\noindent which considers all the neutral, charged hadrons and photons in a cone around the leptons, is refocused to the local isolation of the leptons through the mini-isolation $I_{mini}$ \cite{i_mini} defined as the sum of particle flow candidates \pt within a cone around the lepton, corrected for the effects of pileup and divided by the lepton \pt
\beqn
I_{mini} =\frac{\displaystyle\sum_{R} p_T(h^\pm) - \textrm{max}\left(0,\displaystyle\sum_R p_T(h^0) + p_T(\gamma)-\rho \mathcal{A}\left(\frac{R}{0.3}\right)^2\right)}{p_T(l)}
\eeqn

\noindent where $\rho$ is the pileup energy density, $h^\pm, h^0, \gamma, l,$ represent the charged hadron, neutral hadrons, photons, and the lepton, respectively. The radius R of the cone depends on the \pt of the lepton according to 
\beqn
R = \frac{10 \textrm{GeV}}{\textrm{min}(\textrm{max}(p_T(l), 50 \textrm{GeV}), 200 \textrm{GeV})},
\eeqn

The \pt dependence of the cone size allows for greater signal efficiency. Setting a cut on $I_{mini}$ below a given threshold ensures that the lepton is locally isolated, even in boosted systems. The effect of pileup is mitigated using the so-called effective area correction $\mathcal{A}$ listed in Table\ref{tab:pileup_area}. 

\begin{table}[!htbp]
\centering
\small
\begin{tabular}{ccc}\hline
$|\eta|$ range & $\mathcal{A}$(e) neutral/charged & A ($\mu$) neutral/charged \\\hline
0.0 - 0.8      & 0.1607 / 0.0188                  & 0.1322 / 0.0191 \\
0.8 - 1.3      & 0.1579 / 0.0188                  & 0.1137 / 0.0170 \\
1.3 - 2.0      & 0.1120 / 0.0135                  & 0.0883 / 0.0146 \\
2.0 - 2.2      & 0.1228 / 0.0135                  & 0.0865 / 0.0111 \\
2.2 - 2.5      & 0.2156 / 0.0105                  & 0.1214 / 0.0091 \\ \hline
\end{tabular}
\caption[Effective areas, for electrons and muons.]{ Effective areas, for electrons and muons used to mitigate the effect of pileup by using the so-called effective area correction.}
\label{tab:pileup_area}
\end{table}

A loose cut on $I_{mini}$ is applied to pre-select the muon and electron candidates; however, the full shape is used in the lepton MVA discriminator when performing the signal lepton selection.

\subsubsection*{Jet-related variables}

In order to reject misidentified leptons from $b-$jets, mostly coming from tt+jets, Drell-Yan+jets, and W+jets events, the vertexing and isolation described in previous sections are complemented with additional variables related to the closest reconstructed jet to the lepton, \ie, the PF jets reconstructed\footnote{charged hadrons from PU vertices are not removed prior to the jet clustering.} around the leptons with $\Delta R=\sqrt{( \eta^l - \eta^{jet})^2 + (\phi^l - \phi^{jet})^2} < 0.5$. The identification variables used in the lepton MVA discriminator are the ratio $\pt^l/\pt^{jet}$, the CSV b-tagging discriminator value of the jet, the number of charged tracks of the jet, and the relative \pt given by

\beqn
p_T^{rel}=\frac{(\vec{p}_{jet}-\vec{p}_l)\cdot\vec{p}_l}{||\vec{p}_{jet} - \vec{p}_l||}.
\eeqn


\subsubsection*{LeptonMVA discriminator}\label{sssec:leptonmva}

Electrons and muons passing the basic selection process described above are referred to as \ti{loose leptons}. Additional discrimination between signal leptons and background leptons is crucial considering that the rate of \ttbar production is much larger than the signal, hence, an overwhelming background from \ttbar production. To maximally exploit the available information in each event to that end, the dedicated lepton MVA discriminator, based on a boosted decision tree (BDT) algorithm, has been built so that all the identification variables can be used together.

The lepton MVA discriminator training is performed using simulated signal Loose leptons from the \ttH MC sample and fake leptons from the \ttbar+jets MC sample, separately for muons and electrons. The input variables used include vertexing, isolation and jet-related variables, the \pt and \etac of the lepton, the electron MVA eID discriminator and the muon segment-compatibility variables. An additional requirement known as \ti{tight-charge} requirement, is imposed by comparing two independent measurement of the charge, one from the ECAL supercluster and the other from the tracker; thus, the consistency in the measurements of the electron charge is ensured so that events with a wrong electron charge assignment are rejected; this variable is particularly used in the $2lss$ channel to suppress opposite-sign events for which the charge of one of the leptons has been mismeasured. The tight-charge requirement for muons is represented by the requirement of a consistently well measured track transverse momentum given by $\Delta p_T/p_T < 0.2$.          
Leptons are selected for the final analysis if they pass a given threshold of the BDT output, and are referred to as \ti{tight leptons} in the following.          

The validation of the lepton MVA algorithm and the lepton identification variables is performed using data in various control regions; the details about that validation are not discussed here but can be found in Reference \cite{CMS_AN_2017-029}. 

\subsubsection*{Selection definitions}

Electron and muon object identification is defined in three different sets of selections criteria; the \emph{Loose}, \emph{Fakeable Object}, and \emph{Tight} selection. These three levels of selection are designed to serve for event level vetoes, the fake rate estimation application region (see Section \ref{ssec:fake_rate}), and the final signal selection, respectively. The \pt of fakeable objects is defined as $0.85\times\pt(\mathrm{jet})$, where the jet is the one associated to the lepton object. This mitigates the dependence of the fake rate on the momentum of the fakeable object and thereby improves the precision of the method. 

Tables~\ref{tab:muonIDs} and~\ref{tab:eleIDs} list the full criteria for the different selections of muons and electrons.


\begin{table}[!htbp]
\centering
\small
\begin{tabular}{cccc}\hline
Cut                    & Loose      & Fakeable object    & Tight \\
\hline
$|\eta| < 2.4$         & \checkmark & \checkmark         & \checkmark \\
$\pt$                  & $>5 GeV$   & $>15 GeV$          & $>15 GeV$\\
$|d_{xy}| < 0.05$ (cm) & \checkmark & \checkmark         & \checkmark \\
$|d_z| < 0.1$ (cm)     & \checkmark & \checkmark         & \checkmark \\
$\text{SIP}_{3D} < 8$  & \checkmark & \checkmark         & \checkmark \\
\miniIso $< 0.4$       & \checkmark & \checkmark         & \checkmark \\
is Loose Muon          & \checkmark & \checkmark         & \checkmark \\
%\ptRatio              & --         & $>0.3\dagger$ / -- & -- \\
jet CSV                & --         & $< 0.8484$         & $ < 0.8484$ \\
%mva electron ID       & --         & $\ddagger$         & -- \\
is Medium Muon         & --         & --                 & \checkmark \\
tight-charge           & --         & --                 & \checkmark \\
lepMVA $> 0.90$        & --         & --                 & \checkmark \\
\hline
\end{tabular}
\caption[Requirements on each of the three muon selections.]{Requirements on each of the three muon selections. In the cases where the cut values change between the selections, those values are listed in the table. Otherwise, whether the cut is applied is indicated.}
\label{tab:muonIDs}
\end{table}

\begin{table}
\centering
\small
\resizebox{1.0\linewidth}{!}{
\begin{tabular}{cccc}\hline
Cut                                             & Loose      & Fakeable Object              & Tight \\\hline
$|\eta| < 2.5$                                  & \checkmark & \checkmark                   & \checkmark \\
$\pt$                                           & $>7 GeV$   & $>15 GeV$                    & $>15 GeV$      \\
$|d_{xy}| < 0.05$ (cm)                          & \checkmark & \checkmark                   & \checkmark \\
$|d_z| < 0.1$ (cm)                              & \checkmark & \checkmark                   & \checkmark \\
$\text{SIP}_{3D} < 8$                           & \checkmark & \checkmark                   & \checkmark \\
\miniIso $< 0.4$                                & \checkmark & \checkmark                   & \checkmark \\
MVA eID $> (0.0, 0.0, 0.7)$                     & \checkmark & \checkmark                   & \checkmark \\
$\sigma_{i\eta i\eta} <(0.011,0.011,0.030)$     & --         & \checkmark                   & \checkmark \\ %   & for corr. $\pt>30$ & for corr. $\pt>30$ \\
H/E $< (0.10,0.10,0.07)$                        & --         & \checkmark                   & \checkmark \\ %   & for corr. $\pt>30$ & for corr. $\pt>30$ \\
$\Delta\eta_{\textrm in} < (0.01, 0.01, 0.008)$ & --         & \checkmark                   & \checkmark \\ %   & for corr. $\pt>30$ & for corr. $\pt>30$ \\
$\Delta\phi_{\textrm in} < (0.04, 0.04, 0.07)$  & --         & \checkmark                   & \checkmark \\ %   & for corr. $\pt>30$ & for corr. $\pt>30$ \\
$-0.05 < 1/E-1/p < (0.010,0.010,0.005)$         & --         & \checkmark                   & \checkmark \\ %   & for corr. $\pt>30$ & for corr. $\pt>30$ \\
\ptRatio                                        & --         & $>0.5^\dagger$ / --           & -- \\
jet CSV                                         & --         & $< 0.3^\dagger$ / $< 0.8484$ & $ < 0.8484$ \\
tight-charge                                    & --         & --                           & \checkmark \\
conversion rejection                            & --         & --                           & \checkmark \\
Number of missing hits                          & $<2$       & $== 0$                       & $== 0$ \\
lepton MVA $> 0.90$                             & --         & --                           & \checkmark \\\hline
\end{tabular}}
\caption[Criteria for each of the three electron selections.]{Criteria for each of the three electron selections. In cases where the cut values change between selections, those values are listed in the table. Otherwise, whether the cut is applied is indicated. In some cases, the cut values change for different $\eta$ ranges. These ranges are $0 < |\eta| < 0.8$, $0.8 < |\eta| < 1.479$, and $1.479 < |\eta| < 2.5$ and the respective cut values are given in the form (value$_1$, value$_2$, value$_3$). For the two \ptRatio\ and CSV rows, the cuts marked with a $\dagger$ are applied to leptons that fail the lepton MVA cut, while the loose cut value is applied to those that pass the lepton MVA cut.}
\label{tab:eleIDs}
\end{table}

In addition to the previously defined requirements for jets, they are required to be separated from any lepton candidates passing the fakeable object selections by $\Delta\mathrm{R}>0.4$.

\subsection{Lepton selection efficiency}


\begin{figure}[!ht]
\centering
  \includegraphics[width=0.4\linewidth]{lepmva_efficiency/tnp_eff_eb_2lss_pt.pdf}
  \includegraphics[width=0.4\linewidth]{lepmva_efficiency/tnp_eff_ee_2lss_pt.pdf}\\
  \includegraphics[width=0.4\linewidth]{lepmva_efficiency/tnp_eff_mb_2lss_pt.pdf}
  \includegraphics[width=0.4\linewidth]{lepmva_efficiency/tnp_eff_me_2lss_pt.pdf}
\caption[Tight vs loose lepton selection efficiencies in the $2lss$ channel.]{Tight vs loose selection efficiencies for electrons (top), and muons (bottom), for the $2lss$ definition, \ie, including the tight-charge requirement.}
\label{fig:2lss_eff}
\end{figure}

\begin{figure}[!hb]
\centering
  \includegraphics[width=0.4\linewidth]{lepmva_efficiency/tnp_eff_eb_3l_pt.pdf}
  \includegraphics[width=0.4\linewidth]{lepmva_efficiency/tnp_eff_ee_3l_pt.pdf}\\
  \includegraphics[width=0.4\linewidth]{lepmva_efficiency/tnp_eff_mb_3l_pt.pdf}
  \includegraphics[width=0.4\linewidth]{lepmva_efficiency/tnp_eff_me_3l_pt.pdf}
\caption[Tight vs loose lepton selection efficiencies in the $3l$ channel.]{Tight vs loose selection efficiencies for electrons (top), and muons (bottom), for the $3l$ channel not including the tight-charge requirement.}
\label{fig:3l_eff}
\end{figure}

Efficiencies of reconstruction and selecting loose leptons are measured both for muons and electrons using a tag and probe method on both data and MC, using $Z\rightarrow\ell^{+}\ell^{-}$ \cite{tnp}. The scale factors are derived from the ratio of efficiencies $\varepsilon_{i}(p_T, \eta)$ measured for a given lepton in data/MC, according to 
\beqn
\rho(p_T, \eta)= \frac{\varepsilon_{data}(p_T, \eta)}{\varepsilon_{MC}(p_T, \eta)}.
\eeqn

The scale factor for each event is used to correct the weight of the event in the full sample; therefore, the full simulation correction is given by the product of all the individual scale factors. The scale factors used in this thesis are inherited from Reference \cite{CMS_AN_2017-029} which in turns inherited them from leptonic SUSY analyses using equivalent lepton selections.

The efficiency of applying the tight selection as defined in Tables~\ref{tab:muonIDs} and~\ref{tab:eleIDs}, on the loose leptons are determined by using a tag and probe method on a sample of Drell-Yan enriched events. Figures \ref{fig:2lss_eff} and \ref{fig:3l_eff} show the efficiencies for the $2lss$ channel and $3l$ channel respectively. Efficiencies in the $2lss$ channel have been produced including the tight-charge requirement, while for the $3l$ channel it is not included. Number of passed and failed probes are determined from a fit to the invariant mass of the dilepton system. Simulation is corrected using these scale factors; note that they depends on \etac and \pt.   

\subsection{Jets and \bjet tagging}

In this analysis, jets are reconstructed by clustering PF candidates using the anti-$k_t$ algorithm with parameter distance $\Delta R=0.4$; those charged hadrons that are not consistent with the selected primary vertex are discarded from the clustering. The jet energy is then corrected for the varying response of the detector as a function of transverse momentum \pt and pseudorapidity \etac. Jets are selected for use in the analysis only if they have $\pt > 25$ GeV and are separated from any selected leptons by $\Delta R > 0.4$.

Jets coming from the primary vertex and jets coming from pile-up vertices are distinguished using a MVA discriminator based on the differences in the jet shapes, in the relative multiplicity of charged and neutral components, and in the different fraction of transverse momentum which is carried by the hardest components. Jet tracks are also required to be compatible with the primary
vertex.

Jets originated from the hadronization of a $b$ quark are selected using a MVA likelihood discriminant which uses track-based lifetime information and reconstructed secondary vertices (CSV algorithm). Only jets within the CMS tracker acceptance ($\eta < 2.4$) are identified with this tool. Data samples are used to measure the efficiency of the \bjet tagging and the probability to misidentify jets from light quarks or gluons; in both cases the measurements are parametrized as a function of the jet \pt and \etac and later used to correct differences between the data and MC simulation in the $b$ tagging performance, by applying per-jet weights to the simulation, dependent on the jet \pt, \etac, $b$ tagging discriminator, and flavor (from simulation truth)\cite{btag_corr}. The per-event weight is taken as the product of the per-jet weights, including those of the jets associated to the leptons. The weights are derived on \ttbar and Z+jets events.

Two working points are defined, based on the CSV algorithm output: \ti{loose'} working point (CSV>0.46) with a $b$ signal tagging efficiency of about 83\% and a mistagging rate of about 8\%; and \ti{medium} working point (CSV>0.80) with $b-$tagging efficiency of about 69\% and mistagging rate of order 1\% \cite{btag_points}. Tagging of jets from charm quarks have efficiencies of about 40\%  and 18\% for loose and medium working points respectively. Separate scale factors are applied to jets originating from bottom/charm quarks and from light quarks in simulated events to match the tagging efficiencies measured in the data.


% FIXME Something about forward jets?

\subsection{Missing Energy MET}

As stated in Section \ref{sssec:met}, the MET vector is calculated as the negative of the vector sum of transverse momenta of all PF candidates in the event and its magnitude is referred to as $E_T^{miss}$. Due to pile-up interactions, the performance in determining MET is degraded; in order to correct for that, the energy from the selected jets and leptons that compose the event is assigned to the variable $H_T^{miss}$. It is calculated in the same way as  $E_T^{miss}$ and although it has worse resolution than $E_T^{miss}$, it is more robust in the sense that it does not rely on the soft part of the event. The event selection uses a linear discriminator based on the two variables given by 

\beqn
E^{miss}_{T\quad LD} = 0.00397*E_T^{miss} + 0.00265*H_T^{miss}
\eeqn

\noindent taking advantage of the fact that the correlation between $E_T^{miss}$ and  $H_T^{miss}$ is less for events with instrumental missing energy than for events with real missing energy. The working point $E^{miss}_{T\quad LD} > 0.2$ was chosen to ensure a good signal efficiency while keeping a good background rejection.

\section{Event selection}

Events are selected considering the features of the signal process and the decay signature as described in Section \ref{sec:thq_sign}. At the trigger level, events are selected to contain either one, two, or three leptons with minimal \pt thresholds:
\begin{itemize}
\item single-lepton trigger $\to$ 24 GeV for muons and at 27 GeV for electrons
\item double-lepton triggers $\to$ leading and sub-leading leptons: 17 and 8 GeV for muons and 23 and 12 GeV for electrons.
\item three-lepton triggers $\to$ threshold on the third hardest lepton in the event: 5 and 9 GeV for muons and electrons, respectively.
\end{itemize}

The offline event selection level targets the specific topology of the \tHq signal with $H\to WW$ and $t \to Wb \to l\nu b$; therefore, the resulting state is composed of three W bosons, one $b$ quark, and a light spectator quark at high rapidity. The selection criteria for the two channels exploited in this analysis are summarized in Table \ref{tab:cuts}. This selection includes contributions from $H \to \tau\tau$ and $H\to ZZ$ as well.

\begin{table}[!h]
\centering
\small
\begin{tabular}{p{6.5cm}l} \hline
\textbf{Same-sign $\ell\ell$ channel }           & \textbf{$\ell\ell\ell$ channel}              \\\hline
\multicolumn{2}{c}{have fired one of the corresponding trigger paths}\\
\multicolumn{2}{c}{No loose leptons with $m_{\ell\ell} < 12$GeV} \\
\multicolumn{2}{c}{One or more $b$ tagged jets (CSV medium) |\etac|<2.4} \\
\multicolumn{2}{c}{One or more non-tagged jets: central $\to$ \pt>25 GeV, \etac<2.4} \\
\multicolumn{2}{c}{\textcolor{white}{One or more non-tagged jets:} forward $\to$ \pt>40 GeV, \etac>2.4} \\
\multicolumn{2}{c}{$E^{miss}_{T\quad LD} > 0.2$} \\\hline
Exactly two tight same-sign leptons              & Exactly three tight leptons \\
Lepton $\pt>25/15$GeV                            & Lepton $\pt>25/15/15$GeV               \\
Electrons are triple-charge consistent.          & No OSSF lepton pair with $|m_{\ell\ell}-m_Z|<15$GeV \\
Muon \pt resolution: $\Delta p_T/p_T < 0.2$.     &                                                      \\
No ee pair with $|m_{ee}-m_Z|<10$GeV&                                                    \\\hline
\end{tabular}
\caption{Summary of event pre-selection.}\label{tab:cuts}
\end{table}

In the $2lss$ channel, events with additional tight leptons are vetoed as well as those for which a loose lepton pair has an invariant mass below 12 GeV. A threshold in \pt of the leading and sub-leading leptons is also required. Events where the two electrons have invariant mass within 10 GeV of the Z boson mass (\ti{Z-veto}) are discarded in order to reject events from DY+jets production with charge misidentified electrons. In addition, contribution from the associated production of two W bosons of equal charge and two light jets $W^\pm W^\pm qq$ and from same-sign W boson pairs can also be produced in double parton scattering (DPS) processes, where each of the colliding protons gives two partons, resulting in two hard interactions.

In the $3l$ lepton channel, leptons are required to have respectively $\pt > 25 \textrm{GeV, }> 15\textrm{ GeV, and }> 15\textrm{ GeV}$. Events with an opposite-sign same-flavor lepton combination (OSSF) with invariant mass within 15 GeV of the Z boson mass are discarded in order to reject events from \WZ +jets production.

%______________________ Background predictions ______________________
\section{Background modeling and predictions}\label{sec:bg}

The dominant background contribution is expected to arise from top quark production processes, either \ttbar pair production or in \ttbar associated production with a W/Z. Processes with production of single top quarks also contribute, mainly in the associated production with a Z boson (\tZq) or when produced with both a W and a Z boson (\tZW). Background contamination from diboson processes is strongly suppressed by imposing the Z-veto, vetoing additional leptons and requiring $b$-jets in the event.

The selection criteria in Table \ref{tab:cuts} represent a relatively loose selection that allows to maintain a large signal efficiency while suppressing the main backgrounds; thus that selection is called \ti{pre-selection}. The events obtained from the pre-selection are then used to extract the signal contribution in a second analysis step, using BDT discriminators against the main backgrounds of \ttW/\ttZ\ and non-prompt leptons from \ttbar. The shape of the discriminator variables is then fit to the observed data distribution to estimate the signal and background yields, simultaneously for all channels.

Irreducible backgrounds are reliantly estimated from MC simulated events; therefore, in this analysis all backgrounds involving prompt leptons are estimated in this way. Reducible backgrounds, like non-prompt lepton backgrounds, are not well predicted by simulation, hence, they are estimated using data-driven methods.

\subsection{\ttV\ and diboson backgrounds}

Backgrounds from \ttW and \ttZ\ processes are estimated using simulated events, corrected for data/MC differences and inefficiencies (trigger and lepton selection) in the same way as signal events. Their production cross sections are calculated at NLO order of QCD and EWK, considering theoretical uncertainties from unknown higher orders of 12\% for \ttW and 10\% for \ttZ. Additional uncertainties arise from the knowledge of PDFs and $\alpha_s$ of about 4\% each for \ttW and \ttZ.

The diboson contribution is also estimated from simulated events; however, the overall normalization of this process is obtained from a dedicated control region. The motivation behind that strategy is that even thought the measured inclusive cross section for diboson processes (\WZ,\ZZ) is in good agreement with the NLO calculations \cite{CMS_AN_2017-029}, that agreement is perturbed when leptonic Z decays and hadronic jets in the final state are required; those requirements are precisely the ones that make the diboson production a background for the \tHq signal. Thus, by using a dedicated control region dominated by \WZ production\footnote{\ZZ background is strongly reduced by the cut on MET.}, the overall normalization is constrained.

The control region is defined by the presence of at least three leptons, of which one opposite-sign pair must be compatible with a Z boson decay, \ie, invert the Z-veto which makes the control region orthogonal to signal region; the b-jet tagging requirements is also inverted with respect to the signal region, \ie, require two not $b$-jets. A scale factor is extracted from the predicted distribution of \WZ events in the control region, and the observed data, while keeping other processes fixed; this factor is used to scale the diboson prediction in the signal selection region. More details about the procedure used can be found in Reference \cite{CMS_AN_2017-029} from where the scale factor is taken.

\begin{figure} [!h]
\centering
        \includegraphics[width=0.32\textwidth]{controlplots/3l-Z/Lep1Pt.pdf}
        \includegraphics[width=0.32\textwidth]{controlplots/3l-Z/Lep2Pt.pdf}
        \includegraphics[width=0.32\textwidth]{controlplots/3l-Z/Lep3Pt.pdf} \\
        \includegraphics[width=0.32\textwidth]{controlplots/3l-Z/mZ1.pdf}
        \includegraphics[width=0.32\textwidth]{controlplots/3l-Z/nBJetLoose25.pdf}
        \includegraphics[width=0.32\textwidth]{controlplots/3l-Z/nJet25.pdf}
\caption{Kinematic distributions in the diboson control region.}
\label{fig:3lzcontrol}
\end{figure}
              
In order to test the usability of the diboson background scale factor in this analysis, a Z-enriched control region\footnote{This control region is different to the one used to find the scale factor.} was defined by inverting the Z-veto and requiring exactly three tight leptons with \pt > 25/15/15 GeV, one or more jets passing the CSVv2 loose working point and less than four central jets. Figure \ref{fig:3lzcontrol} shows the distribution of three variables in the diboson control region; the good agreement between MC and data motivates the adoption of the diboson background scale factor.

Most of the diboson events passing the signal selection contain jets from light quarks and gluons that are incorrectly tagged as $b$-jets; it makes the estimate mainly sensitive to the experimental uncertainty in the mis-tag rate rather than the theoretical uncertainty in the jet flavor composition. The overall uncertainty assigned to the diboson prediction is estimated from the statistical uncertainty due to the limited sample size in the control region (30\%), the residual background in the control region (20\%), the uncertainties on the $b$-tagging rate (10-40\%), and from the knowledge of PDFs and the theoretical uncertainties of the extrapolation (up to 10\%).

\subsection{Non-prompt and charge mis-ID backgrounds}\label{ssec:fake_rate}

The non-prompt lepton background contribution to the final selection is estimated using the fake factor method. The main idea of the method is to define a control region of events enriched in the background to estimate and determine a factor that relates (extrapolates) these events to those in the signal region. The method is data-driven in the sense that the control sample is selected from data, and the extrapolation factor is measured from data.

In the signal region of this analysis, non-prompt leptons are predominantly produced in \ttbar events, with a much smaller contribution, from Drell-Yan production; therefore, the control region also know as \ti{application region}, is defined by modifying the event selection criteria in such a way that most of the events after selection are \ttbar events and thus the misidentification rate is increased. The application regions for electrons and muons are defined by the fakeable object definitions in Tables ~\ref{tab:muonIDs} and~\ref{tab:eleIDs}. Since the fakeable definition is a loosened version of the tight definition, in the context of fake rates the fakeable definition it becomes the loose selection. 

The ratio between the number of events that pass both, the loose and tight selections, and the number of events that pass the loose selection but fail the tight one, corresponds to the \ti{fake factor/fake rate (f)}. The measurement of the fake factor is made using two background dominated data samples, collected with dedicated triggers, %(subtracting the residual prompt lepton contribution using MC)
as a function of \pt and |\etac| and separately for muons and electrons:

\begin{itemize}
\item A sample dominated by QCD multijet events, collected using single lepton triggers at relatively high \pt thresholds. It is used to extract ratios for lepton candidates with \pt above 30 GeV.
\item A sample dominated by Z + jets events, where the two high \pt leptons resulting from the Z decay are used to trigger the events without biasing the \pt spectrum of a third lepton at low transverse momentum. It is used to determine the ratios for low \pt leptons. 
\end{itemize}

Processes like $W$ + jets, $Z$ + jets , \WZ and \ZZ produce prompt leptons that contaminate the samples; thus, they are suppressed by vetoing additional leptons in the selection, and the residual contamination is then subtracted using the transverse mass as a discriminating variable.

The extrapolation from the application region to the signal region is performed by weighting the events in the application region using the fake factor according to the following rules:

\begin{itemize}
\item events with one lepton failing the tight criteria are weighted with the factor $\frac{f}{(1-f)}$ for the estimate to the signal region. 
\item events with two leptons (i,j) failing the tight criteria are weighted with the factor $-\frac{f_if_j}{(1-f_i)(1-f_j)}$ for the estimate to the signal region. 
\item events with three leptons (i,j,k) failing the tight criteria are weighted with the factor $\frac{f_if_jf_k}{(1-f_i)(1-f_j)(1-f_k)}$ for the estimate to the signal region. 
\end{itemize}

\begin{figure}[htb]
\centering
        \includegraphics[width=0.33\textwidth]{fr_mu_barrel}
        \includegraphics[width=0.33\textwidth]{fr_el_barrel} \\
        \includegraphics[width=0.33\textwidth]{fr_mu_endcap}
        \includegraphics[width=0.33\textwidth]{fr_el_endcap}
\caption[Fake rates]{Fake rate measurement in events in data for muons (left column) and electrons (right column). Predictions from simulated events in the measurement region (blue) and from non-prompt leptons in \ttbar (red) are included for comparison. Top row is for |\etac|<2.5 and bottom row for |\etac|>2.5.}
\label{fig:frmeas-comb-data}
\end{figure}

Figure \ref{fig:frmeas-comb-data} shows the fake rates for electrons and muons used in this analysis which were taken from the studies in Reference \cite{CMS_AN_2017-029}.

The resulting prediction of the event yield in the signal selection carries an uncertainty of 30-50\% which is composed of the statistical uncertainty in the measurement of the fake rates due to prescaling of lepton triggers, the uncertainty in the subtraction of residual prompt leptons from the control region, and from testing the closure of the method in simulated background events; hence, it is one of the dominant limitations on the performance of multilepton analyses in general and this analysis in particular.

Finally, an additional source of background arises in the $2lss$ channel from events with an originally opposite-sign lepton pair for which the charge of one of the leptons is misidentified (\ti{charge mis-ID}); usually this happens because of the convertion of hard bremsstrahlung photons emitted from the initial lepton, hence, it is more likely to happen for electrons than for muons.

The charge mis-ID background is estimated from the yield of opposite-sign event in the signal region by measuring the charge mis-ID probability in same-sign and opposite-sign events compatible with a Z boson decay, in several bins of \pt\ and \etac, and weighting events with opposite-sign leptons in the signal selection.

\begin{table}[htp]
\centering
\begin{tabular}{lccc}\hline
 Data                & $10\leq\pt<25$ GeV  & $25\leq\pt<50$ GeV  & 50 GeV$\leq\pt$ \\\hline
$0\leq\eta<1.48$     & 0.0442 $\pm$ 0.0011 & 0.0179 $\pm$ 0.0004 & 0.0262 $\pm$ 0.0020 \\
$1.48\leq\eta<2.5$   & 0.1329 $\pm$ 0.0066 & 0.1898 $\pm$ 0.0014 & 0.3067 $\pm$ 0.0113 \\\hline
 MC                  &                     &                     &                      \\\hline
$0\leq\eta<1.48$     & 0.0378 $\pm$ 0.0016 & 0.0222 $\pm$ 0.0003 & 0.0233 $\pm$ 0.0015 \\
$1.48\leq\eta<2.5$   & 0.0956 $\pm$ 0.0044 & 0.2108 $\pm$ 0.0027 & 0.3157 $\pm$ 0.0018 \\\hline
\end{tabular}
\caption[Electron charge mis-ID probabilities.]{Electron charge mis-ID probabilities (in percent), determined in data (top) and Drell-Yan MC (bottom)\cite{CMS_AN_2017-029}.}
\label{tab:chmisid_prob}
\end{table}

The charge mis-ID probability is found to be negligible for this analysis for muons, whereas for
electrons it ranges from about 0.02\% in the barrel section (|\etac| < 1.48) up to about 0.35\% in the detector endcaps (1.48 < |\etac| < 2.5). as shown in Table \ref{tab:chmisid_prob} and Figure \ref{fig:chmisid_prob}.

\begin{figure}[htp]
\centering
\includegraphics[width=0.49\textwidth]{chmid_prob_barrel.pdf}
\includegraphics[width=0.49\textwidth]{chmid_prob_endcap.pdf}
\caption[Elecron mis-ID probabilities.]{Electron charge mis-ID probabilities as a function of \pt\ for |\etac|<2.5 (left) and |\etac|<2.5 (right) \cite{CMS_AN_2017-029}.}
\label{fig:chmisid_prob}
\end{figure}                            

The contribution from charge mis-ID electrons in signal selection of this analysis comes mainly from \ttbar\ and Drell-Yan events. The systematic uncertainty of the normalization of the charge mis-id.\ estimate is evaluated at about 30\%, arising from a slight disagreement of the mis-ID.\ probability between data and simulation. Given that it only affects the $e\mu$ channel, it impact on the final sensitivity is very limited.

\section{Pre-selection yields}

The expected and observed event yields of the pre-selection are shown in Table \ref{tab:yields-sel}. For the \tH\ and \ttH processes, the largest contribution comes from Higgs decays to WW (about 75\%), followed
by \tautau (about 20\%) and ZZ (about 5\%). Other Higgs production modes contribute negligible event yields (< 5\% of the \tH +\ttH yield) as shown in Table ~\ref{tab:yield_hbr}.

\begin{table}[thb]
\centering
\begin{tabular}{lrrrr}\hline
\multicolumn{1}{c@{\qquad}}{} & \multicolumn{1}{c@{\qquad}}{$3\ell$} & \multicolumn{1}{c}{$\mumu$} & \multicolumn{1}{c@{\qquad}}{$\emu$} & \multicolumn{1}{c}{$\ee$} \\ \hline
$\ttW$                        & $  22.50 \pm 0.35$ & $ 68.03 \pm 0.61 $ & $ 97.00 \pm 0.71 $ & $ 29.63 \pm  0.39 $ \\
$\ttZ\!/\!\gamma^*$           & $  32.80 \pm 1.79$ & $ 25.89 \pm 1.12 $ & $ 64.82 \pm 2.42 $ & $ 28.74 \pm  1.70 $ \\
$\WZ$                         & $   8.22 \pm 0.86$ & $ 15.07 \pm 1.19 $ & $ 26.25 \pm 1.57 $ & $  9.31 \pm  0.93 $ \\
$\ZZ$                         & $   1.62 \pm 0.33$ & $  1.16 \pm 0.29 $ & $  2.86 \pm 0.45 $ & $  1.09 \pm  0.27 $ \\
$W^\pm W^\pm qq$              & --                 & $  3.96 \pm 0.52 $ & $  6.99 \pm 0.69 $ & $  2.19 \pm  0.37 $ \\
$W^\pm W^\pm \text{(DPS)}$    & --                 & $  2.48 \pm 0.42 $ & $  4.17 \pm 0.54 $ & $  0.81 \pm  0.24 $ \\
VVV                           & $   0.42 \pm 0.16$ & $  2.99 \pm 0.34 $ & $  4.85 \pm 0.43 $ & $  1.19 \pm  0.21 $ \\
$\mathrm{tttt}$               & $   1.84 \pm 0.44$ & $  2.32 \pm 0.45 $ & $  4.06 \pm 0.57 $ & $  0.89 \pm  0.31 $ \\
$\mathrm{tZq}$                & $   3.92 \pm 1.48$ & $  5.77 \pm 2.24 $ & $ 10.73 \pm 3.03 $ & $  7.56 \pm  1.72 $ \\
$\mathrm{tZW}$                & $   1.70 \pm 0.12$ & $  2.13 \pm 0.13 $ & $  3.91 \pm 0.18 $ & $  1.13 \pm  0.10 $ \\
$\gamma$ conversions          & $   7.43 \pm 1.94$ & --                 & $ 23.81 \pm 6.04 $ & $  9.87 \pm  4.17 $ \\ \hline
Non-prompt                    & $  25.61 \pm 1.26$ & $ 80.94 \pm 2.02 $ & $135.34 \pm 2.83 $ & $ 47.72 \pm  1.79 $ \\
Charge mis-ID                 & --                 & --                 & $ 58.50 \pm 0.31 $ & $ 44.52 \pm  0.31 $ \\ \hline
All backgrounds               & $ 106.05 \pm 3.45$ & $210.74 \pm 3.61 $ & $443.30 \pm 8.01 $ & $184.65 \pm  5.29 $ \\ \hline
$\tHq$ ($\Ct=-1.0$)           & $   7.48 \pm 0.14$ & $ 18.48 \pm 0.22 $ & $ 27.41 \pm 0.27 $ & $  8.47 \pm  0.15 $ \\
$\tHW$ ($\CV=-1.0$)           & $   7.38 \pm 0.16$ & $  7.72 \pm 0.17 $ & $ 11.23 \pm 0.20 $ & $  3.66 \pm  0.11 $ \\
$\ttH$                        & $  18.29 \pm 0.41$ & $ 24.18 \pm 0.48 $ & $ 35.21 \pm 0.58 $ & $ 11.07 \pm  0.32 $ \\ \hline
Data (35.9\fbinv)             & \multicolumn{1}{l}{127}&\multicolumn{1}{l}{280} & \multicolumn{1}{l}{525} & \multicolumn{1}{l}{208}\\\hline
\end{tabular}
\caption{Expected and observed yields for $35.9\fbinv$ after the pre-selection in all final states. Uncertainties are statistical only.}
\label{tab:yields-sel}
\end{table}

\begin{table}[hbt]
\centering
\begin{tabular}{lrrrr}\hline
\multicolumn{1}{c@{\qquad}}{} & \multicolumn{2}{c@{\qquad}}{$3\ell$} & \multicolumn{2}{c}{$\mumu$} \\ \hline
$\tHq (\mathrm{Inclusive})$   & $\mathbf{6.57}$     & 100.0\%  & $\mathbf{17.38}$ & 100.0\% \\
$\tHq (H\to WW)$              & $4.84$   & 73.9\%   & $13.33 $ &  76.9\%  \\
$\tHq (H\to \tau\tau)$        & $1.04$   & 15.9\%   & $ 3.62 $ &  20.6\%  \\
$\tHq (H\to ZZ)$              & $0.48$   &  7.2\%   & $ 0.37 $ &   2.2\%  \\
$\tHq (H\to \mu\mu)$          & $0.21$   &  3.0\%   & $ 0.04 $ &   0.2\%  \\
$\tHq (H\to \gamma\gamma)$    & $<0.01$  &  0.1\%   & $ 0.02 $ &   0.1\%  \\
$\tHq (H\to bb)$              & $<0.01$  & $<0.1$\% & $ 0.01 $ & $<0.1$\% \\ \hline
$\tHW (\mathrm{Inclusive})$   & $\mathbf{7.32}$ & 100.0\% & $\mathbf{7.62}$ & 100.0\% \\
$\tHW (H\to WW)$              & $5.50 $ &  76.9\%  & $ 5.60$ & 74.1\% \\
$\tHW (H\to \tau\tau)$        & $1.40 $ &  20.6\%  & $ 1.81$ & 23.1\% \\
$\tHW (H\to ZZ)$              & $0.31 $ &   2.2\%  & $ 0.21$ &  2.7\% \\
$\tHW (H\to \mu\mu)$          & $0.12 $ &   0.2\%  & $ 0.01$ &  0.1\% \\
$\tHW (H\to \gamma\gamma)$    & $<0.01$ & $<0.1$\% & $<0.01$ &$<0.1$\% \\
$\tHW (H\to bb)$              & $<0.01$ & $<0.1$\% & $<0.01$ &$<0.1$\% \\ \hline
\end{tabular}
\caption[Signal yields split by decay channels of the Higgs boson.]{Signal yields split by decay channels of the Higgs boson. Forward jet \pt cut at 25 GeV.}
\label{tab:yield_hbr}
\end{table}

Figure ~\ref{fig:input_vars_presel} shows the distributions of some relevant kinematic variables, normalized to the cross section of the respective processes and to the integrated luminosity. The remaining variables distributions are shown in Appendix \ref{app:presel_plots} 

\begin{figure}[!htb]
\centering
        \includegraphics[width=0.31\linewidth]{polished/dPhiHighestPtSSPair_mm.pdf}
        \includegraphics[width=0.31\linewidth]{polished/maxEtaJet25_40_mm.pdf}
        \includegraphics[width=0.31\linewidth]{polished/nJet25_mm.pdf} \\
        \includegraphics[width=0.31\linewidth]{polished/dPhiHighestPtSSPair_em.pdf}
        \includegraphics[width=0.31\linewidth]{polished/maxEtaJet25_40_em.pdf}
        \includegraphics[width=0.31\linewidth]{polished/nJet25_em.pdf} \\
        \includegraphics[width=0.31\linewidth]{polished/dPhiHighestPtSSPair_3l.pdf}
        \includegraphics[width=0.31\linewidth]{polished/maxEtaJet25_40_3l.pdf}
        \includegraphics[width=0.31\linewidth]{polished/nJet25_3l.pdf} 
\caption{Distributions of discriminating variables for the event pre-selection for the same-sign \mumu\ channel (top row), the same-sign \emu\ channel (middle row) and three lepton channel (bottom row), normalized to 35.9\fbinv, before fitting the signal discriminant to the observed data. Uncertainties are statistical and unconstrained (pre-fit) normalization systematics. The shape of the two \tH\ signals for $\Ct=-1.0$ is shown, normalized to their respective cross sections for $\Ct=-1.0, \CV=1.0$.}
\label{fig:input_vars_presel}
\end{figure}

A significant fraction of selected data events (about 50\% in the dilepton channels, and about 80\% in the trilepton channel) also passes the selection used in the dedicated search for ttH in multilepton channels \cite{CMS_AN_2017-029}. This is particularly important when considering a possible combination of the measurements from both studies. More details about the overlap between this both analyses are presented in Appendix \ref{app:overlap}.   

%______________________ Signal discrimination ______________________
\section{Signal discrimination }
\label{secc:signal_disc}

The production cross section for the signal processes \tHq, \tHW, and \ttH is only about 600 fb (the enhancement provided by inverted couplings, \Ct = -1 almost double it), resulting in a small signal to background ratio even for a tight selection. A multivariate method is hence employed to train discriminators to separate \tH\ signal events from the dominant background events.

\subsection{MVA classifiers evaluation}

Several MVA classifier algorithms were evaluated in order to determine the most appropriate method for this analysis\footnote{The choice of the tested algorithms was based on the recommendations provided by the official TMVA user guide, the experience from previous analyses and considering the expertise of the members of the \tHq and \ttH analyses groups. Only the BDT classifier is described in this thesis and a detailed description of all available methods can be found in Reference \cite{tmva}}. The comparison is based on the performance of the classifiers, encoded in the plot of the background rejection as a function of the signal efficiency (ROC curve). The top row of Figure ~\ref{roc} shows the ROC curves for the several methods evaluated; two separated training were performed in the $3l$ channel: against \ttbar\ (right) and \ttV\ (left) processes.

\begin{figure} [!h]
  \centering
   \includegraphics[width=0.49\textwidth]{roc_ttv_3l_multimva.pdf}
   \includegraphics[width=0.49\textwidth]{roc_tt_3l_multimva.pdf} \\
   \includegraphics[width=0.49\textwidth]{roc_ttv_2lss.pdf}
   \includegraphics[width=0.49\textwidth]{roc_tt_2lss.pdf} 

\caption[MVA classifiers performance.]{ Top: Background rejection vs signal efficiency (ROC curves) for various MVA classifiers in the $3l$ channel for training against \ttV\ (left) and \ttbar\ (right). Bottom: background rejection vs signal efficiency (ROC curve) in the $2lss$ channel for a single discriminator: BDTG, against \ttV\ (left) and \ttbar\ (right).}
\label{roc}
\end{figure} 

In both cases, the gradient boosted decision tree \ti{BDTG} (BDTA\_GRAD in the plot) classifier offers the best results, followed by the adaptive BDT classifier (\ti{BDTA}); the several Fisher classifiers tested, which differ in their parameters and/or boosting method, they offer similar performance among them, while the k-Nearest Neighbour (kNN) classifier performance is below the rest of the classifiers. The corresponding ROC curves and in the $2lss$ channel for trainings against \ttV (left) and \ttbar (right) processes are shown in the bottom row of Figure ~\ref{roc}; the BDTG performance is similar to that in the $3l$ channel.

\subsection{Discriminating variables}

The classifier chosen to separate the \tHq signal from the main backgrounds is the BDTG classifier, trained on simulated signal and background events. The samples used in the training are the \tHq sample in Table \ref{tab:sigsamples}, the samples in the third section of table \ref{tab:bgsamples} and the samples marked with an * in the same table.

As explained in Section \ref{subsec:dt}, a set of discriminating variables are given as input to the BDTG which combines the individual discrimination power of each input variable to produce a discriminator with the maximum discrimination power. Table~\ref{tab:bdtinputs} lists the input variables used in the BDTG trainings for this analysis. The same set of input variables was used to produce the plots for MVA classifiers evaluation. 

\begin{table}[h!]
\centering
\begin{tabular}{lp{10cm}}\hline
Variable name        & Description\\ \hline
nJet25               & Number of jets with $\pt>25$ GeV, $|\eta|<2.4$\\
nJetEta1             & Number of jets with $|\eta|>1.0$, non-CSV-loose\\\hline
MaxEtaJet25          & Max. $|\eta|$ of any (non-CSV-loose) jet with $\pt>25$ GeV\\
detaFwdJetClosestLep & $\Delta \eta$ forward light jet and closest lepton\\
detaFwdJetBJet       & $\Delta \eta$ forward light jet and hardest CSV loose jet\\
detaFwdJet2BJet      & $\Delta \eta$ forward light jet and second hardest CSV loose jet \\\hline
Lep3Pt/Lep2Pt        & \pt\ of the $3^{rd}$ lepton ($2^{nd}$ for ss2l)\\
totCharge            & Sum of lepton charges \\
minDRll              & Min $\Delta R$ any two leptons\\
dphiHighestPtSSPair  & $\Delta \phi$ of highest \pt\ same-sign lepton pair\\\hline
\end{tabular}
\caption[BDTG input variables.]{BDTG input variables. First section lists variables related to jet multiplicities; second section lists variables related to forward jet activity, and third section lists variables related to lepton kinematics.}
\label{tab:bdtinputs}
\end{table}

Plots in Figure ~\ref{fig:input_vars_3l} shows the BDTG input variables distributions for the signal and background samples, in the $3l$ channels.

\begin{figure} [!h]
 \centering
 \includegraphics[width=0.32\textwidth]{Lep3Pt.pdf} 
 \includegraphics[width=0.32\textwidth]{dEtaFwdJetBJet.pdf}
 \includegraphics[width=0.32\textwidth]{dEtaFwdJet2BJet.pdf}\\
 \includegraphics[width=0.32\textwidth]{dEtaFwdJetClosestLep.pdf}
 \includegraphics[width=0.32\textwidth]{dPhiHighestPtSSPair.pdf}
 \includegraphics[width=0.32\textwidth]{maxEtaJet25.pdf}\\
 \includegraphics[width=0.32\textwidth]{minDRll.pdf}
 \includegraphics[width=0.32\textwidth]{nJet25.pdf} 
 \includegraphics[width=0.32\textwidth]{nJetEta1.pdf}\\
 \includegraphics[width=0.32\textwidth]{totCharge.pdf}
\caption[BDTG classifier Input variables distributions.]{Distributions of the BDTG classifier input variables (not normalized) for signal discrimination in the $3l$ channel.} 
\label{fig:input_vars_3l}
\end{figure}    

All the input variables have some discrimination power, however, that power is bigger for some of them; for instance, the third lepton \pt plot (top left in Figure ~\ref{fig:input_vars_3l}) shows some discrimination power against WZ and VVV backgrounds for which there is a peak around 30 GeV while \tHq peak around 18 GeV; although the discrimination power does not cover all the backgrounds, it counts for the final discriminator. A similar situation can be seen in the plot for the number of jets (row three, column two); \ttW, \ttZ and \ttH processes tend to have more jets compared to the \tHq process. The discrimination power is more evident in other plots like in the plot of the maximum $|\eta|$ of the jets in the event (row two, column three). The same or equivalent input variables are found to be performing well for both $3l$ and $2lss$ channels. Figure ~\ref{fig:input_vars_2lss} shows the corresponding input variables distribution plots for the $2lss$ channel.

\subsubsection*{Discrimination power from BDTG classifier}

%% \begin{figure} [!h]
%%   \centering
%%   \includegraphics[width=\textwidth]{mva_input1_tt.pdf}
%%   \includegraphics[width=\textwidth]{mva_input2_tt.pdf}
%% \caption[BDT input variables. Discrimination against \ttbar in $3l$ channel.]{BDT input variables as seen by BDTG classifier for the $3l$ channel, \tHq signal (blue) discriminated against \ttbar\ background (red).} 
%% \label{mva_input_tt}
%% \end{figure}

The Discrimination power of the input variables can also be evaluated from the BDTG training, exclusively for the training samples, \ie, dominant backgrounds (\ttbar and \ttV); the training samples are submitted to the selection cuts on Table ~\ref{tab:cuts}.

\begin{figure} [!ht]
  \centering
  \includegraphics[width=\textwidth]{mva_input1.pdf}
  \caption[BDT input variables. Discrimination against \ttbar and \ttV\ in $3l$ channel.]{BDT input variables as seen by BDTG classifier for the $3l$ channel, \tHq signal(blue) discriminated against \ttV\ background (red).}
\label{fig:mva_input_comp}
\end{figure}

Figure \ref{fig:mva_input_comp} shows the comparison between input variables for the two trainings in the $3l$ channel; it reveals that some variables show opposite behavior for the two background sources, which results in potentially screening the discrimination power if they were to be used in a single discriminant, \ie, if the training would join \ttbar and \ttV. For some other variables the distributions are similar in both background cases. In contrast to the distributions in Figure ~\ref{fig:input_vars_3l} only the dominant backgrounds are included; however, the discrimination power agrees among plots.

Figures in the Appendix ~\ref{mva_input_2lss_tt}, ~\ref{mva_input_2lss_ttv}, ~\ref{mva_input_tt}, and ~\ref{mva_input_ttv} show the input variables distributions for the $2lss$ and $3l$ channel as seen by the BDTG classifier. 

\subsubsection*{Input variables correlations}

\begin{figure} [!ht]
  \centering
      \includegraphics[width=0.32\textwidth]{sig_corr_tt_2lss.pdf}
      \includegraphics[width=0.32\textwidth]{bkg_corr_tt_2lss.pdf}
      \includegraphics[width=0.32\textwidth]{bkg_corr_ttv_2lss.pdf}\\
      \includegraphics[width=0.32\textwidth]{corr_signal.pdf}
      \includegraphics[width=0.32\textwidth]{corr_tt.pdf}
      \includegraphics[width=0.32\textwidth]{corr_ttv.pdf}
\caption[Correlation matrices for the BDT input variables.]{ Signal (left), \ttbar\ background (middle), and \ttV\ background (right.) correlation matrices for the input variables in the BDTG classifier for the $2lss$ (top) and  $3l$ (bottom) channels.}
\label{mva_corr}
\end{figure}

From Table~\ref{tab:bdtinputs}, it is clear that the input variables are correlated to some extend. These correlations play an important role for some MVA methods like the Fisher discriminant method in which the first step consist of performing a linear transformation to an phase space where the correlations between variables are removed. In the case of BDT, correlations do not affect the performance. Figure ~\ref{mva_corr} shows the linear correlation coefficients for signal and background for the two training cases (the signal values are identical by construction). As expected, strong correlations appears for variables related to the forward jet activity.

\subsection{BDTG classifiers response}

After the training stage, the BDTG classifier is tested to ensure its ability to discriminate between simulated signal and background events. The BDTG classifier output distributions for signal and backgrounds in the $3l$ channel are shown in Figure ~\ref{fig:bdtg_output_default}. As expected, a good discrimination power is obtained using default discriminator parameter values; some overtraining is also visible.

\begin{figure} [!h]
  \centering
   \includegraphics[width=0.49\textwidth]{bdta_grad_output_3l_ttv}
   \includegraphics[width=0.49\textwidth]{bdta_grad_output_3l_tt}
\caption[BDTG classifier response. Default parameters.]{BDTG classifier output for trainings against \ttV (left) and \ttbar(right). Default BDTG  parameters have been used.}
\label{fig:bdtg_output_default}
\end{figure}

In order to explore further optimization in the BDTG performance, several changes from the default BDTG parameters were tested; Table \ref{tab:bdtsettings} list the set of parameters found to be most discriminant with minimal overtraining as shown in Figure \ref{fig:output_2lss}.  

\begin{table} [!h]
\centering
\begin{tabular}{lll}\hline
  \verb|TMVA.Types.kBDT                  | \\\hline
  \verb|Option            Default   Used |\\
  \verb|NTrees            200       800  |\\
  \verb|BoostType         AdaBoost  Grad | \\
  \verb|Shrinkage         1         0.1  | \\ 
  \verb|nCuts             20        50   | \\
  \verb|MaxDepth          3              | \\ \hline
\end{tabular}
\caption[Configuration used in the final BDTG training.]{Configuration used in the final BDTG training. Parameters not listed were not tested.}\label{tab:bdtsettings}
\end{table}

\begin{figure} [!h]
  \centering
   \includegraphics[width=0.49\textwidth]{bdt_output_ttv_2lss.pdf}
   \includegraphics[width=0.49\textwidth]{bdt_output_tt_2lss.pdf}\\
   \includegraphics[width=0.49\textwidth]{bdt_response_ttv_3l.pdf}
   \includegraphics[width=0.49\textwidth]{bdt_response_tt_3l.pdf}
\caption[BDTG classifier output.]{BDTG classifiers output for training against \ttV (left) and \ttbar(right) for $2lss$ channel(top) and $3l$ channel (bottom) .}
\label{fig:output_2lss}
\end{figure}

\begin{table}[h!]
%\centering
\footnotesize
\begin{tabular}{llrlr}\hline
      &\multicolumn{2}{c}{$2lss$ channel}         & \multicolumn{2}{c}{$3l$ channel}            \\\hline
      &\ttbar training       & \ttV training       & \ttbar training      & \ttV training        \\%\hline
Rank  & Variable             & Variable            & Variable             & Variable             \\ \hline
    1 & minDRll              & dEtaFwdJetBJet      & dEtaFwdJetClosestLep & maxEtaJet25          \\
    2 & dEtaFwdJetClosestLep & Lep3Pt              & minDRll              & dEtaFwdJet2BJet      \\
    3 & dEtaFwdJetBJet       & maxEtaJet25         & maxEtaJet25          & dEtaFwdJetBJet       \\
    4 & dPhiHighestPtSSPair  & dEtaFwdJet2BJet     & dPhiHighestPtSSPair  & Lep2Pt               \\
    5 & Lep3Pt               & dEtaFwdJetClosestLep& Lep2Pt               & dEtaFwdJetClosestLep \\
    6 & maxEtaJet25          & minDRll             & dEtaFwdJetBJet       & minDRll              \\
    7 & dEtaFwdJet2BJet      & dPhiHighestPtSSPair & dEtaFwdJet2BJet      & nJet25               \\
    8 & nJetEta1             & nJet25              & nJetEta1             & dPhiHighestPtSSPair  \\
    9 & nJet25               & nJetEta1            & nJet25               & nJetEta1             \\
   10 & lepCharge            & lepCharge           & lepCharge            & lepCharge            \\\hline
\end{tabular}
\caption[Input variables ranking for BDTG classifiers]{ Input variables ranking for BDTG classifiers for the trainings in the $3l$ channel and $2lss$ channel. In both trainings the rankings show almost the same 5 variables in the first places.}
\label{ranking}
\end{table}

The ranking of the input variables by their importance in the classification process is shown in Table~\ref{ranking}; for both trainings the rankings show almost the same 5 variables in the first places.


%% \begin{table}[h!]
%% \centering
%% \footnotesize
%% \begin{tabular}{lllll}\hline
%%       &\multicolumn{2}{c}{\ttbar training}  & \multicolumn{2}{c}{\ttV training}\\\hline
%% Rank  & Variable             & Importance  & Variable             & Importance \\ \hline
%%     1 & minDRll              & 1.329e-01   & dEtaFwdJetBJet       & 1.264e-01\\
%%     2 & dEtaFwdJetClosestLep & 1.294e-01   & Lep3Pt               & 1.224e-01\\
%%     3 & dEtaFwdJetBJet       & 1.209e-01   & maxEtaJet25          & 1.221e-01\\
%%     4 & dPhiHighestPtSSPair  & 1.192e-01   & dEtaFwdJet2BJet      & 1.204e-01\\
%%     5 & Lep3Pt               & 1.158e-01   & dEtaFwdJetClosestLep & 1.177e-01\\
%%     6 & maxEtaJet25          & 1.121e-01   & minDRll              & 1.143e-01\\
%%     7 & dEtaFwdJet2BJet      & 9.363e-02   & dPhiHighestPtSSPair  & 9.777e-02\\
%%     8 & nJetEta1             & 6.730e-02   & nJet25\_Recl         & 9.034e-02\\
%%     9 & nJet25\_Recl         & 6.178e-02   & nJetEta1             & 4.749e-02\\
%%    10 & lepCharge            & 4.701e-02   & lepCharge            & 4.116e-02\\\hline
%%     1 & dEtaFwdJetClosestLep & 1.394e-01   & maxEtaJet25          & 1.357e-01\\ 
%%     2 & minDRll              & 1.359e-01   & dEtaFwdJet2BJet      & 1.267e-01\\
%%     3 & maxEtaJet25          & 1.308e-01   & dEtaFwdJetBJet       & 1.200e-01\\
%%     4 & dPhiHighestPtSSPair  & 1.116e-01   & Lep2Pt               & 1.196e-01\\
%%     5 & Lep2Pt               & 1.111e-01   & dEtaFwdJetClosestLep & 1.145e-01\\
%%     6 & dEtaFwdJetBJet       & 1.067e-01   & minDRll              & 1.077e-01\\
%%     7 & dEtaFwdJet2BJet      & 8.906e-02   & nJet25\_Recl         & 1.020e-01\\
%%     8 & nJetEta1             & 6.445e-02   & dPhiHighestPtSSPair  & 8.232e-02\\
%%     9 & nJet25               & 6.254e-02   & nJetEta1             & 5.948e-02\\
%%    10 & lepCharge            & 4.848e-02   & lepCharge            & 3.198e-02\\ \hline
%% \end{tabular}
%% \caption[Input variables ranking for BDTG classifiers]{ Input variables ranking for BDTG classifiers for the trainings in the $2lss$ channel (first section) and $3l$ channel (second section). For both trainings the rankings show almost the same 5 variables in the first places.}
%% \label{ranking}
%% \end{table}

\subsection{Additional discriminating variables}

\begin{figure} [!h]
  \centering
   \includegraphics[width=0.9\textwidth]{fwd_add_var_ttv_3l.pdf}\\
   \includegraphics[width=0.9\textwidth]{fwd_add_var_tt_3l.pdf}
\caption[Additional discriminating variables distributions.]{Additional discriminating variables distributions for \ttV training (top row) and \ttbar training (bottom row) in the $3l$ channel. The origin of the jets in the forward jet identification distribution is tagged as 0 for \ti{pileup jets} while \ti{primary vertex jets} are tagged as 1.}
\label{fwd_add_var_3l}
\end{figure}

Given that the forward jet in background processes could be originated from pileup, two additional discriminating variables accounting for that were tested. These additional variables describe the forward jet momentum (fwdJetPt25) and the forward jet identification(fwdJetPUID); their distributions in the $3l$ channel are shown in Figure ~\ref{fwd_add_var_3l}. The forward jet identification distribution show that for both, signal and background, jets are mostly originated in the primary vertex. 

The testing was performed by including in the BDTG input one variable at a time, so the discrimination power of each variable can be evaluated individually, and then both simultaneously. fwdJetPUID was ranked in the last place in importance (11) in both training (\ttV and \ttbar) while fwdJetPt25 was ranked 3 in the \ttV training and 7 in the \ttbar training. When training using 12 variables, fwdJetPt25 was ranked 5 and 7 in the \ttV and \ttbar trainings respectively, while fwdJetPUID was ranked 12 in both cases.

\begin{table}[!hb]
\centering
\begin{tabular}{lcc}\hline
               &\multicolumn{2}{c}{ROC-integral} \\               
               & \ttV  & \ttbar\\\hline                        
base 10 var    & 0.848 & 0.777\\      
+ fwdJetPUID   & 0.849 & 0.777\\      
+ fwdJetPt25   & 0.856 & 0.787\\      
12 var         & 0.856 & 0.787\\\hline
\end{tabular}
\caption[ROC-integral for all the testing cases.]{ROC-integral for all the testing cases performed in the evaluation of the additional variables discriminating power. The improvement in the discrimination performance provided by the additional variables is about 1\% .}\label{tab:add_var_improvement}
\end{table}

The improvement in the discrimination performance provided by the additional variables is about 1\%, so it was decided not to include them in the procedure. Table ~\ref{tab:add_var_improvement} show the ROC-integral for all the testing cases performed.

\subsection{Signal extraction procedure}

\begin{figure} [!h]
 \centering
 \includegraphics[width=0.49\textwidth]{hthq.pdf}
 \includegraphics[width=0.49\textwidth]{hthw.pdf}\\
 \includegraphics[width=0.49\textwidth]{hbg.pdf}
 \includegraphics[width=0.49\textwidth]{hratio.pdf}
\caption[2D BDT classifier output planes]{BDT classifier output planes (training vs \ttbar\ on x-axis and vs \ttV\ on y-axis) for the \tHq\ and \tHW\ signals (top row), and for the combined backgrounds (bottom left). Bottom right: S/B ratio (combining \tHq\ and \tHW) in the same plane. Plots are for $3l$ channel.}
\label{fig:mva12}
\end{figure}

Once the two BDTG classifiers, introduced in the previous section, are trained against the dominant backgrounds in each channel, they are used to classify the events in the samples; their outputs are then used to evaluate the signal cross section limits in a fit to the classifier shape. Figure ~\ref{fig:mva12} shows the expected output distributions in a 2D plane of one training vs.\ the other, \ie, \ttV\ vs.\ \ttbar. Top row shows the 2D planes for \tHq and \tHW signals, while the bottom left plot shows the corresponding 2D plane for the combined backgrounds, which are evaluated as in the final background prediction, \ie,\ these are not the samples used in the BDTG training and this includes data-driven backgrounds. The signal (combining of \tHq and \tHW) to background ratio (S/B) is showed in the bottom right plot of Figure ~\ref{fig:mva12}.      

Each event is now classified into one of ten 2D-bins according to its position in the plane, as shown in Figure ~\ref{fig:binning}. The number of bins is chosen such that no bins are entirely empty for any process. The bin boundary positions and number of bins have been studied and optimized with respect to the expected limit on the signal strength (see Sec.~\ref{sec:binopt}).

\begin{figure} [!h]
 \centering
 \includegraphics[width=0.8\textwidth]{hratio_binning.pdf}
\caption{Binning overlaid on the S/B ratio map on the plane of classifier outputs.}
\label{fig:binning}
\end{figure}

From this event categorization, a 1D histogram of expected distribution is produced for each signal and background process, and fit to the observed data (or the Asimov dataset for expected limits).

\subsection{Binning and selection optimization}\label{sec:binopt}

The effect of the choice of pre-selection cuts and the number of bins of the 1D histogram on the cross section limit is evaluated by varying the most important cuts and re-calculating the limit in each case. In this analysis, the optimization was performed in the $3l$ channel, by evaluating the upper limits on the \tHq+\ \tHW\ expected signal strength only (without \ttH component), always evaluated at $\Ct=-1.0$, $\CV=1.0$.

Table~\ref{cut_limit} shows the several variations explored, compared with a baseline; the baseline is similar to the selection reported in Table~\ref{tab:cuts} but only a loose CSV jet and a Z veto of $\pm10$ GeV are required. 

\begin{table}[h!]
\centering
\begin{tabular}{lll}
Selection                         & Variation                & Expected limit \\ \hline
Baseline                          &                          & $<2.93$\\
Loose CSV tags                    & $\geq 1 \to \geq 2$      & $<3.81$\\
Medium CSV tags                   & $\geq 0 \to \geq 1$      & $<2.76$\\
Light forward jet $\eta$          & $\geq 0 \to \geq 1$      & $<2.94$\\
Light forward jet $\eta$          & $\geq 0 \to \geq 1.5$    & $<3.00$\\
MET>30 GeV                        &                          & $<2.91$\\
Z veto ($|m_{\ell\ell}-m_Z|$)     & $>10$GeV $\to >15$ GeV   & $<2.79$\\
One medium CSV + 15 GeV\ Z veto   & combined                 & $<2.62$\\\hline
\end{tabular}
\caption[Selection cuts optimization.]{Signal strength limit variation as a function of tighter cuts. The baseline selection corresponds to a looser selection compared to the one reported in Tab.~\ref{tab:cuts} where only a CSV-loose \bjet is required, and the Z veto is loosened to $\pm10$ GeV. The optimal selection determined here corresponds to the baseline plus the two variations in the last row.}
\label{cut_limit}
\end{table}

The optimal limit is found when requiring a slightly tighter selection with respect to the baseline. The optimal selection is reported in Table~\ref{tab:cuts}.

The signal strength limit also depends on the chosen binning in the 2D plane as the S/B ratio varies across the plane, hence, several sizes and binning combinations were tested in order to improve the limit. Figure ~\ref{bins} shows some of the binning combinations tested; in the default combination all the bins have the same size, while the best limit was found for a set of 10 bins. The bin borders and the resulting limits are shown in Table ~\ref{bin_limits}.

\begin{figure} [!h]
 \centering
 \includegraphics[width=\textwidth]{bin_scheme.pdf} 
\caption{Binning combination scheme.}
\label{bins}
\end{figure}

\begin{table}[h!]
\centering
\begin{tabular}{llllllll}\hline
Number of bins  & \multicolumn{6}{c}{Bin borders}  & Expected limit \\%\hline 
                &$x_1$&$x_2$&$x_3$&$y_1$&$y_2$&$y_3$&\\\hline           
16 (default)    &-0.5 & 0.0 & 0.5 &-0.5 & 0.0 & 0.5 & $<2.91$\\
16              &-0.5 & 0.3 & 0.7 &-0.5 & 0.3 & 0.7 & $<2.83$\\
10              &-0.5 & 0.0 & 0.5 &-0.5 & 0.0 & 0.5 & $<2.93$\\
10              &-0.5 & 0.0 & 0.7 &-0.5 & 0.0 & 0.7 & $<2.86$\\
10              &-0.5 & 0.0 & 0.7 &-0.5 & 0.0 & 0.5 & $<2.84$\\
10              &-0.5 & 0.0 & 0.5 &-0.5 & 0.0 & 0.7 & $<2.87$\\
\textbf{10}     &\textbf{-0.5} &\textbf{0.4} &\textbf{0.7} &\textbf{-0.5} &\textbf{0.4} &\textbf{0.7} &$\mathbf{<2.81}$\\\hline
\end{tabular}
\caption[Limit variation as a function of bin size.]{Limit variation as a function of bin size. The final bin borders used in the $3l$ channel are indicated in bold.}
\label{bin_limits}
\end{table}

Combining the optimization of binning and using the tighter pre-selection cuts, the expected limit in the $3l$ channel alone reaches \textbf{r<2.59}.

A similar binning optimization was made for $2lss$ channel, including other binning combinations. First, the $3l$ channel binning was used to estimate the expected limit, then, bin borders were varied to obtain the best possible expected limit. The bin borders and the resulting signal strength limits for the same-sign dimuon channel are shown in Table~\ref{bin_limits_2lss}:

\begin{table}[h!]
\centering
\begin{tabular}{llllllll}\hline
Number of bins  & \multicolumn{6}{c}{Bin borders}  & Expected limit \\
                &$x_1$&$x_2$&$x_3$&$y_1$&$y_2$&$y_3$&\\\hline
16              &-0.5 & 0.4 & 0.7 &-0.5 & 0.4 & 0.7 & $<1.72$\\
12              &-0.5 & 0.4 & 0.7 &-0.5 & 0.4 & 0.7 & $<1.72$\\
12              &-0.3 & 0.4 & 0.7 &-0.5 & 0.4 & 0.7 & $<1.71$\\
12              &-0.3 & 0.3 & 0.7 &-0.5 & 0.4 & 0.7 & $<1.71$\\
12              &-0.3 & 0.3 & 0.7 &-0.4 & 0.4 & 0.7 & $<1.70$\\
12              &-0.3 & 0.3 & 0.7 &-0.3 & 0.4 & 0.7 & $<1.70$\\
12              &-0.3 & 0.3 & 0.7 &-0.3 & 0.2 & 0.7 & $<1.68$\\
12              &-0.3 & 0.3 & 0.7 &-0.3 & 0.1 & 0.7 & $<1.70$\\
12              &-0.3 & 0.3 & 0.7 &-0.3 & 0.2 & 0.6 & $<1.70$\\
10              &-0.5 & 0.4 & 0.7 &-0.5 & 0.4 & 0.7 & $<1.75$\\
\textbf{10}     &\textbf{-0.3} &\textbf{ 0.3} &\textbf{ 0.7} &\textbf{-0.3} &\textbf{ 0.2} &\textbf{ 0.6} &$\mathbf{<1.69}$\\\hline
\end{tabular}
\caption{Limit variation as a function of bin size in the same-sign dimuon channel. (In bold: the final bin borders used in the $2lss$ channel.)}
\label{bin_limits_2lss}
\end{table}

The expected limit was found to be \textbf{r<1.69} for optimized bin borders in 10 bins and optimized pre-selection cuts.

Two additional binning strategies were tested, however, the obtained limits are degraded; they are documented in Appendix \ref{app:ad_binning}.   












. Multivariate techniques are used to discriminate the signal from the dominant backgrounds. The analysis yields a 95\% confidence level (C.L.) upper limit on the combined tH + ttH production cross section times branching ratio of 0.64 pb, with an expected limit of 0.32 pb, for a scenario with kt = −1.0 and kV = 1.0. Values of kt outside the range of −1.25 to +1.60 are excluded at 95\% C.L., assuming kV = 1.0.

Dont forget to mention previous constrains to ct check Reference \ref{biswas} and References https://link.springer.com/content/pdf/10.1007\%2FJHEP01\%282013\%29088.pdf (paragraph after eq 2)






























%%%%-------------------signal model----------------

\section{Signal model}

The goal of this analysis is to test the compatibility of points in the parameter space of Higgs-to-vector boson and Higgs-to-top quark couplings. The simulated \tHq, \tHW, and \ttH\ signal events are used with event-by-event weights to reflect the impact of the couplings on kinematic distributions, and together with different predictions of the respective production cross sections and branching ratios, we can produce limits for different values of \CV\ and \Ct. (See Tab.~\ref{tab:reweight} for the set of \Ct\ and \CV\ values generated.) The slight shape-dependence of the BDT outputs as a function of the couplings is documented in Appendix ~\ref{sec:bdtvscvct}.

Apart from the \Ct/\CV\ interference of the \tHq\ and \tHW\ production cross sections, the cross section of \ttH\ scales as $\Ct^2$. Furthermore, the Higgs branching fractions to vector bosons depend on \CV, and the overall Higgs decay width depend both on \Ct\ and \CV\ when considering resolved top-quark loops in the $H\to\gamma\gamma$, $H\to Z\gamma$, and $H\to gg$ decays. The relative contributions from $ H\to\WW$, $H\to\ZZ$, and $H\to\tautau$ changes with changing \CV.

We hence set an upper limit on the combined cross section times branching ratio of \tHq, \tHW, and \ttH.

If we assume a modifier for the Higgs-to-tau coupling ($\kappa_\tau$) to be equal to $\Ct$, the relative fractions of $\WW$, $\ZZ$, and $\tautau$ in our selection will only depend on the ratio of $\Ct/\CV$.
Any limit set at any given value of $\Ct/\CV$ is thus valid for all values of $\Ct$ and $\CV$ with that ratio, and could then be compared with theoretical predictions of cross sections at different values of either modifier.
Rather than as a function of the $\Ct/\CV$ ratio, limits could (equivalently) be reported as a function of the relative strength of Higgs-top and Higgs-vector-boson couplings, multiplied by the relative sign.
Such a parameter, further referred to as \ft, as defined in Equation ~\ref{eq:ft}, spans the entire possible parameter space between $-1.0$ and $1.0$, with the SM expectation at $0.5$.
Absolute values of $1.0$ or $0.0$ would then correspond to purely Higgs-top and purely Higgs-V couplings, respectively.

\begin{equation} \label{eq:ft}
	\ft = \mathrm{sign}(\Ct/\CV) \times \frac{\Ct^2}{\Ct^2+\CV^2}.
\end{equation}

Table~\ref{tab:ctcvvalues} shows the points in the $\Ct/\CV$ and \ft\ parameter space that are mapped by the 51 individual \Ct\ and \CV\ points.

\begin{table}[h!]
\centering
\begin{tabular}{rrrrr}
 $\ft$ & \Ct/\CV & $\CV=0.5$ & $\CV=1.0$ & $\CV=1.5$ \\ \hline
  -0.973 & -6.000 & -3.00 &       &       \\
  -0.941 & -4.000 & -2.00 &       &       \\
  -0.900 & -3.000 & -1.50 & -3.00 &       \\
  -0.862 & -2.500 & -1.25 &       &       \\
  -0.800 & -2.000 & -1.00 & -2.00 & -3.00 \\
  -0.692 & -1.500 & -0.75 & -1.50 &       \\
  -0.640 & -1.333 &       &       & -2.00 \\
  -0.610 & -1.250 &       & -1.25 &       \\
  -0.500 & -1.000 & -0.50 & -1.00 & -1.50 \\
  -0.410 & -0.833 &       &       & -1.25 \\
  -0.360 & -0.750 &       & -0.75 &       \\
  -0.308 & -0.667 &       &       & -1.00 \\
  -0.200 & -0.500 & -0.25 & -0.50 & -0.75 \\
  -0.100 & -0.333 &       &       & -0.50 \\
  -0.059 & -0.250 &       & -0.25 &       \\
  -0.027 & -0.167 &       &       & -0.25 \\
   0.000 &  0.000 &  0.00 &  0.00 &  0.00 \\
   0.027 &  0.167 &       &       &  0.25 \\
   0.059 &  0.250 &       &  0.25 &       \\
   0.100 &  0.333 &       &       &  0.50 \\
   0.200 &  0.500 &  0.25 &  0.50 &  0.75 \\
   0.308 &  0.667 &       &       &  1.00 \\
   0.360 &  0.750 &       &  0.75 &       \\
   0.410 &  0.833 &       &       &  1.25 \\
   0.500 &  1.000 &  0.50 &  1.00 &  1.50 \\
   0.610 &  1.250 &       &  1.25 &       \\
   0.640 &  1.333 &       &       &  2.00 \\
   0.692 &  1.500 &  0.75 &  1.50 &       \\
   0.800 &  2.000 &  1.00 &  2.00 &  3.00 \\
   0.862 &  2.500 &  1.25 &       &       \\
   0.900 &  3.000 &  1.50 &  3.00 &       \\
   0.941 &  4.000 &  2.00 &       &       \\
   0.973 &  6.000 &  3.00 &       &       \\ \hline
\end{tabular}
\caption{The 33 distinct values of $\Ct/\CV$ and \ft\ as mapped by the 51 \Ct\ and \CV\ points.}
\label{tab:ctcvvalues}
\end{table}

The overall higgs decay width (modified by both \Ct\ and \CV) becomes irrelevant if limits are quoted as absolute cross sections rather than multiples of the expected cross section (which depends on the overall Higgs decay width).

% Two possibilities are explored: one where the $\gamma\gamma$, $\Z\gamma$, and $\Pg\Pg$ decays are modified with \Ct\ (referred to as the ``resolved'' model henceforth), and one where they are kept fixed at their SM values.
% In both cases, the $\PH\to\cPqc\cPqc$ branching is left unchanged with \Ct.

The 1D histograms of events as categorized in regions of the 2D BDT plane is then used in a maximum likelihood fit of signal and background shapes, where the \tHq, \tHW, and \ttH\ signals are floating with a common signal strength modifier $r$, producing a 95\% C.L. upper limit the observed cross section of $\tHq+\tHW+\ttH$.

This is done separately for each point of \Ct\ and \CV, where the cross sections and branching fractions are scaled accordingly in each point.
Limits at fixed values of $\Ct/\CV$ are by construction identical.
Tables~\ref{tab:brscalingK6_0p5}--\ref{tab:brscalingK6_1p5} and~\ref{tab:xsbrscalingK6_0p5}--\ref{tab:xsbrscalingK6_1p5} in Appendix~\ref{sec:xsbrscalings} show the scalings of cross section times branching fraction, as well as branching fractions alone for each of the Higgs decay modes and each of the signal components.

%% Leaving out the limit plots in each channel for now.
% \begin{figure} [!h]
%  \centering
%  \includegraphics[width=0.32\textwidth]{figures/limits/limits_3l_cv_1p0.pdf}
%  \includegraphics[width=0.32\textwidth]{figures/limits/limits_2lss_mm_cv_1p0.pdf}
%  \includegraphics[width=0.32\textwidth]{figures/limits/limits_2lss_em_cv_1p0.pdf} \\
%  \includegraphics[width=0.32\textwidth]{figures/limits/limits_3l_cv_1p5.pdf}
%  \includegraphics[width=0.32\textwidth]{figures/limits/limits_2lss_mm_cv_1p5.pdf}
%  \includegraphics[width=0.32\textwidth]{figures/limits/limits_2lss_em_cv_1p5.pdf} \\
%  \includegraphics[width=0.32\textwidth]{figures/limits/limits_3l_cv_0p5.pdf}
%  \includegraphics[width=0.32\textwidth]{figures/limits/limits_2lss_mm_cv_0p5.pdf}
%  \includegraphics[width=0.32\textwidth]{figures/limits/limits_2lss_em_cv_0p5.pdf}
% \
% \caption{Expected asymptotic limit on $\frac{\sigma}{\sigma_{theor}}$ as a function of \Ct\ for $\CV=1.0$, $\CV=1.5$, $\CV=0.5$ (top to bottom) for the three lepton channel (left), the \mumu\ channel (middle), and the \emu\ channel (right).}
% \label{fig:limits_cv_3l}
% \end{figure}














































































%% This is done separately for each point of \Ct\ and \CV, where the cross sections and branching fractions are scaled accordingly in each point.
%% Limits at fixed values of $\Ct/\CV$ are by construction identical.
%% Tables~\ref{tab:brscalingK6_0p5}--\ref{tab:brscalingK6_1p5} and~\ref{tab:xsbrscalingK6_0p5}--\ref{tab:xsbrscalingK6_1p5} in Appendix~\ref{sec:xsbrscalings} show the scalings of cross section times branching fraction, as well as branching fractions alone for each of the Higgs decay modes and each of the signal components.

%% %% Leaving out the limit plots in each channel for now.
%% % \begin{figure} [!h]
%% %  \centering
%% %  \includegraphics[width=0.32\textwidth]{limits/limits_3l_cv_1p0.pdf}
%% %  \includegraphics[width=0.32\textwidth]{limits/limits_2lss_mm_cv_1p0.pdf}
%% %  \includegraphics[width=0.32\textwidth]{limits/limits_2lss_em_cv_1p0.pdf} \\
%% %  \includegraphics[width=0.32\textwidth]{limits/limits_3l_cv_1p5.pdf}
%% %  \includegraphics[width=0.32\textwidth]{limits/limits_2lss_mm_cv_1p5.pdf}
%% %  \includegraphics[width=0.32\textwidth]{limits/limits_2lss_em_cv_1p5.pdf} \\
%% %  \includegraphics[width=0.32\textwidth]{limits/limits_3l_cv_0p5.pdf}
%% %  \includegraphics[width=0.32\textwidth]{limits/limits_2lss_mm_cv_0p5.pdf}
%% %  \includegraphics[width=0.32\textwidth]{limits/limits_2lss_em_cv_0p5.pdf}
%% % \
%% % \caption{Expected asymptotic limit on $\frac{\sigma}{\sigma_{theor}}$ as a function of \Ct\ for $\CV=1.0$, $\CV=1.5$, $\CV=0.5$ (top to bottom) for the $3l$ channel (left), the \mumu\ channel (middle), and the \emu\ channel (right).}
%% % \label{fig:limits_cv_3l}
%% % \end{figure}





%% %______________________ Systematic errors ______________________

Systematic uncertainties on the signal selection efficiency arise from correction factors applied
to the simulated events to better match the measured detector performance and also from theoretical
uncertainties in the modeling of the signal process.
Scale factors applied to correct for data/MC differences in the trigger efficiency, lepton reconstruction
and identification performance, and lepton selection efficiency carry a combined
uncertainty of about 5% per lepton. The impact of the uncertainty in the signal selection efficiency
from jet energy corrections is evaluated by varying the correction factors within their
uncertainty and propagating the effect to the final result by recalculating all kinematic quantities.
Effects on the overall normalization of event yields and on the shape of kinematic properties
are both taken into account. Jet energy resolution effects have negligible impact on this

analysis. Correction factors for data/MC differences in the b-tagging performance are applied
depending on the pT and η, and on the flavor of the jet, and their effect on the signal efficiency
is evaluated by varying the factors within their measured uncertainty and recalculating the
overall event scale factors.
The uncertainties from unknown higher orders of tHq and tHW production are estimated from
a change in the Q2
scale of double and half the initial value, evaluated for each point of κt
and κV. The ttH signal component has an uncertainty of about +5.8/− 9.2% from Q2
scale
variations and a further 3.6% from the knowledge of PDFs and αS [41].
Uncertainties related to the choice of PDF set and its scale are estimated to be about 3.7% for
tHq and about 4.0% for tHW.
































%% \section{Systematic errors }
%% \label{secc:sys}

%% Table~\ref{tab:uncertainties} shows all sources of systematic uncertainty currently considered in the analysis.
%% \begin{table}[h!]
%%   \centering
%%   \begin{tabular}{lll}\hline
%% Source                          & Channel     & Size \\\hline
%% \multicolumn{3}{l}{\bf Experimental uncertainties} \\
%% Luminosity                      & all         & 1.026 \\
%% Loose lepton efficiency         &             & 1.02 per lepton  \\
%% Tight lepton efficiency         &             & 1.03 per lepton  \\
%% Trigger efficiency              & \mumu\      & 1.01 \\
%%                                 & \emu\       & 1.01 \\
%%                                 & \ee\        & 1.02 \\
%%                                 & \threel\    & 1.03 \\
%% Jet energy scale                & all         & templates \\
%% Forward jet modeling            & all         & templates, see Tab.~\ref{tab:ratioFwdJet} \\
%% \cPqb\ tagging efficiency       & all         & templates \\ \hline

%% \multicolumn{3}{l}{\bf Theory uncertainties} \\
%% $Q^2$ scale (\tHq)              & all         & 0.92--1.06 (depending on \Ct, \CV)\\
%% $Q^2$ scale (\tHW)              & all         & 0.93--1.05 (depending on \Ct, \CV)\\
%% $Q^2$ scale (\ttH)              & all         & 0.915/1.058\\
%% $Q^2$ scale (\ttW)              & all         & 1.12\\
%% $Q^2$ scale (\ttZ)              & all         & 1.11\\
%% pdf (\ttH)                      & all         & 1.036\\
%% pdf $\Pg\Pg$ (\ttZ)             & all         & 0.966\\
%% pdf $\Pq\Paq$ (\ttW)            & all         & 1.04\\
%% pdf $\Pq\Pg$ (\tHq)             & all         & 1.037\\
%% pdf $\Pq\Pg$ (\tHW)             & all         & 1.040\\ \hline
%% \multicolumn{3}{l}{\bf Higgs branching fractions} \\
%% \verb|param_alphaS|             & all         & 1.012\\
%% \verb|param_mB|                 & all         & 0.981\\
%% \verb|HiggsDecayWidthTHU_hqq|   & all         & 0.988\\
%% \verb|HiggsDecayWidthTHU_hvv|   & all         & 1.004\\
%% \verb|HiggsDecayWidthTHU_hll|   & all         & 1.019\\\hline

%% \multicolumn{3}{l}{\bf Backgrounds}         \\
%% \WZ\ control region statistics  & \threel\    & 1.10 \\
%% \WZ\ control region backgrounds & \threel\    & 1.20 \\
%% \WZ\ modeling                   & \threel\    & 1.07  \\
%% $\WZ+2\text{jet}$ background    & \mumu,\emu\ & 1.50 \\
%% Rare SM processes               & all         & 1.50 \\
%% Charge flips                    & \emu\       & 1.30 \\\hline
%% \multicolumn{3}{l}{\bf Fake rate estimate}     \\
%% Electron FR measurement         &             & templates \\
%% Muon FR measurement             &             & templates \\
%% Electron closure                & \ee\        & 1.05 norm., (0.99 (\ttbar)/1.06 (\ttV)) shape var. \\
%%                                 & \emu\       & 0.94 norm., (0.98 (\ttbar)/1.07 (\ttV)) shape var. \\
%%                                 & \threel\    & 1.40 norm., (1.09 (\ttbar)/1.05 (\ttV)) shape var. \\
%% Muon closure                    & \mumu\      & 1.07 norm., (0.97 (\ttbar)/0.91 (\ttV)) shape var. \\
%%                                 & \emu\       & 1.09 norm., (1.06 (\ttbar)/1.03 (\ttV)) shape var. \\
%%                                 & \threel\    & 1.09 norm., (0.95 (\ttbar)/0.83 (\ttV)) shape var. \\\hline
%%    \end{tabular} 
%%    \caption{Pre-fit size of systematic uncertainties.}\label{tab:uncertainties}
%%  \end{table}

%% \textbf{Experimental uncertainties}
%% A normalization uncertainty is derived from the measurement of data/MC scale factors for lepton and trigger efficiencies.
%% Jet energy scale uncertainties and \cPqb\ tagging efficiency are evaluated using dedicated shape templates derived from a variation of the jet energy scale within its uncertainty and from varying the \cPqb\ tagging data/MC scale factors within their uncertainty.

%% The forward jet $\eta$ distribution is poorly modeled in simulation, see Appendix~\ref{app:fwdcontrol}.
%% To estimate the effect of a mismodeled forward jet distribution, we reweight the events in simulation (\ie\ for signal and the irreducible backgrounds) based on the normalized data/MC ratio in the control region and thereby derive an alternative shape of the BDT output distributions that reflects a hypothetical perfect data/MC agreement.

%% \textbf{Theory uncertainties}
%% $Q^2$ scale and parton distribution function (pdf) uncertainties are applied as an overall normalization uncertainty using numbers from the NLO theory calculation.

%% \textbf{Backgrounds}
%% In addition to the theory uncertainties on the main irreducible backgrounds of \ttW, \ttZ, and \ttH, the smaller irreducible backgrounds and the charge mis-identification estimate are covered with flat normalization uncertainties.
%% The \WZ\ contribution is normalized in a data control region and an uncertainty on the scale factor is derived in the process.
%% Finally, the dominant uncertainty relates to the estimate of the reducible non-prompt lepton contribution using a fake rate method.
%% The main normalization uncertainty on the used fake rates derives from limited statistics in the data control region, and the subtraction of residual prompt lepton contribution, see Ref.~\cite{CMS_AN_2017-029}.
%% Furthermore, shape variations resembling data/MC differences and deviations in closure test are evaluated as shape uncertainties.

%% \textbf{Fake rate closure uncertainties}
%% The BDT output shapes are compared between a pure MC estimation of fake leptons (in \ttbar), and an application of fake-rates as measured in QCD MC, applied in \ttbar\ MC events.
%% The difference in the resulting normalization and output shapes, both the training vs. \ttbar\ and vs. \ttV, are estimated and propagated to the fit as normalization and shape variations.
%% See Figs~\ref{fig:frclosure_2lss_ee} to~\ref{fig:frclosure_3l_mufake} for the results of these closure tests and Tab.~\ref{tab:uncertainties} for the resulting pre-fit uncertainties.

%% \begin{figure}[htb]
%%  \centering
%%  \includegraphics[width=0.245\textwidth]{FR_closures/thqMVA_tt_2lss_ee_norm.pdf} 
%%  \includegraphics[width=0.245\textwidth]{FR_closures/thqMVA_ttv_2lss_ee_norm.pdf} 
%%  \includegraphics[width=0.245\textwidth]{FR_closures/thqMVA_tt_2lss_ee_shape.pdf} 
%%  \includegraphics[width=0.245\textwidth]{FR_closures/thqMVA_ttv_2lss_ee_shape.pdf}\\ 
%% \caption{BDT outputs comparing \ttbar\ MC to a fake-rate prediction using fake rates measured in QCD MC.\@ Agreement in normalization is estimated from the left two plots, shape disagreement is estimated from the right two (normalized) plots. Same-sign \ee\ selection.} 
%% \label{fig:frclosure_2lss_ee}
%% \end{figure} 

%% \begin{figure}[htb]
%%  \centering
%%  \includegraphics[width=0.245\textwidth]{FR_closures/thqMVA_tt_2lss_em_elfake_norm.pdf} 
%%  \includegraphics[width=0.245\textwidth]{FR_closures/thqMVA_ttv_2lss_em_elfake_norm.pdf} 
%%  \includegraphics[width=0.245\textwidth]{FR_closures/thqMVA_tt_2lss_em_elfake_shape.pdf} 
%%  \includegraphics[width=0.245\textwidth]{FR_closures/thqMVA_ttv_2lss_em_elfake_shape.pdf}\\ 
%% \caption{BDT outputs comparing \ttbar\ MC to a fake-rate prediction using fake rates measured in QCD MC.\@ Agreement in normalization is estimated from the left two plots, shape disagreement is estimated from the right two (normalized) plots. Same-sign \emu\ selection with electron fakes.} 
%% \label{fig:frclosure_2lss_em_elfake}
%% \end{figure} 

%% \begin{figure}[htb]
%%  \centering
%%  \includegraphics[width=0.245\textwidth]{FR_closures/thqMVA_tt_2lss_em_mufake_norm.pdf} 
%%  \includegraphics[width=0.245\textwidth]{FR_closures/thqMVA_ttv_2lss_em_mufake_norm.pdf} 
%%  \includegraphics[width=0.245\textwidth]{FR_closures/thqMVA_tt_2lss_em_mufake_shape.pdf} 
%%  \includegraphics[width=0.245\textwidth]{FR_closures/thqMVA_ttv_2lss_em_mufake_shape.pdf}\\ 
%% \caption{BDT outputs comparing \ttbar\ MC to a fake-rate prediction using fake rates measured in QCD MC.\@ Agreement in normalization is estimated from the left two plots, shape disagreement is estimated from the right two (normalized) plots. Same-sign \emu\ selection with muon fakes.} 
%% \label{fig:frclosure_2lss_em_mufake}
%% \end{figure} 

%% \begin{figure}[htb]
%%  \centering
%%  \includegraphics[width=0.245\textwidth]{FR_closures/thqMVA_tt_2lss_mm_norm.pdf} 
%%  \includegraphics[width=0.245\textwidth]{FR_closures/thqMVA_ttv_2lss_mm_norm.pdf} 
%%  \includegraphics[width=0.245\textwidth]{FR_closures/thqMVA_tt_2lss_mm_shape.pdf} 
%%  \includegraphics[width=0.245\textwidth]{FR_closures/thqMVA_ttv_2lss_mm_shape.pdf} \\
%% \caption{BDT outputs comparing \ttbar\ MC to a fake-rate prediction using fake rates measured in QCD MC.\@ Agreement in normalization is estimated from the left two plots, shape disagreement is estimated from the right two (normalized) plots. Same-sign \mumu\ selection.} 
%% \label{fig:frclosure_2lss_mm}
%% \end{figure} 

%% \begin{figure}[htb]
%%  \centering
%%  \includegraphics[width=0.245\textwidth]{FR_closures/thqMVA_tt_3l_elfake_norm.pdf} 
%%  \includegraphics[width=0.245\textwidth]{FR_closures/thqMVA_ttv_3l_elfake_norm.pdf} 
%%  \includegraphics[width=0.245\textwidth]{FR_closures/thqMVA_tt_3l_elfake_shape.pdf} 
%%  \includegraphics[width=0.245\textwidth]{FR_closures/thqMVA_ttv_3l_elfake_shape.pdf} \\
%% \caption{BDT outputs comparing \ttbar\ MC to a fake-rate prediction using fake rates measured in QCD MC.\@ Agreement in normalization is estimated from the left two plots, shape disagreement is estimated from the right two (normalized) plots. $3l$ selection with electron fakes.} 
%% \label{fig:frclosure_3l_elfake}
%% \end{figure} 

%% \begin{figure}[htb]
%%  \centering
%%  \includegraphics[width=0.245\textwidth]{FR_closures/thqMVA_tt_3l_mufake_norm.pdf} 
%%  \includegraphics[width=0.245\textwidth]{FR_closures/thqMVA_ttv_3l_mufake_norm.pdf} 
%%  \includegraphics[width=0.245\textwidth]{FR_closures/thqMVA_tt_3l_mufake_shape.pdf} 
%%  \includegraphics[width=0.245\textwidth]{FR_closures/thqMVA_ttv_3l_mufake_shape.pdf} 
%% \caption{BDT outputs comparing \ttbar\ MC to a fake-rate prediction using fake rates measured in QCD MC.\@ Agreement in normalization is estimated from the left two plots, shape disagreement is estimated from the right two (normalized) plots. $3l$ selection with muon fakes.} 
%% \label{fig:frclosure_3l_mufake}
%% \end{figure}


%% %______________________ Results ______________________
%% \section{Results}
%% \label{secc:results}

%% The unblinded distributions of BDT outputs are shown in Fig.~\ref{fig:bdt_outputs}.
%% The pre-fit distributions in the final binning used in the signal extraction are shown in Fig.~\ref{fig:finalbins}, with the post-fit distributions shown in Fig.~\ref{fig:postfit}.
%% \begin{figure} [!h]
%%  \centering
%%  \includegraphics[width=0.32\textwidth]{3lsignal/thqMVA_ttv_3l_40.pdf}
%%  \includegraphics[width=0.32\textwidth]{signalregion_2lss/mumu/thqMVA_ttv_2lss_40.pdf}
%%  \includegraphics[width=0.32\textwidth]{signalregion_2lss/emu/thqMVA_ttv_2lss_40.pdf} \\
%%  % \includegraphics[width=0.24\textwidth]{signalregion_2lss/ee/thqMVA_ttv_2lss_40.pdf} \\
%%  \includegraphics[width=0.32\textwidth]{3lsignal/thqMVA_tt_3l_40.pdf} 
%%  \includegraphics[width=0.32\textwidth]{signalregion_2lss/mumu/thqMVA_tt_2lss_40.pdf}
%%  \includegraphics[width=0.32\textwidth]{signalregion_2lss/emu/thqMVA_tt_2lss_40.pdf}
%%  % \includegraphics[width=0.24\textwidth]{signalregion_2lss/ee/thqMVA_tt_2lss_40.pdf}
%% \
%% \caption{Distribution of individual BDT outputs for (from left to right) the $3l$ channel, the \mumu\ channel, and the \emu\ channel, for training against \ttV\ (top row) and against \ttbar\ (bottom row).}
%% \label{fig:bdt_outputs}
%% \end{figure}

%% \begin{figure} [!h]
%%  \centering
%%  % \includegraphics[width=0.24\textwidth]{3lsignal/finalBins_40.pdf}
%%  % \includegraphics[width=0.24\textwidth]{signalregion_2lss/mumu/finalBins_40.pdf}
%%  % \includegraphics[width=0.24\textwidth]{signalregion_2lss/emu/finalBins_40.pdf}
%%  % \includegraphics[width=0.24\textwidth]{signalregion_2lss/ee/finalBins_40.pdf} \\
%%  % \includegraphics[width=0.24\textwidth]{3lsignal/finalBins_log_40.pdf}
%%  % \includegraphics[width=0.24\textwidth]{signalregion_2lss/mumu/finalBins_log_mm_40.pdf}
%%  % \includegraphics[width=0.24\textwidth]{signalregion_2lss/emu/finalBins_log_em_40.pdf}
%%  % \includegraphics[width=0.24\textwidth]{signalregion_2lss/ee/finalBins_log_ee_40.pdf}
%%  \includegraphics[width=0.32\textwidth]{postfit/tHq_3l_13TeV_prefit.pdf}
%%  \includegraphics[width=0.32\textwidth]{postfit/tHq_2lss_mm_13TeV_prefit.pdf}
%%  \includegraphics[width=0.32\textwidth]{postfit/tHq_2lss_em_13TeV_prefit.pdf} \\
%%  \includegraphics[width=0.32\textwidth]{postfit/tHq_3l_13TeV_prefit_log.pdf}
%%  \includegraphics[width=0.32\textwidth]{postfit/tHq_2lss_mm_13TeV_prefit_log.pdf}
%%  \includegraphics[width=0.32\textwidth]{postfit/tHq_2lss_em_13TeV_prefit_log.pdf}
%% \caption{Expected (pre-fit) distributions in the final binning used for the signal extraction, for (from left to right) the $3l$ channel, the \mumu\ channel, and the \emu\ channel. Linear scale (top row), and logarithmic scale (bottom row).}
%% \label{fig:finalbins}
%% \end{figure}

%% \begin{figure} [!h]
%%  \centering
%%  \includegraphics[width=0.32\textwidth]{postfit/tHq_3l_13TeV_fit_s.pdf}
%%  \includegraphics[width=0.32\textwidth]{postfit/tHq_2lss_mm_13TeV_fit_s.pdf}
%%  \includegraphics[width=0.32\textwidth]{postfit/tHq_2lss_em_13TeV_fit_s.pdf} \\
%%  \includegraphics[width=0.32\textwidth]{postfit/tHq_3l_13TeV_fit_s_log.pdf}
%%  \includegraphics[width=0.32\textwidth]{postfit/tHq_2lss_mm_13TeV_fit_s_log.pdf}
%%  \includegraphics[width=0.32\textwidth]{postfit/tHq_2lss_em_13TeV_fit_s_log.pdf}
%% \caption{Post-fit distributions in the final binning used for the signal extraction, for (from left to right) the $3l$ channel, the \mumu\ channel, and the \emu\ channel. Linear scale (top row), and logarithmic scale (bottom row).}
%% \label{fig:postfit}
%% \end{figure}

%% \begin{figure} [!h]
%%  \centering
%%  \includegraphics[width=0.32\textwidth]{postfit/bgsub/ITC/tHq_3l_13TeV_prefit.pdf}
%%  \includegraphics[width=0.32\textwidth]{postfit/bgsub/ITC/tHq_2lss_mm_13TeV_prefit.pdf}
%%  \includegraphics[width=0.32\textwidth]{postfit/bgsub/ITC/tHq_2lss_em_13TeV_prefit.pdf} \\
%%  \includegraphics[width=0.32\textwidth]{postfit/bgsub/ITC/tHq_3l_13TeV_fit_s.pdf}
%%  \includegraphics[width=0.32\textwidth]{postfit/bgsub/ITC/tHq_2lss_mm_13TeV_fit_s.pdf}
%%  \includegraphics[width=0.32\textwidth]{postfit/bgsub/ITC/tHq_2lss_em_13TeV_fit_s.pdf}
%% \caption{Background-subtracted pre- (top) and post-fit (bottom) distributions in the final binning used for the signal extraction, for (from left to right) the $3l$ channel, the \mumu\ channel, and the \emu\ channel. For a fit in the inverted couplings scenario, as Figs.~\ref{fig:finalbins} and~\ref{fig:postfit}.}
%% \label{fig:postfit_bgsub_ITC}
%% \end{figure}

%% \begin{figure} [!h]
%%  \centering
%%  \includegraphics[width=0.32\textwidth]{postfit/bgsub/SM/tHq_3l_13TeV_prefit.pdf}
%%  \includegraphics[width=0.32\textwidth]{postfit/bgsub/SM/tHq_2lss_mm_13TeV_prefit.pdf}
%%  \includegraphics[width=0.32\textwidth]{postfit/bgsub/SM/tHq_2lss_em_13TeV_prefit.pdf} \\
%%  \includegraphics[width=0.32\textwidth]{postfit/bgsub/SM/tHq_3l_13TeV_fit_s.pdf}
%%  \includegraphics[width=0.32\textwidth]{postfit/bgsub/SM/tHq_2lss_mm_13TeV_fit_s.pdf}
%%  \includegraphics[width=0.32\textwidth]{postfit/bgsub/SM/tHq_2lss_em_13TeV_fit_s.pdf}
%% \caption{Background-subtracted pre- (top) and post-fit (bottom) distributions in the final binning used for the signal extraction, for (from left to right) the $3l$ channel, the \mumu\ channel, and the \emu\ channel. For a fit in the SM-like scenario ($\Ct=\CV=1$).}
%% \label{fig:postfit_bgsub_SM}
%% \end{figure}

%% We calculate asymptotic upper CL$_\text{S}$ limits at 95\% C.L. on the combined production cross section of \tHq, \tHW, and \ttH\ (reported as an upper limit on the cross section times modified branching ratio), for each of the 51 coupling configurations, see Tab.~\ref{tab:limits}.
%% The limits and best-fit values of the signal cross section (and corresponding signal strength at $\CV=1.0$) for each point are given in Tab.~\ref{tab:xslimits}, and in Tab.~\ref{tab:xslimits_chan} for the two main hypotheses, split by channel.
%% In the SM point a signal strength of $1.82\,^{+0.34}_{-0.33}\mathrm{(stat.)}\,^{+0.55}_{-0.59}\mathrm{(syst.)}$ (compared to the SM cross section at $\CV=1.0$) is obtained, corresponding to a cross section of $0.33\pm0.12\,\mathrm{pb}$.
%% The observed significance of the signal, in a background-only hypothesis, is $2.7\,\sigma$, with an a-priori expected significance of $1.5\,\sigma$.
%% Without considering systematic uncertainties, the significance increases to $6.2\,\sigma$.
%% A scan of the observed and expected significances for each coupling configuration is shown in Fig.~\ref{fig:significances}.

%% % The full results are shown in Fig.~\ref{fig:r_limits_cv} for the non-resolved model with fixed $\PH\to\gamma\gamma/\Z\gamma/\Pg\Pg$ branching ratios, and the two scenarios of exactly inverted couplings ($\CV=1.0$,$\Ct=-1.0$) and the standard model are reported in Tab.~\ref{tab:limits}.  %% FIXME
%% The pulls and impacts of the most important nuisance parameters are shown in Fig.~\ref{fig:impacts}.

%% \begin{table}[h!]
%% \centering
%% \begin{tabular}{llcccccc}
%% Scenario  & Channel   & Obs. Limit & \multicolumn{5}{c}{Exp. Limit}         \\
%%            &                                  &     & $-2\sigma$ &$-1\sigma$ & Median        & $+1\sigma$ & $+2\sigma$  \\ \hline
%% $\CV=1.0$  & \mumu\                           & 2.3          & 0.71 & 0.94 &         1.32  & 1.88 & 2.60 \\
%% $\Ct=-1.0$ & \emu\                            & 1.9          & 0.65 & 0.87 &         1.21  & 1.71 & 2.32 \\
%%            % & \ee\                             & 3.3          & 1.05 & 1.41 &         1.98  & 2.80 & 3.86 \\
%%            & \threel\                         & 1.6          & 0.43 & 0.59 &         0.86  & 1.26 & 1.78 \\
%%            & Combined ($\mu\mu,3\ell$)        & \textbf{1.6} & 0.40 & 0.54 & \textbf{0.78} & 1.12 & 1.57 \\
%%            & Combined ($\mu\mu,\Pe\mu,3\ell$) & \textbf{1.4} & 0.37 & 0.50 & \textbf{0.71} & 1.03 & 1.43 \\ \hline
%%            % & Combined (all channels)          & \textbf{1.6} & 0.37 & 0.51 & \textbf{0.72} & 1.03 & 1.43 \\ \hline
%%   (SM)     & \mumu\                           & 4.9          & 1.20 & 1.61 &         2.27  & 3.24 & 4.54 \\
%% $\CV=1.0$  & \emu\                            & 3.3          & 1.10 & 1.48 &         2.07  & 2.95 & 4.06 \\
%% % $\Ct=1.0$  & \ee\                             & 4.4          & 1.65 & 2.24 &         3.20  & 4.62 & 6.52 \\
%% $\Ct=1.0$  & \threel\                         & 3.0          & 0.91 & 1.22 &         1.73  & 2.49 & 3.47 \\
%%            & Combined ($\mu\mu,3\ell$)        & \textbf{3.4} & 0.79 & 1.07 & \textbf{1.51} & 2.17 & 3.01 \\
%%            & Combined ($\mu\mu,\Pe\mu,3\ell$) & \textbf{3.1} & 0.71 & 0.96 & \textbf{1.36} & 1.94 & 2.70 \\ \hline
%%            % & Combined (all channels)          & \textbf{3.1} & 0.71 & 0.95 & \textbf{1.34} & 1.92 & 2.65 \\ \hline
%% \end{tabular}
%% \caption{Expected and observed CL$_\text{S}$ limits (at 95\% C.L.) on the signal strength of combined $\tH+\ttH$ production in each channel, and for different combinations thereof, for a scenario with inverted couplings ($\CV=1.0$, $\Ct=-1.0$, top section), and for the standard model ($\CV=\Ct=1.0$, bottom section). Numbers are for 35.9\fbinv.}
%% \label{tab:limits}
%% \end{table}

%% \begin{table}[h!]
%%   \begin{center}
%%     \begin{tabular}{llcccc} \hline 
%%       Scenario  & Channel  & Obs. Limit    & \multicolumn{3}{c}{Exp. Limit (pb)}         \\
%%                 &          & (pb)          & Median        & $\pm1\sigma$ & $\pm2\sigma$ \\ \hline \hline
%%    $\Ct/\CV=-1$ & \mumu\   & 1.00          &         0.58  & [0.42, 0.83] & [0.31, 1.15] \\
%%                 & \emu\    & 0.84          &         0.54  & [0.39, 0.76] & [0.29, 1.03] \\
%%                 & \threel\ & 0.70          &         0.38  & [0.26, 0.56] & [0.19, 0.79] \\ 
%%                 & Combined & \textbf{0.64} & \textbf{0.32} & [0.22, 0.46] & [0.16, 0.64] \\ \hline
%%     $\Ct/\CV=1$ & \mumu\   & 0.87          &         0.41  & [0.29, 0.58] & [0.22, 0.82] \\
%%     (SM-like)   & \emu\    & 0.59          &         0.37  & [0.26, 0.53] & [0.20, 0.73] \\
%%                 & \threel\ & 0.54          &         0.31  & [0.22, 0.43] & [0.16, 0.62] \\
%%                 & Combined & \textbf{0.56} & \textbf{0.24} & [0.17, 0.35] & [0.13, 0.49] \\ \hline
%%     \end{tabular}
%%     \caption{Expected and observed 95\% C.L. upper limits on the $\tH+\ttH$ production cross section times $\PH\to\W\W^*+\tautau+\Z\Z^*$ branching ratio for a scenario of inverted couplings ($\Ct/\CV=-1.0$, top rows) and for a standard-model-like signal ($\Ct/\CV=1.0$, bottom rows), in pb. The expected limit is calculated on a background-only Asimov dataset and quoted with $\pm$1$\sigma$ and $\pm$2$\sigma$ probability ranges.
%%     \label{tab:xslimits_chan}}
%%   \end{center}
%% \end{table}


%% \begin{table}[h!]
%% \centering
%% \begin{tabular}{rr|ccc|cc}
%% $f_t$  & \Ct/\CV\ & Exp.\ lim. & SM exp. & Obs.\ lim. & Best fit $\sigma$ [pb] & Best fit $r$ \\ \hline
%% -0.973 & -6.000 & $0.328~_{-0.090}^{+0.136}$ & $0.507~_{-0.158}^{+0.206}$ & 0.603 & $0.305~_{-0.169}^{+0.155}$ & $0.013~_{-0.007}^{+0.007}$ \\
%% -0.941 & -4.000 & $0.335~_{-0.098}^{+0.137}$ & $0.509~_{-0.166}^{+0.215}$ & 0.627 & $0.322~_{-0.174}^{+0.157}$ & $0.036~_{-0.020}^{+0.018}$ \\
%% -0.900 & -3.000 & $0.335~_{-0.096}^{+0.138}$ & $0.510~_{-0.172}^{+0.215}$ & 0.639 & $0.334~_{-0.173}^{+0.160}$ & $0.075~_{-0.039}^{+0.036}$ \\
%% -0.862 & -2.500 & $0.334~_{-0.097}^{+0.139}$ & $0.505~_{-0.173}^{+0.217}$ & 0.649 & $0.341~_{-0.174}^{+0.160}$ & $0.119~_{-0.061}^{+0.056}$ \\
%% -0.800 & -2.000 & $0.330~_{-0.095}^{+0.141}$ & $0.500~_{-0.176}^{+0.212}$ & 0.656 & $0.345~_{-0.176}^{+0.165}$ & $0.202~_{-0.103}^{+0.097}$ \\
%% -0.692 & -1.500 & $0.325~_{-0.095}^{+0.139}$ & $0.485~_{-0.172}^{+0.209}$ & 0.660 & $0.340~_{-0.176}^{+0.164}$ & $0.369~_{-0.191}^{+0.178}$ \\
%% -0.640 & -1.333 & $0.325~_{-0.097}^{+0.139}$ & $0.482~_{-0.173}^{+0.210}$ & 0.659 & $0.334~_{-0.174}^{+0.169}$ & $0.456~_{-0.238}^{+0.231}$ \\
%% -0.610 & -1.250 & $0.321~_{-0.095}^{+0.140}$ & $0.474~_{-0.169}^{+0.210}$ & 0.653 & $0.328~_{-0.177}^{+0.164}$ & $0.505~_{-0.272}^{+0.252}$ \\
%% -0.500 & -1.000 & $0.315~_{-0.093}^{+0.142}$ & $0.450~_{-0.160}^{+0.213}$ & 0.638 & $0.304~_{-0.176}^{+0.175}$ & $0.685~_{-0.396}^{+0.395}$ \\
%% -0.410 & -0.833 & $0.312~_{-0.095}^{+0.138}$ & $0.424~_{-0.147}^{+0.210}$ & 0.615 & $0.276~_{-0.177}^{+0.168}$ & $0.819~_{-0.526}^{+0.498}$ \\
%% -0.360 & -0.750 & $0.307~_{-0.093}^{+0.138}$ & $0.409~_{-0.136}^{+0.200}$ & 0.593 & $0.256~_{-0.176}^{+0.170}$ & $0.874~_{-0.601}^{+0.581}$ \\
%% -0.308 & -0.667 & $0.301~_{-0.092}^{+0.138}$ & $0.384~_{-0.124}^{+0.198}$ & 0.566 & $0.231~_{-0.174}^{+0.165}$ & $0.915~_{-0.689}^{+0.655}$ \\
%% -0.200 & -0.500 & $0.292~_{-0.090}^{+0.136}$ & $0.345~_{-0.109}^{+0.181}$ & 0.497 & $0.166~_{-0.162}^{+0.163}$ & $0.895~_{-0.871}^{+0.879}$ \\
%% -0.100 & -0.333 & $0.278~_{-0.086}^{+0.132}$ & $0.303~_{-0.092}^{+0.156}$ & 0.409 & $0.092~_{-0.092}^{+0.157}$ & $0.679~_{-0.679}^{+1.159}$ \\
%% -0.059 & -0.250 & $0.268~_{-0.083}^{+0.129}$ & $0.283~_{-0.085}^{+0.152}$ & 0.365 & $0.059~_{-0.059}^{+0.148}$ & $0.515~_{-0.515}^{+1.285}$ \\
%% -0.027 & -0.167 & $0.260~_{-0.081}^{+0.125}$ & $0.266~_{-0.077}^{+0.135}$ & 0.328 & $0.029~_{-0.029}^{+0.142}$ & $0.297~_{-0.297}^{+1.434}$ \\
%%  0.000 &  0.000 & $0.254~_{-0.079}^{+0.123}$ & $0.252~_{-0.073}^{+0.123}$ & 0.294 & $0.000~_{-0.000}^{+0.132}$ & $0.002~_{-0.002}^{+1.776}$ \\
%%  0.027 &  0.167 & $0.275~_{-0.086}^{+0.132}$ & $0.284~_{-0.084}^{+0.148}$ & 0.357 & $0.040~_{-0.040}^{+0.154}$ & $0.650~_{-0.650}^{+2.514}$ \\
%%  0.059 &  0.250 & $0.297~_{-0.093}^{+0.141}$ & $0.329~_{-0.099}^{+0.171}$ & 0.458 & $0.119~_{-0.119}^{+0.183}$ & $2.015~_{-2.015}^{+3.098}$ \\
%%  0.100 &  0.333 & $0.322~_{-0.099}^{+0.148}$ & $0.405~_{-0.135}^{+0.220}$ & 0.611 & $0.246~_{-0.184}^{+0.166}$ & $4.147~_{-3.103}^{+2.802}$ \\
%%  0.200 &  0.500 & $0.324~_{-0.096}^{+0.141}$ & $0.505~_{-0.181}^{+0.212}$ & 0.730 & $0.413~_{-0.177}^{+0.150}$ & $5.982~_{-2.559}^{+2.174}$ \\
%%  0.308 &  0.667 & $0.281~_{-0.082}^{+0.122}$ & $0.462~_{-0.159}^{+0.172}$ & 0.651 & $0.382~_{-0.144}^{+0.136}$ & $4.186~_{-1.574}^{+1.492}$ \\
%%  0.360 &  0.750 & $0.268~_{-0.079}^{+0.116}$ & $0.442~_{-0.154}^{+0.160}$ & 0.620 & $0.364~_{-0.135}^{+0.130}$ & $3.392~_{-1.253}^{+1.214}$ \\
%%  0.410 &  0.833 & $0.258~_{-0.075}^{+0.112}$ & $0.427~_{-0.147}^{+0.162}$ & 0.599 & $0.351~_{-0.130}^{+0.127}$ & $2.754~_{-1.022}^{+0.999}$ \\
%%  0.500 &  1.000 & $0.244~_{-0.072}^{+0.105}$ & $0.401~_{-0.137}^{+0.154}$ & 0.562 & $0.328~_{-0.121}^{+0.118}$ & $1.821~_{-0.671}^{+0.657}$ \\
%%  0.610 &  1.250 & $0.240~_{-0.070}^{+0.104}$ & $0.394~_{-0.133}^{+0.154}$ & 0.545 & $0.315~_{-0.119}^{+0.118}$ & $1.072~_{-0.403}^{+0.399}$ \\
%%  0.640 &  1.333 & $0.242~_{-0.071}^{+0.105}$ & $0.398~_{-0.136}^{+0.156}$ & 0.547 & $0.316~_{-0.121}^{+0.122}$ & $0.921~_{-0.352}^{+0.354}$ \\
%%  0.692 &  1.500 & $0.244~_{-0.071}^{+0.106}$ & $0.401~_{-0.136}^{+0.159}$ & 0.543 & $0.312~_{-0.120}^{+0.120}$ & $0.678~_{-0.261}^{+0.262}$ \\
%%  0.800 &  2.000 & $0.256~_{-0.075}^{+0.109}$ & $0.416~_{-0.138}^{+0.169}$ & 0.552 & $0.311~_{-0.127}^{+0.121}$ & $0.317~_{-0.129}^{+0.123}$ \\
%%  0.862 &  2.500 & $0.268~_{-0.078}^{+0.114}$ & $0.433~_{-0.142}^{+0.169}$ & 0.558 & $0.310~_{-0.130}^{+0.127}$ & $0.170~_{-0.072}^{+0.070}$ \\
%%  0.900 &  3.000 & $0.276~_{-0.080}^{+0.118}$ & $0.442~_{-0.144}^{+0.177}$ & 0.563 & $0.308~_{-0.134}^{+0.128}$ & $0.102~_{-0.044}^{+0.042}$ \\
%%  0.941 &  4.000 & $0.290~_{-0.084}^{+0.122}$ & $0.459~_{-0.149}^{+0.184}$ & 0.566 & $0.304~_{-0.140}^{+0.134}$ & $0.046~_{-0.021}^{+0.020}$ \\
%%  0.973 &  6.000 & $0.306~_{-0.081}^{+0.122}$ & $0.474~_{-0.150}^{+0.192}$ & 0.571 & $0.300~_{-0.150}^{+0.131}$ & $0.016~_{-0.008}^{+0.007}$ \\
%%  \hline
%% \end{tabular}
%% \caption{Expected (for background only, and for a SM-like Higgs signal) and observed 95\% C.L. upper limits (in pb), and best fit signal strength $r$ and corresponding best fit cross section for the combined $\tH+\ttH$ cross section times modified branching ratio for the combination of all three channels, for different values of $\Ct/\CV$ or the equivalent $\ft$ numbers.}
%% \label{tab:xslimits}
%% \end{table}

%% \begin{table}[h!]
%% \centering
%% \begin{tabular}{lr}
%%  \hline
%%   \threel\  & $r=1.44 {}_{-0.84} {}^{+0.91}$ \\
%%   \emu\     & $r=1.42 {}_{-1.03} {}^{+1.06}$ \\
%%   \mumu\    & $r=2.75 {}_{-1.11} {}^{+1.22}$ \\
%%   Combined  & $r=1.82 {}_{-0.69} {}^{+0.76}$ \\ 
%%   Expected  & $r=1.00 {}_{-0.65} {}^{+0.70}$ \\ 
%%  \hline
%% \end{tabular}
%% \caption{Best-fit signal strengths for a SM-like Higgs signal for the individual channels.}
%% \label{tab:sigstrengths}
%% \end{table}


%% \begin{figure} [!h]
%%  \centering
%%  \includegraphics[width=0.48\textwidth]{limits/xs_limits_K6.pdf}
%%  \includegraphics[width=0.48\textwidth]{limits/xs_limits_K6_split_cv.pdf}\\
%%  \includegraphics[width=0.48\textwidth]{limits/xs_limits_K6_alpha.pdf}
%%  \includegraphics[width=0.48\textwidth]{limits/xs_limits_K6_alpha_split_cv.pdf}
%% \caption{Expected (from background-only) and observed asymptotic limits on the combined $\tH+\ttH$ cross section times modified BR as a function of $\Ct/\CV$ (top) and $\mathrm{sign}(\Ct/\CV)\times\frac{\Ct^2}{(\Ct^2+\CV^2)}$ (bottom) for the combination of $3l$ channel, \mumu, and \emu\ channel.}
%% \label{fig:xs_limits_cv}
%% \end{figure}

%% \begin{figure} [!h]
%%  \centering
%%  \includegraphics[width=0.48\textwidth]{limits/xs_limits_K6_smexp.pdf}
%%  \includegraphics[width=0.48\textwidth]{limits/xs_limits_K6_smexp_split_cv.pdf}\\
%%  \includegraphics[width=0.48\textwidth]{limits/xs_limits_K6_alpha_smexp.pdf}
%%  \includegraphics[width=0.48\textwidth]{limits/xs_limits_K6_alpha_smexp_split_cv.pdf}
%% \caption{As Fig.~\ref{fig:xs_limits_cv} but calculating the expected limit on an Asimov dataset that includes SM-like \ttH\ and \tH\ signals.}
%% \label{fig:xs_limits_cv_smexp}
%% \end{figure}

%% \begin{figure} [!h]
%%  \centering
%%  \includegraphics[width=0.48\textwidth]{limits/xs_fits_K6.pdf}
%%  \includegraphics[width=0.48\textwidth]{limits/xs_fits_K6_split_cv.pdf}\\
%%  \includegraphics[width=0.48\textwidth]{limits/xs_fits_K6_alpha.pdf}
%%  \includegraphics[width=0.48\textwidth]{limits/xs_fits_K6_alpha_split_cv.pdf}
%% \caption{Best fit values of the combined $\tH+\ttH$ cross section times modified BR as a function of $\Ct/\CV$ (top) and $\mathrm{sign}(\Ct/\CV)\times\frac{\Ct^2}{(\Ct^2+\CV^2)}$ (bottom) for the combination of $3l$ channel, \mumu, and \emu\ channel.}
%% \label{fig:xs_fits_cv}
%% \end{figure}


%% \begin{figure} [!h]
%%  \centering
%%  \includegraphics[width=0.48\textwidth]{significances/significances.pdf} \\
%%  \includegraphics[width=0.48\textwidth]{significances/significances_alpha.pdf}
%% \caption{Observed and a priori expected significance of the fit result (in a background-only hypothesis) as a function of $\Ct/\CV$ (top) and \ft\ (bottom) for the combination of $3l$ channel, \mumu, and \emu\ channel.}
%% \label{fig:significances}
%% \end{figure}

%% % \begin{figure} [!h] %% FIXME: Put in appendix
%% %  \centering
%% %  \includegraphics[width=0.48\textwidth]{limits/limits_comb3_K4_cv_1p0.pdf}
%% %  \includegraphics[width=0.48\textwidth]{limits/limits_comb3_K5_cv_1p0.pdf}\\
%% %  \includegraphics[width=0.48\textwidth]{limits/limits_comb3_K4_cv_1p5.pdf}
%% %  \includegraphics[width=0.48\textwidth]{limits/limits_comb3_K5_cv_1p5.pdf}\\
%% %  \includegraphics[width=0.48\textwidth]{limits/limits_comb3_K4_cv_0p5.pdf}
%% %  \includegraphics[width=0.48\textwidth]{limits/limits_comb3_K5_cv_0p5.pdf}
%% % \caption{Expected asymptotic limits on the combined $\tH+\ttH$ signal strength as a function of \Ct\ for $\CV=1.0$, $\CV=1.5$, $\CV=0.5$ (top to bottom) for the combination of $3l$ channel, \mumu, and \emu\ channel, for the resolved model (left, with modified $\PH\to\gamma\gamma/\Z\gamma/\Pg\Pg$ branching ratios), and the unresolved model (right, with unmodified overall Higgs decay width.). (Note that the slight dip at $\Ct=0,\CV=0.5$ is the result of a bug, and not physical.)}
%% % \label{fig:r_limits_cv}
%% % \end{figure}

%% \begin{figure} [!h]
%%  \centering
%%  \includegraphics[width=0.8\textwidth]{limits/impacts/impacts1.pdf}\\
%%  \includegraphics[width=0.8\textwidth]{limits/impacts/impacts2.pdf}\\
%% \caption{Post-fit pulls and impacts of the 40 nuisance parameters with largest impacts for the fit on the observed data, for the $\Ct/\CV=-1.0$ hypothesis.}
%% \label{fig:impacts}
%% \end{figure}

%% \begin{figure} [!h]
%%  \centering
%%  \includegraphics[width=0.8\textwidth]{limits/impacts/sm/impacts1.pdf}\\
%%  \includegraphics[width=0.8\textwidth]{limits/impacts/sm/impacts2.pdf}\\
%% \caption{Post-fit pulls and impacts of the 40 nuisance parameters with largest impacts for the fit on the observed data, for the standard model ($\Ct/\CV=1.0$) hypothesis.}
%% \label{fig:impacts_sm}
%% \end{figure}

%% \begin{figure} [!h]
%%  \centering
%%  \includegraphics[width=0.8\textwidth]{limits/impacts/asimov/impacts1.pdf}\\
%%  \includegraphics[width=0.8\textwidth]{limits/impacts/asimov/impacts2.pdf}\\
%% \caption{Post-fit pulls and impacts of the 40 nuisance parameters with largest impacts for a fit to the Asimov dataset with fixed signal strength, for the $\Ct/\CV=-1.0$ hypothesis.}
%% \label{fig:impacts_asimov}
%% \end{figure}


